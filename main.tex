% File: main.tex
% Ho Chi Minh University of Transport Thesis template

\documentclass[a4paper,12pt]{report}

% =========================
% LOAD PACKAGES FIRST
% =========================
\usepackage{fontspec}
\usepackage[vietnamese]{babel}
\usepackage{graphicx}

\setmainfont{Times New Roman}
\setsansfont{Arial}
\setmonofont{Courier New}
\usepackage{url}
\urlstyle{rm}

\usepackage{geometry}
\usepackage{fancybox}
\usepackage{xcolor}
\usepackage{anyfontsize}
\usepackage{amsmath}
\usepackage{amssymb}
\usepackage{amsfonts}
\usepackage{enumitem}
\usepackage{float}
\usepackage{tikz}
\usetikzlibrary{arrows.meta, positioning, shapes.geometric, shapes.multipart}
\usepackage{placeins}
\usepackage{booktabs}
\usepackage{longtable}
\usepackage{titlesec}
\usepackage[acronym]{glossaries}
\usepackage{tocloft}
\usepackage{setspace}
\onehalfspacing
\usepackage{indentfirst}
\usepackage{caption}
\usepackage{ragged2e} 
\usepackage{array}
\usepackage[hidelinks]{hyperref}

% =========================
% GEOMETRY
% =========================
\geometry{
    a4paper,
    left=3cm,
    right=2cm,
    top=2cm,
    bottom=2cm
}

\setlength{\parindent}{1.25cm}

% =========================
% FORMAT CHAPTER, SECTION, SUBSECTION
% =========================

% Format CHAPTER - 14pt, in hoa, in đậm, căn giữa
\titleformat{\chapter}[hang]
{\normalfont\fontsize{14}{17}\bfseries\centering}
{\MakeUppercase{\chaptertitlename\ \thechapter.}}
{1em}
{\MakeUppercase}

\titlespacing*{\chapter}{0pt}{0pt}{20pt}

% CHAPTER không số (chapter*)
\titleformat{name=\chapter,numberless}[hang]
{\normalfont\fontsize{14}{17}\bfseries\centering}
{}
{0pt}
{\MakeUppercase}

\titlespacing*{\chapter}{0pt}{0pt}{20pt}

% Format SECTION - 13pt, in đậm
\titleformat{\section}
{\normalfont\fontsize{13}{16}\bfseries}
{\thesection}
{1em}
{}

% Format SUBSECTION - 13pt, in đậm, in nghiêng
\titleformat{\subsection}
{\normalfont\fontsize{13}{16}\bfseries\itshape}
{\thesubsection}
{1em}
{}

\makeglossaries

\captionsetup{
    labelfont=it,
    textfont=it
}

\captionsetup[table]{
    labelfont=it,
    textfont=it
}

\begin{document}

% ===== TRANG BÌA - KHÔNG ĐÁNH SỐ =====
\pagenumbering{gobble}
\begin{titlepage}
\begin{tikzpicture}[remember picture,overlay]
% Khung ngoài
\draw[line width=5pt, blue!50!black]
  ([shift={(1.5cm,-1.5cm)}]current page.north west) rectangle
  ([shift={(-1.5cm,1.5cm)}]current page.south east);
% Khung trong
\draw[line width=0.8pt, blue!50!black]
  ([shift={(1.8cm,-1.8cm)}]current page.north west) rectangle
  ([shift={(-1.8cm,1.8cm)}]current page.south east);
% Góc trang trí
\foreach \corner in {north west, north east, south west, south east}{
  \draw[line width=1pt] ([shift={(1.3cm,-1.3cm)}]current page.\corner) -- ++(1cm,0);
  \draw[line width=1pt] ([shift={(1.3cm,-1.3cm)}]current page.\corner) -- ++(0,-1cm);
}
\end{tikzpicture}

\begin{center}
\vspace*{0.3cm}
{\fontsize{13}{15}\selectfont\bfseries
TRƯỜNG ĐẠI HỌC GIAO THÔNG VẬN TẢI TP. HỒ CHÍ MINH\\
VIỆN CÔNG NGHỆ THÔNG TIN VÀ ĐIỆN, ĐIỆN TỬ}

\vspace{1.8cm}
\includegraphics[width=0.5\textwidth]{graphics/front/logo_uth.png}

\vspace{1.8cm}
{\fontsize{22}{26}\selectfont\bfseries BÁO CÁO ĐỒ ÁN}

\vspace{0.6cm}
{\fontsize{16}{19}\selectfont\bfseries MÔN: THƯƠNG MẠI ĐIỆN TỬ}

\vspace{2cm}
{\fontsize{14}{17}\selectfont\bfseries TÊN ĐỀ TÀI:}\\[0.3cm]
{\fontsize{14}{17}\selectfont\bfseries
XÂY DỰNG WEBSITE BÁN MỸ PHẨM KẾT HỢP AI HỖ TRỢ\\
TƯ VẤN DA MẶT TRỊ MỤN VÀ CHATBOT GỢI Ý SẢN PHẨM}

\vspace{0.8cm}
\begin{flushleft}
\hspace{1.5cm}{\fontsize{13}{15}\selectfont\bfseries\underline{SINH VIÊN THỰC HIỆN:}}\\[0.4cm]
\hspace{1.5cm}{\fontsize{13}{15}\selectfont\bfseries
1. Nhóm trưởng: Lê Tuấn Khang -- MSSV: 2251120420}\\[0.25cm]
\hspace{1.5cm}{\fontsize{13}{15}\selectfont\bfseries
2. Thành viên: Hồ Huỳnh Nhu -- MSSV: 2251120433}\\[0.25cm]
\hspace{1.5cm}{\fontsize{13}{15}\selectfont\bfseries
3. Thành viên: Lê Nguyễn Minh Phúc -- MSSV: 2251120040}\\[0.25cm]
\hspace{1.5cm}{\fontsize{13}{15}\selectfont\bfseries
4. Thành viên: Bùi Trọng Vũ -- MSSV: 2251120132}\\[0.25cm]
\hspace{1.5cm}{\fontsize{13}{15}\selectfont\bfseries
5. Thành viên: Lê Hà Chí Tâm -- MSSV: 2251320031}
\end{flushleft}

\vspace{0.6cm}
\begin{flushleft}
\hspace{1.5cm}{\fontsize{13}{15}\selectfont\bfseries\underline{GIẢNG VIÊN HƯỚNG DẪN:} ThS. Trần Vũ Đại}
\end{flushleft}

\vspace{1.5cm}
{\fontsize{13}{15}\selectfont\bfseries Thành phố Hồ Chí Minh, tháng 2 năm 2026}
\end{center}
\end{titlepage}

\clearpage

% ===== MỤC LỤC - ĐÁNH SỐ LA MÃ (i, ii, iii...) =====
\begin{spacing}{1.5}
\chapter*{LỜI MỞ ĐẦU}
\addcontentsline{toc}{chapter}{LỜI MỞ ĐẦU}

Cùng với sự phát triển mạnh mẽ của công nghệ số ở Việt Nam, thương mại điện tử không chỉ dừng lại ở việc mua bán trực tuyến
mà còn được tích hợp nhiều công nghệ thông minh nhằm nâng cao trải nghiệm người dùng.
Đối với các cửa hàng tư nhân, việc xây dựng một website thương mại điện tử riêng
giúp chủ động trong hoạt động kinh doanh, quản lý dữ liệu và chăm sóc khách hàng hiệu quả hơn.

Đề tài này tập trung phân tích và thiết kế hệ thống website thương mại điện tử
phục vụ cho một cửa hàng tư nhân, với hai nhóm người dùng chính là người mua và người bán.
Bên cạnh các chức năng mua bán cơ bản, hệ thống còn tích hợp các chức năng hỗ trợ và trí tuệ nhân tạo
như chatbot chăm sóc khách hàng và AI tư vấn sản phẩm, góp phần nâng cao chất lượng dịch vụ.

\chapter*{LỜI CÁM ƠN}


Nhóm xin gửi lời cảm ơn chân thành đến ThS. Trần Vũ Đại, giảng viên phụ trách học phần Thương mại điện tử, đã tận tình hướng dẫn, định hướng và hỗ trợ nhóm trong suốt quá trình thực hiện đề tài “Xây dựng website bán mỹ phẩm kết hợp AI hỗ trợ tư vấn da mặt trị mụn và chatbot gợi ý sản phẩm”.

Với sự hướng dẫn chi tiết, những góp ý chuyên môn sâu sắc cùng sự tận tâm của thầy trong từng giai đoạn thực hiện, nhóm đã có thể từng bước hoàn thiện hệ thống website cả về mặt kỹ thuật lẫn nội dung ứng dụng AI vào tư vấn chăm sóc da và đề xuất sản phẩm phù hợp cho người dùng. Những định hướng thực tế của thầy không chỉ giúp nhóm hoàn thành đề tài đúng tiến độ mà còn giúp nhóm nâng cao tư duy thiết kế hệ thống thương mại điện tử gắn với công nghệ AI trong thực tiễn.

Nhóm xin bày tỏ lòng biết ơn sâu sắc đến thầy Trần Vũ Đại và kính chúc thầy nhiều sức khỏe, thành công trong công tác giảng dạy và khoa học.




\clearpage

\chapter*{Bảng phân chia công việc}
\end{spacing}

\pagenumbering{roman}
\setcounter{page}{1}

% Tạm tắt format của tocloft
\renewcommand{\cfttoctitlefont}{\hspace*{\fill}\fontsize{14}{17}\selectfont\bfseries\MakeUppercase}
\renewcommand{\cftaftertoctitle}{\hspace*{\fill}}

\renewcommand{\cftchappresnum}{CHƯƠNG~}
\renewcommand{\cftchapaftersnum}{:}
\renewcommand{\cftchapfont}{\bfseries}
\renewcommand{\cftchappagefont}{\bfseries}
\setlength{\cftchapnumwidth}{6em}

\setlength{\cftaftertoctitleskip}{5pt}
\setlength{\cftbeforechapskip}{10pt}
\setlength{\cftbeforesecskip}{3pt}
\setlength{\cftbeforesubsecskip}{2pt}

\tableofcontents
\clearpage

% ===== DANH MỤC HÌNH ẢNH =====
\addcontentsline{toc}{chapter}{DANH MỤC HÌNH ẢNH}

\renewcommand{\cftloftitlefont}{\hspace*{\fill}\fontsize{14}{17}\selectfont\bfseries}
\renewcommand{\cftafterloftitle}{\hspace*{\fill}}
\renewcommand{\listfigurename}{DANH MỤC HÌNH ẢNH}

\renewcommand{\cftfigfont}{\normalfont}
\renewcommand{\cftfigpagefont}{\normalfont}
\renewcommand{\cftfigpresnum}{Hình~}
\renewcommand{\cftfigaftersnum}{:~}
\setlength{\cftfignumwidth}{5em}
\setlength{\cftfigindent}{0pt}

\setlength{\cftbeforeloftitleskip}{-20pt}
\setlength{\cftafterloftitleskip}{10pt}
\setlength{\cftbeforefigskip}{5pt}

\listoffigures
\clearpage

% ===== DANH MỤC BẢNG =====
\addcontentsline{toc}{chapter}{DANH MỤC BẢNG}

\renewcommand{\cftlottitlefont}{\hspace*{\fill}\fontsize{14}{17}\selectfont\bfseries}
\renewcommand{\cftafterlottitle}{\hspace*{\fill}}
\renewcommand{\listtablename}{DANH MỤC BẢNG}

\renewcommand{\cfttabfont}{\normalfont}
\renewcommand{\cfttabpagefont}{\normalfont}
\renewcommand{\cfttabpresnum}{Bảng~}
\renewcommand{\cfttabaftersnum}{:~}
\setlength{\cftbeforetabskip}{0pt}
\setlength{\cfttabnumwidth}{5em}
\setlength{\cfttabindent}{0pt}

\setlength{\cftbeforelottitleskip}{-20pt}
\setlength{\cftafterlottitleskip}{10pt}
\setlength{\cftbeforetabskip}{5pt}

\listoftables
\clearpage

% ===== DANH MỤC TỪ VIẾT TẮT =====
\addcontentsline{toc}{chapter}{DANH MỤC TỪ VIẾT TẮT}
\chapter*{DANH MỤC TỪ VIẾT TẮT}

\begin{table}[H]
\centering
\renewcommand{\arraystretch}{1.3}
\begin{tabular}{|p{4cm}|p{9cm}|}
\hline
\textbf{Từ viết tắt} & \textbf{Nghĩa đầy đủ} \\ \hline
AI & Artificial Intelligence \\ \hline
API & Application Programming Interface \\ \hline
CNN & Convolutional Neural Network \\ \hline
CRUD & Create, Read, Update, Delete \\ \hline
HTML & HyperText Markup Language \\ \hline
HTTP & HyperText Transfer Protocol \\ \hline
JS & JavaScript \\ \hline
JSON & JavaScript Object Notation \\ \hline
JWT & JSON Web Token \\ \hline
RAG & Retrieval-Augmented Generation \\ \hline
REST & Representational State Transfer \\ \hline
SQL & Structured Query Language \\ \hline
SSR & Server-Side Rendering \\ \hline
UI & User Interface \\ \hline
UX & User Experience \\ \hline
VNPay & Vietnam Payment Gateway \\ \hline
\end{tabular}
\end{table}

\clearpage

% ===== NỘI DUNG CHÍNH - ĐÁNH SỐ Ả-RẬP (1, 2, 3...) =====
\pagenumbering{arabic}
\setcounter{page}{1}

\begin{spacing}{1.5}
% Các chapter
\chapter{TỔNG QUAN VỀ HỆ THỐNG}
\label{chap:chap1-system-overview}

\section{Đặt vấn đề}
Trong bối cảnh công nghệ thông tin và thương mại điện tử phát triển mạnh mẽ, hành vi mua sắm của người tiêu dùng đang dần chuyển dịch từ hình thức truyền thống sang mua sắm trực tuyến. Đặc biệt, ngành mỹ phẩm là một trong những lĩnh vực có tốc độ tăng trưởng nhanh, nhu cầu cao và khả năng tiếp cận khách hàng rộng rãi thông qua nền tảng số.

Tuy nhiên, trên thực tế, không phải website bán mỹ phẩm nào cũng đáp ứng tốt các yêu cầu về giao diện thân thiện, trải nghiệm người dùng, khả năng quản lý sản phẩm cũng như tính an toàn và thuận tiện trong quá trình mua hàng. Nhiều website còn tồn tại các hạn chế như bố cục chưa hợp lý, tốc độ tải trang chậm, thiếu thông tin sản phẩm rõ ràng hoặc chưa tối ưu cho các thiết bị di động, gây ảnh hưởng đến quyết định mua sắm của khách hàng.

Xuất phát từ thực tế đó, việc thiết kế một website bán mỹ phẩm chuyên nghiệp, hiện đại, dễ sử dụng và đáp ứng đầy đủ các chức năng cần thiết là yêu cầu quan trọng, không chỉ giúp doanh nghiệp tiếp cận khách hàng hiệu quả hơn mà còn nâng cao uy tín thương hiệu và khả năng cạnh tranh trên thị trường trực tuyến. Vì vậy, đề tài ``Thiết kế website bán mỹ phẩm'' được lựa chọn nhằm nghiên cứu và xây dựng một hệ thống website phù hợp với nhu cầu thực tế hiện nay.

\section{Dữ liệu thực tế }
Năm 2023, doanh thu toàn cầu của ngành mỹ phẩm và chăm sóc cá nhân ước đạt khoảng 625,7 tỷ USD. Thị trường này đang tăng trưởng đều đặn với tốc độ $\sim$3,3\%/năm giai đoạn 2023--2028. Một số báo cáo khác cho rằng quy mô thị trường năm 2023 dao động từ 617,2 đến 626 tỷ USD và dự kiến tăng 9\% lên khoảng 670,8 tỷ USD vào năm 2024.
	Tại Việt Nam, theo Statista và các khảo sát quốc tế, thị trường mỹ phẩm và chăm sóc cá nhân năm 2023 đạt khoảng 2,4 tỷ USD và dự kiến tăng lên khoảng 2,74 tỷ USD vào năm 2025. Tốc độ tăng trưởng trung bình khoảng 3,2--3,3\%/năm trong giai đoạn 2023--2027. Trong đó, phân khúc chăm sóc cá nhân chiếm thị phần lớn nhất, ước khoảng 1,20 tỷ USD vào năm 2025, trong khi các sản phẩm chăm sóc da và trang điểm tăng trưởng nhanh nhờ sự quan tâm ngày càng cao của người tiêu dùng trẻ.
	
\subsection*{Hành vi người tiêu dùng và xu hướng mua sắm trực tuyến}
	
	Trên toàn cầu, đến năm 2024, khoảng 26\% doanh thu ngành mỹ phẩm đến từ kênh trực tuyến và tỷ lệ này được dự báo sẽ tiếp tục tăng. Riêng tại Mỹ, thương mại điện tử chiếm khoảng 41\% doanh thu mỹ phẩm. Người tiêu dùng, đặc biệt là thế hệ Gen Z, có xu hướng tìm hiểu kỹ sản phẩm thông qua nền tảng số và chịu ảnh hưởng mạnh từ mạng xã hội.
	Tại Việt Nam, hơn 60\% phụ nữ sử dụng sản phẩm chăm sóc da hằng ngày. Các sản phẩm phổ biến bao gồm sữa rửa mặt (49\%), nước hoa (41\%), kem chống nắng (31\%) và kem dưỡng (25\%). Thị trường chủ yếu do các thương hiệu nước ngoài chiếm lĩnh với hơn 90\% thị phần, trong đó thương hiệu Hàn Quốc dẫn đầu.

\subsection*{Báo cáo doanh thu thị trường Việt Nam trong tương lai}
	Kênh mua sắm trực tuyến tại Việt Nam tăng trưởng mạnh với mức tăng 47,6\% năm 2022 và 52,2\% năm 2023, đạt khoảng 37,7 nghìn tỷ đồng (xấp xỉ 1,5 tỷ USD). Đến năm 2023, doanh thu online chiếm khoảng 19\% tổng thị trường mỹ phẩm, tăng mạnh so với mức 8\% năm 2018.
	Sự phát triển của các sàn thương mại điện tử như Shopee, Lazada và TikTok Shop cùng xu hướng livestream bán hàng đã thay đổi đáng kể thói quen mua sắm của người tiêu dùng, đặc biệt là giới trẻ.

\section{Công cụ sử dụng}
Trong đề tài xây dựng website bán mỹ phẩm kết hợp AI hỗ trợ tư vấn da và gợi ý sản phẩm, các công cụ và công nghệ chính được sử dụng gồm:

\textbf{IDE và lập trình:}
Visual Studio Code.

\textbf{Frontend:}
Nuxt.js.

\textbf{Backend:}
ExpressJS.

\textbf{Cơ sở dữ liệu:}
Microsoft SQL Server.

\textbf{Trí tuệ nhân tạo:}
Mô hình CNN dùng để nhận diện mụn từ hình ảnh khuôn mặt.  
Qdrant dùng để lưu trữ vector embeddings phục vụ truy xuất ngữ nghĩa.  
Ollama dùng để triển khai chatbot tư vấn và gợi ý sản phẩm.

\textbf{Quản lý và kiểm thử:}
GitHub, Postman.

\textbf{Debug và chạy hệ thống:}
Chrome, Edge, npm start và npm run build.

\section{Công cụ sử dụng}
Các công cụ chính được sử dụng trong đề tài Thiết kế website bán mỹ phẩm
	
	\textbf{IDE và lập trình:} Visual Studio Code.
	
	\textbf{Frontend:} ReactJS, React Router, Axios, Ant Design, Redux, SASS cùng các thư viện hỗ trợ như React Slick, Recharts, SweetAlert2 và jwt-decode.
	
	\textbf{Backend:} Node.js (18+), ExpressJS, bcrypt, jsonwebtoken, dotenv, cors và multer.
	
	\textbf{Cơ sở dữ liệu:} SQL Server.
	
	\textbf{Quản lý và triển khai:} Git và GitHub.
	
	\textbf{Debug và chạy hệ thống:} Chrome, Edge, npm start và npm run build.

\section{Sơ đồ tuần tự (Sequence Diagram)}

\subsection{Sơ đồ tuần tự xử lý thanh toán}
\begin{figure}[H]
    \centering
    \includegraphics[width=0.8\textwidth]{graphics/main/chapter2/section2/sequence_vnpay.jpg}
    \caption{Sơ đồ tuần tự xử lý thanh toán}
    \label{fig:sequence_vnpay}
\end{figure}
\subsubsection{Mục đích}
Sequence diagram này mô tả quy trình thanh toán trực tuyến của hệ thống với hai phương thức: thanh toán khi nhận hàng (COD) và thanh toán qua cổng VNPay. Sơ đồ thể hiện tương tác giữa khách hàng, hệ thống backend và cổng thanh toán VNPay trong suốt quá trình đặt hàng và thanh toán.

\subsubsection{Mô tả luồng hoạt động}
Luồng thanh toán COD
\begin{enumerate}[noitemsep]
    \item Khách hàng chọn phương thức thanh toán COD
    \item Khách hàng xác nhận thanh toán COD
    \item Backend tạo đơn hàng mới
    \item Backend hiển thị kết quả thanh toán COD cho khách hàng
\end{enumerate}

Luồng thanh toán VNPay
\begin{enumerate}[noitemsep]
    \item Khách hàng chọn phương thức thanh toán VNPay
    \item Backend chuyển hướng khách hàng đến cổng thanh toán VNPay
    \item VNPay hiển thị giao diện thanh toán cho khách hàng
    \item Khách hàng thực hiện thanh toán trên VNPay
    \item VNPay trả kết quả giao dịch (thành công/thất bại) về Backend
    \item Backend xử lý kết quả:
    \begin{itemize}[noitemsep]
        \item Nếu thanh toán thành công: Backend hiển thị kết quả thành công
        \item Nếu thanh toán thất bại: Backend hiển thị kết quả thất bại
    \end{itemize}
\end{enumerate}

Xem lịch sử đơn hàng
\begin{enumerate}[noitemsep]
    \item Sau khi thanh toán, khách hàng có thể xem lịch sử đơn hàng
    \item Backend hiển thị danh sách đơn hàng của khách hàng
\end{enumerate}
\subsection{Sơ đồ tuần tự phân tích da mặt}
\begin{figure}[H]
    \centering
    \includegraphics[width=0.8\textwidth]{graphics/main/chapter2/section2/sequence_acne.jpg}
    \caption{Sơ đồ tuần tự phân tích da mặt}
    \label{fig:sequence_acne}
\end{figure}
\subsubsection{Mục đích}
Sequence diagram này mô tả quy trình phân tích da mặt sử dụng trí tuệ nhân tạo. Sơ đồ thể hiện luồng xử lý từ khi người dùng upload ảnh khuôn mặt, qua các bước nhận diện khuôn mặt bằng MediaPipe, phân tích mụn bằng mô hình CNN, đến khi trả về kết quả phân tích cùng lời khuyên và gợi ý sản phẩm phù hợp.

\subsubsection{Mô tả luồng hoạt động}

\begin{enumerate}[noitemsep]
    \item Người dùng upload ảnh khuôn mặt lên hệ thống Backend
    \item Backend gửi ảnh đến MediaPipe để nhận diện khuôn mặt và các điểm landmark
    \item MediaPipe trả về tọa độ khuôn mặt và các điểm landmark
    \item Backend cắt từng vùng da mặt (Trán, Má, Mũi, Cằm) dựa trên landmark
    \item Backend gửi từng vùng da đã cắt đến mô hình CNN để phân tích mụn
    \item CNN Model phát hiện mụn và trả về kết quả (số lượng, vị trí, mức độ)
    \item Backend tổng hợp kết quả từ tất cả các vùng da
    \item Backend trả về kết quả phân tích chi tiết cho người dùng
    \item Backend đưa ra lời khuyên chăm sóc da dựa trên kết quả phân tích
    \item Backend gợi ý các sản phẩm phù hợp với tình trạng da của người dùng
\end{enumerate}

%Seque admin quản lýđánh giá sản phẩm

\subsection{Sơ đồ tuần tự Admin quản lý đánh giá sản phẩm}
\begin{figure}[H]
    \centering
    \includegraphics[width=0.8\textwidth]{graphics/main/chapter2/section2/sequead1.jpg}
    \caption{Admin quản lý đánh giá}
    \label{fig:sequence_adreview}
\end{figure}
 %Mô tả admin đánh giá sản phẩm

\textbf{Mục đích:}

Sơ đồ tuần tự mô tả quá trình Admin tương tác với hệ thống để quản lý đánh giá sản phẩm, bao gồm việc xem danh sách đánh giá, lọc theo tiêu chí và thực hiện xóa đánh giá khi cần thiết. Quá trình này đảm bảo Admin có thể kiểm soát nội dung đánh giá hiển thị trên website.

\textbf{Luồng hoạt động:}

Admin bắt đầu bằng việc mở trang quản lý đánh giá.

\begin{itemize}
    \item Trang quản lý đánh giá gửi yêu cầu lấy danh sách đánh giá đến Hệ thống xử lý.
    \item Hệ thống xử lý thực hiện truy vấn dữ liệu từ cơ sở dữ liệu đánh giá.
    \item Dữ liệu đánh giá được trả về cho Hệ thống xử lý.
    \item Hệ thống gửi kết quả về Trang quản lý để hiển thị danh sách đánh giá cho Admin.
\end{itemize}

\textbf{Trường hợp lọc đánh giá:}

\begin{itemize}
    \item Admin nhập tiêu chí lọc.
    \item Trang quản lý gửi yêu cầu lọc đến Hệ thống xử lý.
    \item Hệ thống truy vấn dữ liệu theo tiêu chí đã nhập.
    \item Kết quả lọc được trả về và danh sách đánh giá được cập nhật trên giao diện.
\end{itemize}

\textbf{Trường hợp xóa đánh giá:}

\begin{itemize}
    \item Admin chọn một đánh giá và nhấn nút ``Xóa''.
    \item Hệ thống hiển thị thông báo xác nhận.
    \item Nếu Admin chọn ``Đồng ý'', Trang quản lý gửi yêu cầu xóa (ID đánh giá) đến Hệ thống xử lý.
    \item Hệ thống xử lý yêu cầu xóa dữ liệu trong cơ sở dữ liệu.
    \item Sau khi xóa thành công, hệ thống trả về kết quả và cập nhật lại danh sách đánh giá.
    \item Giao diện hiển thị thông báo thành công và làm mới danh sách.
\end{itemize}

Quy trình kết thúc khi hệ thống hoàn tất cập nhật và hiển thị dữ liệu mới nhất cho Admin.

%Mô tả hoạt động khách hàng đánh giá sản phẩm

\subsection{Sơ đồ tuần tự đánh giá sản phẩm}
\begin{figure}[H]
    \centering
    \includegraphics[width=0.8\textwidth]{graphics/main/chapter2/section2/sequekh1.jpg}
    \caption{Khách hàng đánh giá sản phẩm}
    \label{fig:sequence_adreview}
\end{figure}
%Mô tả hoạt động khách hàng đánh giá sản phẩm
\subsubsection{Mô tả hoạt động Khách hàng -- Sơ đồ tuần tự đánh giá sản phẩm}

\textbf{Mục đích:}

Sơ đồ tuần tự mô tả quá trình khách hàng thực hiện đánh giá sản phẩm sau khi mua hàng. Quy trình bao gồm kiểm tra điều kiện hợp lệ (đã nhận hàng), nhập nội dung đánh giá, xử lý dữ liệu và lưu trữ vào hệ thống.

\textbf{Luồng hoạt động:}

Khách hàng bắt đầu bằng việc chọn chức năng đánh giá tại trang chi tiết sản phẩm.

\begin{itemize}
    \item Trang chi tiết sản phẩm gửi yêu cầu kiểm tra (UserID, ProductID) đến Hệ thống xử lý.
    \item Hệ thống xử lý kiểm tra trạng thái đơn hàng trong cơ sở dữ liệu.
    \item Nếu đơn hàng ở trạng thái đã nhận hàng, hệ thống trả kết quả hợp lệ và hiển thị form nhập đánh giá.
\end{itemize}

Khách hàng nhập số sao và nội dung bình luận, sau đó nhấn nút ``Gửi đánh giá''.

\begin{itemize}
    \item Trang chi tiết sản phẩm truyền dữ liệu đánh giá đến Hệ thống xử lý.
    \item Hệ thống kiểm tra tính hợp lệ của nội dung (spam, từ cấm, thiếu thông tin,...).
\end{itemize}

\textbf{Trường hợp nội dung hợp lệ:}

\begin{itemize}
    \item Hệ thống lưu thông tin đánh giá mới vào cơ sở dữ liệu.
    \item Cập nhật điểm trung bình của sản phẩm.
    \item Trả về kết quả lưu thành công.
    \item Giao diện hiển thị thông báo ``Cảm ơn bạn đã đánh giá''.
\end{itemize}

\textbf{Trường hợp nội dung không hợp lệ:}

\begin{itemize}
    \item Hệ thống không lưu dữ liệu.
    \item Giao diện hiển thị thông báo lỗi và yêu cầu khách hàng nhập lại nội dung.
\end{itemize}


\chapter{PHÂN TÍCH THIẾT KẾ HỆ THỐNG}
\label{chap:chap4-system-analysis-and-design}

\section{Đặc tả yêu cầu hệ thống}

\subsection{Yêu cầu chức năng}

Hệ thống website thương mại điện tử được xây dựng nhằm phục vụ hoạt động bán hàng trực tuyến cho một cửa hàng tư nhân. 
Hệ thống cung cấp các chức năng cho hai đối tượng sử dụng chính là người mua và người bán, đảm bảo hỗ trợ đầy đủ các nghiệp vụ 
mua bán, thanh toán, quản lý đơn hàng và hỗ trợ khách hàng.

\textbf{a) Sơ đồ phân rã chức năng (BFD)}

Sơ đồ phân rã chức năng (Business Function Decomposition – BFD) mô tả các chức năng nghiệp vụ chính của hệ thống website thương mại điện tử.
Các chức năng được phân rã từ mức tổng quát xuống các chức năng chi tiết, tương ứng với từng vai trò người dùng trong hệ thống.

\begin{figure}[H]
    \centering
    \includegraphics[width=0.95\textwidth]{graphics/front/BFDtm.png}
    \caption{Sơ đồ phân rã chức năng (BFD) của website thương mại điện tử}
    \label{fig:bfd}
\end{figure}

\textbf{b) Mô tả ngắn gọn các nhóm chức năng}

\begin{itemize}
  \item \textbf{Quản lý người mua}: Cho phép người mua đăng ký, đăng nhập, duyệt và tìm kiếm sản phẩm, quản lý giỏ hàng, đặt hàng, theo dõi đơn hàng và đánh giá sản phẩm.
  \item \textbf{Quản lý thanh toán}: Hỗ trợ thanh toán trực tuyến qua cổng VNPAY và thanh toán trực tiếp khi nhận hàng.
  \item \textbf{Quản lý giao hàng}: Hỗ trợ người bán chuẩn bị giao hàng và cập nhật trạng thái vận chuyển cho đơn hàng.
  \item \textbf{Hỗ trợ và AI}: Cung cấp các chức năng hỗ trợ khách hàng như chatbot AI chăm sóc khách hàng, tiếp nhận khiếu nại, thắc mắc và AI tư vấn da mặt gợi ý sản phẩm phù hợp.
  \item \textbf{Quản lý người bán}: Cho phép người bán quản lý danh mục, sản phẩm, đơn hàng, đánh giá của khách hàng, theo dõi thống kê doanh thu thông qua dashboard và quản lý người dùng.
\end{itemize}

\subsection{Yêu cầu phi chức năng}

Bên cạnh các yêu cầu chức năng, hệ thống cần đáp ứng các yêu cầu phi chức năng sau:

\begin{itemize}
  \item \textbf{Tính bảo mật}: Đảm bảo an toàn thông tin người dùng, đặc biệt là dữ liệu tài khoản và thông tin thanh toán.
  \item \textbf{Hiệu năng}: Hệ thống có thời gian phản hồi nhanh, đáp ứng tốt khi nhiều người dùng truy cập đồng thời.
  \item \textbf{Tính khả dụng}: Giao diện thân thiện, dễ sử dụng đối với cả người mua và người bán.
  \item \textbf{Tính mở rộng}: Hệ thống có khả năng mở rộng và nâng cấp thêm các chức năng trong tương lai.
  \item \textbf{Độ tin cậy}: Hệ thống hoạt động ổn định, hạn chế lỗi và đảm bảo dữ liệu không bị mất mát.
\end{itemize}

\section{Xây dựng Use Case Diagram}
\subsection{Use case Diagram tổng quát}
\begin{figure}[H]
    \centering
    \includegraphics[scale=0.3]{graphics/main/chapter2/section2/usecasetq.jpg}
    \caption{Use case Diagram tổng quát}
    \label{fig:usecase_overall}
\end{figure}
Các Actor trong hệ thống bao gồm:
\begin{itemize}[noitemsep]
    \item \textbf{UserBase}: 
    Là actor cơ sở đại diện cho tất cả người dùng của hệ thống. UserBase thực hiện các chức năng xác thực và quản lý tài khoản bao gồm: đăng ký, đăng nhập, đăng xuất và quản lý thông tin cá nhân.

    \item \textbf{Customer}: 
    Là actor kế thừa từ UserBase, đại diện cho khách hàng đã đăng nhập hệ thống. Customer có thể thực hiện các chức năng: xem giới thiệu cửa hàng, xem sản phẩm và danh mục, quản lý giỏ hàng, quản lý đơn hàng, thanh toán, đánh giá sản phẩm, quản lý địa chỉ liên hệ, liên hệ cửa hàng, sử dụng chatbot AI tư vấn và AI phân tích khuôn mặt để gợi ý sản phẩm phù hợp.

    \item \textbf{Guest}: 
    Là actor đại diện cho người dùng chưa đăng nhập. Guest chỉ có thể thực hiện các chức năng cơ bản như: xem giới thiệu cửa hàng, xem sản phẩm và danh mục.

    \item \textbf{Admin}: 
    Là actor kế thừa từ UserBase, đại diện cho quản trị viên hệ thống. Admin thực hiện các chức năng quản trị bao gồm: quản lý danh mục, quản lý sản phẩm, quản lý đơn hàng, quản lý người dùng, quản lý vận chuyển, quản lý đánh giá, quản lý khuyến mãi và quản lý thông tin cửa hàng.
\end{itemize}

%----------------------------------------------------------------Đăng nhập đăng ký---------------
\subsection{Use case Diagram chức năng xác thực}
\begin{figure}[H]
    \centering
    \includegraphics[width=0.8\textwidth]{graphics/main/chapter2/section2/usecase_authen.jpg}
    \caption{Use case Diagram chức năng xác thực}
    \label{fig:usecase_user}
\end{figure}

%đặc tả Đăng ký tài khoản
\subsubsection{Đặc tả Use Case Đăng ký tài khoản}

\begin{table}[H]
    \caption{Đặc tả Use Case Đăng ký tài khoản}
\centering
\renewcommand{\arraystretch}{1.3}
\begin{tabular}{|p{4cm}|p{10cm}|}
\hline
\textbf{Thuộc tính} & \textbf{Nội dung} \\ \hline

Tên Use Case & Đăng ký tài khoản \\ \hline

Tác nhân & UserBase (Customer, Admin) \\ \hline

Mô tả & 
Cho phép người dùng tạo tài khoản mới để sử dụng hệ thống. \\ \hline

Điều kiện tiên quyết & 
- Người dùng chưa có tài khoản. \\ \hline

Hậu điều kiện & 
Tài khoản mới được tạo trong cơ sở dữ liệu. \\ \hline

Luồng chính & 
1. Người dùng mở trang đăng ký. \newline
2. Hệ thống hiển thị form đăng ký. \newline
3. Người dùng nhập thông tin (email, mật khẩu, họ tên, số điện thoại). \newline
4. Người dùng nhấn đăng ký. \newline
5. Hệ thống kiểm tra validation. \newline
6. Hệ thống kiểm tra email đã tồn tại. \newline
7. Hệ thống mã hóa mật khẩu. \newline
8. Hệ thống lưu tài khoản mới vào cơ sở dữ liệu. \newline
9. Hệ thống hiển thị thông báo thành công và chuyển đến trang đăng nhập. \\ \hline

Luồng mở rộng (Extend) & 
Không có. \\ \hline

Luồng thay thế & 
- Thông tin không hợp lệ (thiếu trường bắt buộc, sai định dạng email) → hệ thống hiển thị lỗi validation. \newline
- Email đã tồn tại → hệ thống hiển thị "Email đã được sử dụng". \\ \hline

\end{tabular}

\end{table}

%đặc tả Đăng nhập
\subsubsection{Đặc tả Use Case Đăng nhập}

\begin{table}[H]
    \caption{Đặc tả Use Case Đăng nhập}
\centering
\renewcommand{\arraystretch}{1.3}
\begin{tabular}{|p{4cm}|p{10cm}|}
\hline
\textbf{Thuộc tính} & \textbf{Nội dung} \\ \hline

Tên Use Case & Đăng nhập \\ \hline

Tác nhân & UserBase (Customer, Admin) \\ \hline

Mô tả & 
Cho phép người dùng đăng nhập vào hệ thống bằng email và mật khẩu để được cấp Access Token. \\ \hline

Điều kiện tiên quyết & 
- Người dùng đã có tài khoản hợp lệ. \\ \hline

Hậu điều kiện & 
Người dùng được cấp Access Token và truy cập được các chức năng của hệ thống. \\ \hline

Luồng chính & 
1. Người dùng mở trang đăng nhập. \newline
2. Hệ thống hiển thị form đăng nhập. \newline
3. Người dùng nhập email và mật khẩu. \newline
4. Người dùng nhấn đăng nhập. \newline
5. Hệ thống kiểm tra validation. \newline
6. Hệ thống tìm tài khoản theo email. \newline
7. Hệ thống so sánh mật khẩu. \newline
8. Hệ thống tạo Access Token. \newline
9. Hệ thống trả về token và thông tin user. \newline
10. Giao diện lưu token vào bộ nhớ và chuyển đến trang chủ. \\ \hline

Luồng mở rộng (Extend) & 
Không có. \\ \hline

Luồng thay thế & 
- Thông tin không hợp lệ → hệ thống hiển thị lỗi validation. \newline
- Email không tồn tại → hệ thống hiển thị "Email không tồn tại". \newline
- Mật khẩu không đúng → hệ thống hiển thị "Mật khẩu không đúng". \newline
- Tài khoản bị khóa → hệ thống thông báo liên hệ admin. \\ \hline

\end{tabular}

\end{table}

%đặc tả Đăng xuất
\subsubsection{Đặc tả Use Case Đăng xuất}

\begin{table}[H]
    \caption{Đặc tả Use Case Đăng xuất}
\centering
\renewcommand{\arraystretch}{1.3}
\begin{tabular}{|p{4cm}|p{10cm}|}
\hline
\textbf{Thuộc tính} & \textbf{Nội dung} \\ \hline

Tên Use Case & Đăng xuất \\ \hline

Tác nhân & UserBase (Customer, Admin) \\ \hline

Mô tả & 
Cho phép người dùng thoát khỏi phiên làm việc hiện tại. \\ \hline

Điều kiện tiên quyết & 
- Người dùng đã đăng nhập. \\ \hline

Hậu điều kiện & 
Phiên làm việc kết thúc, token bị xóa. \\ \hline

Luồng chính & 
1. Người dùng nhấn đăng xuất. \newline
2. Hệ thống xóa token khỏi bộ nhớ. \newline
3. Hệ thống chuyển về trang đăng nhập. \\ \hline

Luồng mở rộng (Extend) & 
Không có. \\ \hline

Luồng thay thế & 
Không có. \\ \hline

\end{tabular}

\end{table}

%đặc tả Quản lý hồ sơ cá nhân
\subsubsection{Đặc tả Use Case Quản lý hồ sơ cá nhân}

\begin{table}[H]
    \caption{Đặc tả Use Case Quản lý hồ sơ cá nhân}
\centering
\renewcommand{\arraystretch}{1.3}
\begin{tabular}{|p{4cm}|p{10cm}|}
\hline
\textbf{Thuộc tính} & \textbf{Nội dung} \\ \hline

Tên Use Case & Quản lý hồ sơ cá nhân \\ \hline

Tác nhân & UserBase (Customer, Admin) \\ \hline

Mô tả & 
Cho phép người dùng xem và quản lý thông tin hồ sơ cá nhân của mình. \\ \hline

Điều kiện tiên quyết & 
- Người dùng đã đăng nhập. \\ \hline

Hậu điều kiện & 
Thông tin cá nhân được hiển thị hoặc cập nhật trong cơ sở dữ liệu. \\ \hline

Luồng chính & 
1. Người dùng truy cập trang hồ sơ cá nhân. \newline
2. Hệ thống tải thông tin người dùng từ cơ sở dữ liệu. \newline
3. Hệ thống hiển thị thông tin cá nhân hiện tại (họ tên, email, SĐT). \newline
4. Người dùng chọn thao tác cần thực hiện. \\ \hline

Luồng mở rộng (Extend) & 
- Cập nhật thông tin cá nhân: Người dùng chỉnh sửa họ tên, số điện thoại, ảnh đại diện, hệ thống kiểm tra validation và lưu vào cơ sở dữ liệu. \newline
- Đổi mật khẩu: Người dùng nhập mật khẩu cũ và mật khẩu mới, hệ thống xác thực mật khẩu cũ, mã hóa mật khẩu mới và lưu vào cơ sở dữ liệu. \\ \hline

Luồng thay thế & 
- Thông tin không hợp lệ → hệ thống hiển thị lỗi validation. \newline
- Mật khẩu cũ không đúng → hệ thống từ chối đổi mật khẩu. \\ \hline

\end{tabular}

\end{table}

\subsection{Use case Diagram chi tiết về xử lý thanh toán}
\begin{figure}[H]
    \centering
    \includegraphics[width=1\textwidth]{graphics/main/chapter2/section2/usecase_vnpay.jpg}
    \caption{Use case Diagram xử lý thanh toán}
    \label{fig:usecase_payment}
\end{figure}

\begin{table}[H]
\centering
\caption{Đặc tả Use Case: Thanh toán khi nhận hàng (COD)}
\begin{tabular}{|p{4cm}|p{10cm}|}
\hline
\textbf{Thông tin} & \textbf{Mô tả} \\
\hline
Tên Use Case & Thanh toán khi nhận hàng (COD) \\
\hline
Actor & Customer \\
\hline
Mô tả & Khách hàng chọn thanh toán tiền mặt khi nhận hàng \\
\hline
Điều kiện trước & Khách hàng đã thêm sản phẩm vào giỏ hàng và chọn phương thức thanh toán \\
\hline
Luồng chính & 
\begin{enumerate}[noitemsep]
    \item Khách hàng chọn phương thức "Thanh toán khi nhận hàng"
    \item Khách hàng xác nhận đặt hàng
    \item Hệ thống tạo đơn hàng mới với trạng thái "Chờ giao hàng"
    \item Hệ thống hiển thị thông báo đặt hàng thành công
    \item Hệ thống gửi email xác nhận đơn hàng
\end{enumerate} \\
\hline
Điều kiện sau & Đơn hàng được tạo thành công với trạng thái chờ giao hàng \\
\hline
\end{tabular}
\end{table}

\begin{table}[H]
\centering
\caption{Đặc tả Use Case: Thanh toán qua VNPay}
\begin{tabular}{|p{4cm}|p{10cm}|}
\hline
\textbf{Thông tin} & \textbf{Mô tả} \\
\hline
Tên Use Case & Thanh toán qua VNPay \\
\hline
Actor & Customer \\
\hline
Mô tả & Khách hàng thanh toán trực tuyến qua cổng thanh toán VNPay \\
\hline
Điều kiện trước & Khách hàng đã thêm sản phẩm vào giỏ hàng và chọn phương thức thanh toán \\
\hline
Luồng chính & 
\begin{enumerate}[noitemsep]
    \item Khách hàng chọn phương thức "Thanh toán VNPay"
    \item Khách hàng xác nhận thanh toán
    \item Hệ thống tạo đơn hàng tạm thời và mã giao dịch
    \item Hệ thống chuyển hướng khách hàng đến cổng VNPay
    \item Khách hàng nhập thông tin thanh toán và xác nhận
    \item VNPay xử lý giao dịch và gửi kết quả về hệ thống
    \item Hệ thống xác thực chữ ký và cập nhật trạng thái đơn hàng
    \item Hệ thống hiển thị kết quả thanh toán
    \item Hệ thống gửi email xác nhận thanh toán
\end{enumerate} \\
\hline
Luồng thay thế & 
\textbf{6a. Thanh toán thất bại:}
\begin{enumerate}[noitemsep]
    \item VNPay trả về kết quả thất bại
    \item Hệ thống cập nhật trạng thái "Thanh toán thất bại"
    \item Hệ thống hiển thị thông báo lỗi
    \item Đề xuất khách hàng thử lại hoặc chọn phương thức khác
\end{enumerate} \\
\hline
Điều kiện sau & Đơn hàng được tạo với trạng thái "Đã thanh toán" hoặc "Thanh toán thất bại" \\
\hline
\end{tabular}
\end{table}

\begin{table}[H]
\centering
\caption{Đặc tả Use Case: Xem lịch sử đơn hàng}
\begin{tabular}{|p{4cm}|p{10cm}|}
\hline
\textbf{Thông tin} & \textbf{Mô tả} \\
\hline
Tên Use Case & Xem lịch sử đơn hàng \\
\hline
Actor & Customer \\
\hline
Mô tả & Khách hàng xem danh sách các đơn hàng đã đặt \\
\hline
Điều kiện trước & Khách hàng đã đăng nhập vào hệ thống \\
\hline
Luồng chính & 
\begin{enumerate}[noitemsep]
    \item Khách hàng truy cập trang lịch sử đơn hàng
    \item Hệ thống hiển thị danh sách đơn hàng của khách hàng
    \item Khách hàng xem chi tiết từng đơn hàng
\end{enumerate} \\
\hline
Điều kiện sau & Danh sách đơn hàng được hiển thị thành công \\
\hline
\end{tabular}
\end{table}

\begin{table}[H]
\centering
\caption{Đặc tả Use Case: Quản lý đơn hàng}
\begin{tabular}{|p{4cm}|p{10cm}|}
\hline
\textbf{Thông tin} & \textbf{Mô tả} \\
\hline
Tên Use Case & Quản lý đơn hàng \\
\hline
Actor & Admin \\
\hline
Mô tả & Admin quản lý và cập nhật trạng thái các đơn hàng trong hệ thống \\
\hline
Điều kiện trước & Admin đã đăng nhập vào hệ thống \\
\hline
Luồng chính & 
\begin{enumerate}[noitemsep]
    \item Admin truy cập trang quản lý đơn hàng
    \item Hệ thống hiển thị danh sách tất cả đơn hàng
    \item Admin xem chi tiết đơn hàng
    \item Admin cập nhật trạng thái đơn hàng (đang xử lý, đang giao, hoàn thành, hủy)
    \item Hệ thống lưu thay đổi và thông báo cho khách hàng
\end{enumerate} \\
\hline
Điều kiện sau & Trạng thái đơn hàng được cập nhật thành công \\
\hline
\end{tabular}
\end{table}


\subsection{Use case Diagram chi tiết về phân tích da mặt}
\begin{figure}[H]
    \centering
    \includegraphics[width=1\textwidth]{graphics/main/chapter2/section2/usecase_acne.jpg}
    \caption{Use case Diagram phân tích da mặt}
    \label{fig:usecase_acne}
\end{figure}
\begin{table}[H]
\centering
\caption{Đặc tả Use Case: Phân tích Da mặt}
\begin{tabular}{|p{4cm}|p{10cm}|}
\hline
\textbf{Thông tin} & \textbf{Mô tả} \\
\hline
Tên Use Case & Phân tích Da mặt \\
\hline
Actor & Customer \\
\hline
Mô tả & Khách hàng upload ảnh khuôn mặt để phân tích tình trạng da và nhận gợi ý sản phẩm phù hợp \\
\hline
Điều kiện trước & Khách hàng đã đăng nhập vào hệ thống và có ảnh khuôn mặt rõ nét \\
\hline
Luồng chính & 
\begin{enumerate}[noitemsep]
    \item Khách hàng truy cập chức năng phân tích da mặt
    \item Khách hàng upload ảnh khuôn mặt
    \item Hệ thống sử dụng MediaPipe để nhận diện khuôn mặt
    \item Hệ thống cắt từng vùng da mặt (trán, má, mũi, cằm)
    \item Hệ thống sử dụng mô hình CNN để phân tích mụn trên từng vùng
    \item Hệ thống trả về kết quả phân tích chi tiết
    \item Hệ thống đưa ra lời khuyên chăm sóc da
    \item Hệ thống gợi ý các sản phẩm phù hợp với tình trạng da
\end{enumerate} \\
\hline
Luồng thay thế & 
\textbf{3a. Không phát hiện khuôn mặt:}
\begin{enumerate}[noitemsep]
    \item Hệ thống thông báo lỗi "Không tìm thấy khuôn mặt"
    \item Yêu cầu khách hàng upload lại ảnh rõ nét hơn
    \item Quay lại bước 2
\end{enumerate}
\textbf{3b. Ảnh không đủ chất lượng:}
\begin{enumerate}[noitemsep]
    \item Hệ thống thông báo "Ảnh không đủ rõ nét"
    \item Đề xuất khách hàng chụp ảnh trong điều kiện ánh sáng tốt
    \item Quay lại bước 2
\end{enumerate} \\
\hline
Điều kiện sau & Khách hàng nhận được kết quả phân tích da, lời khuyên chăm sóc và danh sách sản phẩm gợi ý \\
\hline
\end{tabular}
\end{table}



%----------------------------------------------------------------Giỏ hàng và đặt hàng---------------
\subsection{Use case Diagram chi tiết về giỏ hàng và đặt hàng}
\begin{figure}[H]
    \centering
    \includegraphics[width=0.9\textwidth]{graphics/main/chapter2/section2/usecase_order.png}
    \caption{Use case Diagram giỏ hàng và đặt hàng}
    \label{fig:usecase_order}
\end{figure}
Dưới đây là các đặc tả use case của quy trình thêm sản phẩm vào giỏ hàng và đặt hàng.

\begin{table}[H]
\centering
\caption{Đặc tả Use Case UC-05: Thêm vào giỏ hàng}
\label{tab:uc05}
\begin{tabular}{|p{4cm}|p{11cm}|}
\hline
\textbf{Actor} & Customer \\
\hline
\textbf{Mô tả} & Khách hàng thêm sản phẩm yêu thích vào giỏ hàng để chuẩn bị mua sắm \\
\hline
\textbf{Tiền điều kiện} & Khách hàng đang xem chi tiết sản phẩm hoặc danh sách sản phẩm \\
\hline
\textbf{Luồng chính} & 
1. Khách hàng chọn sản phẩm \newline
2. Chọn số lượng và các tùy chọn (nếu có) \newline
3. Nhấn nút "Thêm vào giỏ hàng" \newline
4. Hệ thống kiểm tra tồn kho \newline
5. Hệ thống thêm sản phẩm vào giỏ hàng và thông báo thành công \\
\hline
\textbf{Luồng phụ} & 
4a. Sản phẩm hết hàng $\rightarrow$ Hệ thống thông báo và không cho phép thêm vào giỏ \\
\hline
\textbf{Hậu điều kiện} & Sản phẩm được lưu trong giỏ hàng của khách hàng \\
\hline
\end{tabular}
\end{table}

\begin{table}[H]
\centering
\caption{Đặc tả Use Case UC-06: Đặt hàng}
\label{tab:uc06}
\begin{tabular}{|p{4cm}|p{11cm}|}
\hline
\textbf{Actor} & Customer \\
\hline
\textbf{Mô tả} & Khách hàng tiến hành đặt mua các sản phẩm đã có trong giỏ hàng \\
\hline
\textbf{Tiền điều kiện} & Có sản phẩm trong giỏ hàng và đã đăng nhập \\
\hline
\textbf{Luồng chính} & 
1. Khách hàng truy cập giỏ hàng \newline
2. Chọn các sản phẩm muốn thanh toán \newline
3. Nhấn "Đặt hàng" \newline
4. Điền thông tin cá nhân (UC-07) \newline
5. Chọn phương thức thanh toán (UC-08) \newline
6. Xác nhận đơn hàng \newline
7. Hệ thống tạo đơn hàng và trừ số lượng tồn kho \\
\hline
\textbf{Luồng phụ} & 
Extend: Chọn mã giảm giá (UC-09) \\
\hline
\textbf{Hậu điều kiện} & Đơn hàng được tạo thành công với trạng thái "Chờ xác nhận" \\
\hline
\end{tabular}
\end{table}

\begin{table}[H]
\centering
\caption{Đặc tả Use Case UC-07: Điền thông tin cá nhân}
\label{tab:uc07}
\begin{tabular}{|p{4cm}|p{11cm}|}
\hline
\textbf{Actor} & Customer \\
\hline
\textbf{Mô tả} & Cung cấp thông tin nhận hàng cho đơn hàng \\
\hline
\textbf{Luồng chính} & 
1. Nhập họ tên người nhận \newline
2. Nhập số điện thoại \newline
3. Nhập địa chỉ giao hàng cụ thể \newline
4. Hệ thống ghi nhận thông tin \\
\hline
\end{tabular}
\end{table}

\begin{table}[H]
\centering
\caption{Đặc tả Use Case UC-08: Chọn phương thức thanh toán}
\label{tab:uc08}
\begin{tabular}{|p{4cm}|p{11cm}|}
\hline
\textbf{Actor} & Customer \\
\hline
\textbf{Mô tả} & Khách hàng lựa chọn cách thức thanh toán cho đơn hàng \\
\hline
\textbf{Luồng chính} & 
1. Khách hàng chọn giữa các phương thức: COD, VNPay, Chuyển khoản \newline
2. Hệ thống ghi nhận phương thức đã chọn \\
\hline
\end{tabular}
\end{table}

\begin{table}[H]
\centering
\caption{Đặc tả Use Case UC-09: Chọn mã giảm giá}
\label{tab:uc09}
\begin{tabular}{|p{4cm}|p{11cm}|}
\hline
\textbf{Actor} & Customer \\
\hline
\textbf{Mô tả} & Khách hàng áp dụng mã khuyến mãi để được giảm giá đơn hàng \\
\hline
\textbf{Luồng chính} & 
1. Nhập mã giảm giá hoặc chọn từ danh sách \newline
2. Hệ thống kiểm tra điều kiện áp dụng \newline
3. Tính toán lại tổng tiền đơn hàng \\
\hline
\textbf{Luồng phụ} & 
2a. Mã không hợp lệ/hết hạn $\rightarrow$ Thông báo lỗi \\
\hline
\end{tabular}
\end{table}

% usecase đánh giá sản phẩm admin
\subsection{Use case Diagram đánh giá sản phẩm admin}
\begin{figure}[H]
    \centering
    \includegraphics[width=1\textwidth]{graphics/main/chapter2/section2/usecasedgamin.png}
    \caption{Use case Diagram đánh giá sản phẩm }
    \label{fig:usecase_addg}
\end{figure}
 
%đặc tả admin
\subsubsection{Đặc tả Use Case Quản lý đánh giá sản phẩm}

\begin{table}[H]
    \caption{Đặc tả Use Case Quản lý đánh giá sản phẩm (Admin)}
\centering
\renewcommand{\arraystretch}{1.3}
\begin{tabular}{|p{4cm}|p{10cm}|}
\hline
\textbf{Thuộc tính} & \textbf{Nội dung} \\ \hline

Tên Use Case & Quản lý đánh giá sản phẩm \\ \hline

Actor & Admin \\ \hline

Mô tả & 
Cho phép Admin xem, kiểm duyệt hoặc xóa các đánh giá sản phẩm do khách hàng gửi lên hệ thống. \\ \hline

Điều kiện tiên quyết & 
Admin đã đăng nhập thành công vào hệ thống. \\ \hline

Hậu điều kiện & 
Đánh giá được cập nhật trạng thái (hiển thị/ẩn) hoặc bị xóa khỏi hệ thống. \\ \hline

Luồng chính & 
1. Admin truy cập trang quản lý đánh giá. \newline
2. Hệ thống hiển thị danh sách đánh giá của khách hàng. \newline
3. Admin chọn một đánh giá cụ thể. \newline
4. Admin có thể lọc đánh giá. \newline
4. Thực hiện thao tác duyệt, ẩn hoặc xóa đánh giá. \newline
5. Hệ thống cập nhật dữ liệu và hiển thị danh sách mới. \\ \hline

Luồng thay thế & 
Nếu xảy ra lỗi hệ thống hoặc không tìm thấy đánh giá, hệ thống hiển thị thông báo lỗi và không thực hiện thay đổi. \\ \hline

\end{tabular}

\end{table}

% usecase khách hàng đánh giá sản phẩm 
\subsection{Use case Diagram khách hàng đánh giá sản phẩm}
\begin{figure}[H]
    \centering
    \includegraphics[width=1\textwidth]{graphics/main/chapter2/section2/usecasecusdg.png}
    \caption{Use case Diagram khách hàng đánh giá sản phẩm }
    \label{fig:usecase_gdg}
\end{figure}
 
%đặc tả khách hàng đánh giá sản phẩm 
\subsubsection{Đặc tả Use Case Khách hàng đánh giá sản phẩm}

\begin{table}[H]
    \caption{Đặc tả Use Case Khách hàng đánh giá sản phẩm}
\centering
\renewcommand{\arraystretch}{1.3}
\begin{tabular}{|p{4cm}|p{10cm}|}
\hline
\textbf{Thuộc tính} & \textbf{Nội dung} \\ \hline

Tên Use Case & Đánh giá sản phẩm \\ \hline

Tác nhân & Customer \\ \hline

Mô tả & 
Cho phép khách hàng đánh giá sản phẩm đã mua khi đơn hàng ở trạng thái hoàn thành. \\ \hline

Điều kiện tiên quyết & 
- Khách hàng đã đăng nhập. \newline
- Đơn hàng đã hoàn thành. \\ \hline

Hậu điều kiện & 
Đánh giá được lưu vào hệ thống và hiển thị tại trang sản phẩm. \\ \hline

Luồng chính & 
1. Khách hàng đăng nhập. \newline
2. Truy cập mục “Đơn hàng của tôi”. \newline
3. Chọn đơn hàng đã hoàn thành. \newline
4. Chọn “Đánh giá sản phẩm”. \newline
5. Nhập nội dung và chọn số sao. \newline
6. Gửi đánh giá. \newline
7. Hệ thống lưu và thông báo thành công. \\ \hline

Luồng mở rộng (Extend) & 
- Đánh giá sao (1–5 sao). \newline
- Viết bình luận chi tiết. \newline
- Gửi ảnh/video minh họa. \newline
- Sửa đánh giá sau khi đã gửi. \\ \hline

Luồng thay thế & 
- Đơn hàng chưa hoàn thành → không được đánh giá. \newline
- Nội dung vi phạm → hệ thống từ chối. \newline
- Ảnh/video sai định dạng → báo lỗi. \\ \hline

\end{tabular}

\end{table}

% usecase ChatbotAI tư vấn sản phẩm 
\subsection{Use case Diagram ChatbotAI tư vấn sản phẩm}
\begin{figure}[H]
    \centering
    \includegraphics[width=1\textwidth]{graphics/main/chapter2/section2/uc_chat.jpg}
    \caption{Use case Diagram ChatbotAI tư vấn sản phẩm}
    \label{fig:usecase_chat}
\end{figure}

%đặc tả Gửi tin nhắn hỏi đáp
\subsubsection{Đặc tả Use Case ChatbotAI tư vấn sản phẩm}

\begin{table}[H]
    \caption{Đặc tả Use Case ChatbotAI tư vấn sản phẩm}
\centering
\renewcommand{\arraystretch}{1.3}
\begin{tabular}{|p{4cm}|p{10cm}|}
\hline
\textbf{Thuộc tính} & \textbf{Nội dung} \\ \hline

Tên Use Case & Gửi tin nhắn hỏi đáp \\ \hline

Tác nhân & Customer \\ \hline

Mô tả & 
Cho phép khách hàng gửi câu hỏi về sản phẩm, chăm sóc da thông qua chatbot AI và nhận phản hồi tự động. \\ \hline

Điều kiện tiên quyết & 
- Khách hàng đã truy cập giao diện chatbot. \\ \hline

Hậu điều kiện & 
Câu trả lời từ AI được hiển thị trên giao diện chat. \\ \hline

Luồng chính & 
1. Khách hàng mở chatbot. \newline
2. Hệ thống hiển thị giao diện chat và lời chào. \newline
3. Khách hàng nhập câu hỏi và gửi. \newline
4. Hệ thống chuyển câu hỏi thành vector embedding. \newline
5. Hệ thống tìm kiếm context liên quan trong Qdrant. \newline
6. Hệ thống gửi câu hỏi kèm context đến Ollama LLM. \newline
7. AI trả về câu trả lời. \newline
8. Hệ thống hiển thị câu trả lời cho khách hàng. \\ \hline

Luồng mở rộng (Extend) & 
Không có. \\ \hline

Luồng thay thế & 
- Không tìm thấy context phù hợp → hệ thống trả về câu trả lời mặc định. \newline
- Câu hỏi ngoài lề (không liên quan sản phẩm/skincare) → AI từ chối trả lời và hướng dẫn hỏi về chủ đề phù hợp. \newline
- Lỗi kết nối đến Ollama → hệ thống thông báo lỗi. \\ \hline

\end{tabular}

\end{table}

% usecase Quản lý cửa hàng
\subsection{Use case Diagram quản lý cửa hàng}
\begin{figure}[H]
    \centering
    \includegraphics[width=1\textwidth]{graphics/main/chapter2/section2/uc_store.jpg}
    \caption{Use case Diagram quản lý cửa hàng}
    \label{fig:usecase_store}
\end{figure}

%đặc tả Cập nhật thông tin cửa hàng
\subsubsection{Đặc tả Use Case Cập nhật thông tin cửa hàng}

\begin{table}[H]
    \caption{Đặc tả Use Case Cập nhật thông tin cửa hàng}
\centering
\renewcommand{\arraystretch}{1.3}
\begin{tabular}{|p{4cm}|p{10cm}|}
\hline
\textbf{Thuộc tính} & \textbf{Nội dung} \\ \hline

Tên Use Case & Cập nhật thông tin cửa hàng \\ \hline

Tác nhân & Admin \\ \hline

Mô tả & 
Cho phép Admin cập nhật các thông tin cơ bản của cửa hàng như tên, số điện thoại, email. \\ \hline

Điều kiện tiên quyết & 
- Admin đã đăng nhập. \newline
- Cửa hàng đã tồn tại trong hệ thống. \\ \hline

Hậu điều kiện & 
Thông tin cửa hàng được cập nhật trong cơ sở dữ liệu. \\ \hline

Luồng chính & 
1. Admin mở trang quản lý cửa hàng. \newline
2. Hệ thống tải và hiển thị thông tin cửa hàng hiện tại. \newline
3. Admin chỉnh sửa thông tin (tên, SĐT, email). \newline
4. Admin xác nhận cập nhật. \newline
5. Hệ thống kiểm tra validation. \newline
6. Hệ thống lưu thông tin vào cơ sở dữ liệu. \newline
7. Hệ thống hiển thị thông báo cập nhật thành công. \\ \hline

Luồng mở rộng (Extend) & 
Không có. \\ \hline

Luồng thay thế & 
- Thông tin không hợp lệ (thiếu tên, sai định dạng email/SĐT) → hệ thống hiển thị lỗi validation. \newline
- Không tìm thấy cửa hàng → hệ thống thông báo lỗi. \\ \hline

\end{tabular}

\end{table}

%đặc tả Quản lý địa chỉ cửa hàng
\subsubsection{Đặc tả Use Case Quản lý địa chỉ cửa hàng}

\begin{table}[H]
    \caption{Đặc tả Use Case Quản lý địa chỉ cửa hàng}
\centering
\renewcommand{\arraystretch}{1.3}
\begin{tabular}{|p{4cm}|p{10cm}|}
\hline
\textbf{Thuộc tính} & \textbf{Nội dung} \\ \hline

Tên Use Case & Quản lý địa chỉ cửa hàng \\ \hline

Tác nhân & Admin \\ \hline

Mô tả & 
Cho phép Admin quản lý địa chỉ cửa hàng bao gồm thêm, sửa, xóa địa chỉ và cập nhật tọa độ trên bản đồ. \\ \hline

Điều kiện tiên quyết & 
- Admin đã đăng nhập. \\ \hline

Hậu điều kiện & 
Địa chỉ cửa hàng được cập nhật trong cơ sở dữ liệu. \\ \hline

Luồng chính & 
1. Admin mở trang quản lý cửa hàng. \newline
2. Hệ thống hiển thị thông tin cửa hàng kèm bản đồ vị trí hiện tại. \newline
3. Admin chọn thao tác (Thêm / Sửa / Xóa địa chỉ). \\ \hline

Luồng mở rộng (Extend) & 
- Thêm địa chỉ cửa hàng: Admin nhập địa chỉ chi tiết, tìm kiếm và chấm vị trí trên bản đồ, hệ thống lấy tọa độ và lưu vào cơ sở dữ liệu. \newline
- Sửa địa chỉ cửa hàng: Admin chỉnh sửa thông tin địa chỉ hoặc cập nhật vị trí trên bản đồ, hệ thống cập nhật tọa độ mới vào cơ sở dữ liệu. \newline
- Xóa địa chỉ cửa hàng: Admin chọn địa chỉ cần xóa, xác nhận xóa, hệ thống xóa khỏi cơ sở dữ liệu. \\ \hline

Luồng thay thế & 
- Thông tin không hợp lệ (thiếu địa chỉ, thiếu tọa độ GPS) → hệ thống hiển thị lỗi validation. \newline
- Không tìm thấy địa chỉ cần sửa/xóa → hệ thống thông báo lỗi. \\ \hline

\end{tabular}

\end{table}

%đặc tả Xem giới thiệu cửa hàng
\subsubsection{Đặc tả Use Case Xem giới thiệu cửa hàng}

\begin{table}[H]
    \caption{Đặc tả Use Case Xem giới thiệu cửa hàng}
\centering
\renewcommand{\arraystretch}{1.3}
\begin{tabular}{|p{4cm}|p{10cm}|}
\hline
\textbf{Thuộc tính} & \textbf{Nội dung} \\ \hline

Tên Use Case & Xem giới thiệu cửa hàng \\ \hline

Tác nhân & Customer \\ \hline

Mô tả & 
Cho phép khách hàng xem thông tin giới thiệu của cửa hàng bao gồm tên, địa chỉ và vị trí trên bản đồ. \\ \hline

Điều kiện tiên quyết & 
Không có. \\ \hline

Hậu điều kiện & 
Thông tin cửa hàng được hiển thị cho khách hàng. \\ \hline

Luồng chính & 
1. Khách hàng truy cập trang giới thiệu cửa hàng. \newline
2. Hệ thống tải thông tin cửa hàng từ cơ sở dữ liệu. \newline
3. Hệ thống hiển thị tên, địa chỉ, vị trí trên bản đồ. \\ \hline

Luồng mở rộng (Extend) & 
Không có. \\ \hline

Luồng thay thế & 
- Lỗi tải dữ liệu → hệ thống thông báo lỗi. \\ \hline

\end{tabular}

\end{table}

%đặc tả Liên hệ cửa hàng
\subsubsection{Đặc tả Use Case Liên hệ cửa hàng}

\begin{table}[H]
    \caption{Đặc tả Use Case Liên hệ cửa hàng}
\centering
\renewcommand{\arraystretch}{1.3}
\begin{tabular}{|p{4cm}|p{10cm}|}
\hline
\textbf{Thuộc tính} & \textbf{Nội dung} \\ \hline

Tên Use Case & Liên hệ cửa hàng \\ \hline

Tác nhân & Customer \\ \hline

Mô tả & 
Cho phép khách hàng xem thông tin liên hệ của cửa hàng (số điện thoại, email) để liên hệ trực tiếp. \\ \hline

Điều kiện tiên quyết & 
Không có. \\ \hline

Hậu điều kiện & 
Thông tin liên hệ được hiển thị cho khách hàng. \\ \hline

Luồng chính & 
1. Khách hàng truy cập trang liên hệ. \newline
2. Hệ thống hiển thị thông tin liên hệ (SĐT, email). \newline
3. Khách hàng chọn phương thức liên hệ (gọi điện hoặc gửi email). \\ \hline

Luồng mở rộng (Extend) & 
Không có. \\ \hline

Luồng thay thế & 
- Lỗi tải dữ liệu → hệ thống thông báo lỗi. \\ \hline

\end{tabular}

\end{table}

% usecase Quản lý vận chuyển 
\subsection{Use case Diagram quản lý vận chuyển}
\begin{figure}[H]
    \centering
    \includegraphics[width=1\textwidth]{graphics/main/chapter2/section2/uc_vanchuyen.jpg}
    \caption{Use case Diagram quản lý vận chuyển}
    \label{fig:usecase_vanchuyen}
\end{figure}

%đặc tả Xem phương thức vận chuyển
\subsubsection{Đặc tả Use Case Xem phương thức vận chuyển}

\begin{table}[H]
    \caption{Đặc tả Use Case Xem phương thức vận chuyển}
\centering
\renewcommand{\arraystretch}{1.3}
\begin{tabular}{|p{4cm}|p{10cm}|}
\hline
\textbf{Thuộc tính} & \textbf{Nội dung} \\ \hline

Tên Use Case & Xem phương thức vận chuyển \\ \hline

Tác nhân & Admin \\ \hline

Mô tả & 
Cho phép Admin xem danh sách các nhà vận chuyển và phương thức vận chuyển hiện có trong hệ thống. \\ \hline

Điều kiện tiên quyết & 
- Admin đã đăng nhập. \\ \hline

Hậu điều kiện & 
Danh sách nhà vận chuyển và phương thức vận chuyển được hiển thị. \\ \hline

Luồng chính & 
1. Admin mở trang quản lý vận chuyển. \newline
2. Hệ thống tải danh sách nhà vận chuyển đang hoạt động. \newline
3. Hệ thống hiển thị danh sách nhà vận chuyển kèm các phương thức (tên, mã, phí cơ bản, phí theo km, thời gian giao dự kiến). \\ \hline

Luồng mở rộng (Extend) & 
Không có. \\ \hline

Luồng thay thế & 
- Không có nhà vận chuyển nào đang hoạt động → hệ thống hiển thị danh sách trống. \newline
- Lỗi tải dữ liệu → hệ thống thông báo lỗi. \\ \hline

\end{tabular}

\end{table}

%đặc tả Tính phí theo khoảng cách
\subsubsection{Đặc tả Use Case Tính phí theo khoảng cách}

\begin{table}[H]
    \caption{Đặc tả Use Case Tính phí theo khoảng cách}
\centering
\renewcommand{\arraystretch}{1.3}
\begin{tabular}{|p{4cm}|p{10cm}|}
\hline
\textbf{Thuộc tính} & \textbf{Nội dung} \\ \hline

Tên Use Case & Tính phí theo khoảng cách \\ \hline

Tác nhân & Customer \\ \hline

Mô tả & 
Cho phép khách hàng tính phí vận chuyển dựa trên phương thức vận chuyển đã chọn và khoảng cách giao hàng. \\ \hline

Điều kiện tiên quyết & 
- Khách hàng đang trong quy trình đặt hàng. \\ \hline

Hậu điều kiện & 
Phí vận chuyển được tính và hiển thị cho khách hàng. \\ \hline

Luồng chính & 
1. Khách hàng chọn phương thức vận chuyển. \newline
2. Khách hàng nhập địa chỉ giao hàng (include: Nhập địa chỉ giao hàng). \newline
3. Hệ thống tính khoảng cách từ cửa hàng đến địa chỉ giao. \newline
4. Hệ thống tính phí vận chuyển (phí cơ bản + phí theo km × khoảng cách). \newline
5. Hệ thống hiển thị phí vận chuyển cho khách hàng (include: Xem phí vận chuyển). \\ \hline

Luồng mở rộng (Extend) & 
Không có. \\ \hline

Luồng thay thế & 
- Không tìm thấy phương thức vận chuyển → hệ thống thông báo lỗi. \newline
- Địa chỉ giao hàng không hợp lệ → hệ thống yêu cầu nhập lại. \\ \hline

\end{tabular}

\end{table}

%đặc tả Cập nhật trạng thái vận chuyển
\subsubsection{Đặc tả Use Case Cập nhật trạng thái vận chuyển}

\begin{table}[H]
    \caption{Đặc tả Use Case Cập nhật trạng thái vận chuyển}
\centering
\renewcommand{\arraystretch}{1.3}
\begin{tabular}{|p{4cm}|p{10cm}|}
\hline
\textbf{Thuộc tính} & \textbf{Nội dung} \\ \hline

Tên Use Case & Cập nhật trạng thái vận chuyển \\ \hline

Tác nhân & Admin \\ \hline

Mô tả & 
Cho phép Admin cập nhật trạng thái vận chuyển của đơn hàng và ghi nhận lịch sử tracking. \\ \hline

Điều kiện tiên quyết & 
- Admin đã đăng nhập. \newline
- Đơn hàng đã tồn tại trong hệ thống. \\ \hline

Hậu điều kiện & 
Bản ghi tracking mới được lưu vào cơ sở dữ liệu. \\ \hline

Luồng chính & 
1. Admin mở trang quản lý vận chuyển. \newline
2. Hệ thống hiển thị danh sách đơn vận chuyển. \newline
3. Admin chọn đơn hàng cần cập nhật. \newline
4. Hệ thống hiển thị chi tiết đơn hàng và lịch sử trạng thái. \newline
5. Admin chọn trạng thái mới (Đang lấy hàng / Đang giao / Giao thành công / Giao thất bại). \newline
6. Admin nhập thông tin bổ sung (vị trí, mô tả). \newline
7. Admin xác nhận cập nhật. \newline
8. Hệ thống tạo bản ghi tracking mới vào cơ sở dữ liệu. \newline
9. Hệ thống hiển thị thông báo cập nhật thành công. \\ \hline

Luồng mở rộng (Extend) & 
Không có. \\ \hline

Luồng thay thế & 
- Thiếu trạng thái → hệ thống hiển thị lỗi "Trạng thái là bắt buộc". \newline
- Không tìm thấy đơn hàng → hệ thống thông báo lỗi. \\ \hline

\end{tabular}

\end{table}

\section{Sơ đồ ERD (Entity Relationship Diagram)}
\begin{figure}[H]
    \centering
    \includegraphics[width=0.9\textwidth]{graphics/main/chapter2/section6/erd.jpg}
    \caption{Sơ đồ ERD của hệ thống GlowUp}
    \label{fig:erd_glowup}
\end{figure}

\subsection*{Các thực thể}

Cơ sở dữ liệu của hệ thống được thiết kế gồm các thực thể chính sau:

\begin{itemize}[noitemsep]
    \item \textbf{Users}: Lưu trữ thông tin tài khoản người dùng, phục vụ cho việc đăng nhập, phân quyền và quản lý trạng thái sử dụng hệ thống.
    \item \textbf{Categories}: Lưu thông tin các danh mục sản phẩm, giúp phân loại sản phẩm theo từng nhóm chức năng.
    \item \textbf{Products}: Lưu trữ thông tin chi tiết của sản phẩm như tên, thương hiệu, giá bán, số lượng tồn kho và mô tả.
    \item \textbf{Reviews}: Lưu thông tin đánh giá và nhận xét của người dùng đối với sản phẩm.
    \item \textbf{Carts}: Lưu thông tin giỏ hàng của người dùng trong quá trình mua sắm.
    \item \textbf{Cart\_items}: Lưu danh sách các sản phẩm và số lượng tương ứng trong từng giỏ hàng.
    \item \textbf{Orders}: Lưu thông tin đơn hàng của người dùng sau khi thực hiện thanh toán.
    \item \textbf{Order\_items}: Lưu chi tiết các sản phẩm thuộc từng đơn hàng.
    \item \textbf{Payments}: Lưu thông tin giao dịch thanh toán cho các đơn hàng.
    \item \textbf{Vouchers}: Lưu thông tin mã giảm giá được áp dụng trong quá trình mua sắm.
    \item \textbf{Shipping\_providers}: Lưu thông tin các đơn vị cung cấp dịch vụ vận chuyển.
    \item \textbf{Shipping\_methods}: Lưu thông tin các phương thức vận chuyển do từng đơn vị cung cấp.
    \item \textbf{Shipping\_trackings}: Lưu trạng thái theo dõi quá trình vận chuyển của đơn hàng.
\end{itemize}

\subsection*{Mối quan hệ giữa các thực thể}

Các thực thể trong hệ thống có mối liên kết chặt chẽ nhằm đảm bảo các chức năng nghiệp vụ được vận hành thống nhất.

Người dùng có thể thực hiện nhiều hoạt động trong hệ thống như tạo giỏ hàng, đặt đơn hàng và gửi đánh giá cho sản phẩm. Trong quá trình mua sắm, mỗi người dùng có thể phát sinh nhiều đơn hàng và nhiều lượt đánh giá khác nhau đối với các sản phẩm đã mua.

Sản phẩm được tổ chức theo từng danh mục nhằm hỗ trợ người dùng dễ dàng tìm kiếm và duyệt sản phẩm. Mỗi danh mục bao gồm nhiều sản phẩm, trong khi mỗi sản phẩm chỉ thuộc về một danh mục cụ thể.

Giỏ hàng được sử dụng để lưu trữ tạm thời các sản phẩm mà người dùng lựa chọn. Một giỏ hàng có thể chứa nhiều sản phẩm với số lượng khác nhau và có thể được chuyển đổi thành đơn hàng khi người dùng tiến hành đặt mua.

Đơn hàng được hình thành từ giỏ hàng của người dùng và bao gồm danh sách các sản phẩm đã chọn. Mỗi đơn hàng gắn liền với thông tin thanh toán tương ứng và được theo dõi trạng thái vận chuyển trong suốt quá trình giao hàng. Đơn hàng được vận chuyển thông qua các phương thức và đơn vị vận chuyển khác nhau tùy theo lựa chọn của người dùng.

Ngoài ra, hệ thống cho phép áp dụng mã giảm giá trong quá trình đặt hàng nhằm hỗ trợ các chương trình khuyến mãi và tăng trải nghiệm mua sắm cho người dùng.

\section{Sơ đồ Class (Class Diagram)}
\section{Sơ đồ hoạt động (Activity Diagram)}




\subsection{Sơ đồ hoạt động xử lý thanh toán}
\begin{figure}[H]
    \centering
    \includegraphics[width=0.8\textwidth]{graphics/main/chapter2/section2/activity_vnpay.jpg}
    \caption{Sơ đồ hoạt động thanh toán qua VNPAY}
    \label{fig:activity_vnpay}
\end{figure}

\textbf{Mục đích:}  
Chức năng thanh toán và quản lý đơn hàng nhằm hỗ trợ người dùng hoàn tất quá trình mua hàng trên hệ thống một cách thuận tiện và an toàn. Hệ thống cung cấp hai hình thức thanh toán là thanh toán khi nhận hàng (COD) và thanh toán trực tuyến qua cổng VNPay, giúp người dùng linh hoạt lựa chọn phương thức phù hợp. Đồng thời, chức năng này cho phép người dùng theo dõi lịch sử đơn hàng và gửi yêu cầu hủy đơn khi cần thiết, hỗ trợ Admin quản lý và xử lý trạng thái đơn hàng một cách hiệu quả.

\textbf{Luồng hoạt động:}  
Người dùng bắt đầu bằng việc lựa chọn phương thức thanh toán cho đơn hàng.  

\begin{itemize}
    \item Trường hợp người dùng chọn thanh toán khi nhận hàng (COD), hệ thống tiến hành tạo đơn hàng và hiển thị kết quả thanh toán cho người dùng.
    \item Trường hợp người dùng chọn thanh toán qua VNPay, hệ thống chuyển hướng người dùng đến cổng thanh toán VNPay để thực hiện giao dịch. Sau khi quá trình thanh toán được xử lý:
    \begin{itemize}
        \item Nếu giao dịch thành công, hệ thống cập nhật trạng thái đơn hàng và hiển thị kết quả thanh toán thành công.
        \item Nếu giao dịch thất bại, hệ thống thông báo kết quả thất bại để người dùng có thể thực hiện lại hoặc lựa chọn phương thức thanh toán khác.
    \end{itemize}
\end{itemize}

Sau khi hoàn tất thanh toán, người dùng có thể xem lịch sử đơn hàng. Trong trường hợp người dùng có nhu cầu hủy đơn, hệ thống tiếp nhận yêu cầu hủy và chuyển cho Admin xử lý. Sau khi Admin xác nhận, hệ thống cập nhật lại trạng thái đơn hàng tương ứng và thông báo kết quả cho người dùng.

\subsection{Sơ đồ hoạt động phân tích da mặt}
\begin{figure}[H]
    \centering
    \includegraphics[width=0.6\textwidth]{graphics/main/chapter2/section2/activity_acne.jpg}
    \caption{Sơ đồ hoạt động phân tích da mặt}
    \label{fig:activity_acne}
\end{figure}


\textbf{Mục đích:}  
Chức năng phân tích da mặt bằng AI nhằm hỗ trợ người dùng nhận biết tình trạng da và mức độ mụn thông qua hình ảnh khuôn mặt được tải lên hệ thống. Hệ thống ứng dụng công nghệ MediaPipe để nhận diện khuôn mặt và các điểm đặc trưng, kết hợp mô hình CNN để phân tích tình trạng mụn trên từng vùng da. Dựa trên kết quả phân tích, hệ thống đưa ra lời khuyên chăm sóc da và gợi ý sản phẩm phù hợp, giúp người dùng lựa chọn sản phẩm hiệu quả và cá nhân hóa trải nghiệm mua sắm.

\textbf{Luồng hoạt động:}  
Người dùng thực hiện tải ảnh khuôn mặt lên hệ thống. Hệ thống sử dụng MediaPipe để nhận diện khuôn mặt và xác định các vùng da chính gồm trán, má trái/phải, mũi và cằm. Sau đó, từng vùng da được cắt ra và đưa vào mô hình CNN để phân tích tình trạng mụn.  

Kết quả phân tích (bao gồm số lượng, vị trí và mức độ mụn) được hệ thống tổng hợp và hiển thị cho người dùng. Dựa trên kết quả này, hệ thống đồng thời thực hiện hai chức năng: đưa ra lời khuyên chăm sóc da phù hợp và gợi ý các sản phẩm tương ứng với tình trạng da của người dùng.
%Sơ đồ hoạt động admin đánh giá sản phẩm
\subsection{Sơ đồ hoạt động admin quản lý đánh giá}

\begin{figure}[H]
    \centering
    \includegraphics[width=0.6\textwidth]{graphics/main/chapter2/section2/acctivityadmindanhgia.png}
    \caption{Sơ đồ hoạt động ADMIN quản lý đánh giá}
    \label{fig:activity_dg}
\end{figure}

%Mô tả sơ đồ hoạt động đánh giá sản phẩm

\textbf{Mục đích:}

Chức năng quản lý đánh giá cho phép Admin theo dõi, kiểm soát và xử lý các đánh giá sản phẩm do khách hàng gửi lên hệ thống. Thông qua chức năng này, Admin có thể xem danh sách đánh giá, tìm kiếm hoặc lọc đánh giá theo tiêu chí nhất định, xem chi tiết nội dung đánh giá và thực hiện các thao tác như ẩn hoặc xóa những đánh giá vi phạm quy định.

Mục đích chính của chức năng là đảm bảo nội dung hiển thị trên website phù hợp, duy trì tính minh bạch, chuyên nghiệp và nâng cao chất lượng trải nghiệm người dùng.

\textbf{Luồng hoạt động:}

Admin bắt đầu bằng việc truy cập vào chức năng quản lý đánh giá trong hệ thống. Hệ thống hiển thị danh sách các đánh giá sản phẩm hiện có.

\begin{itemize}
    \item Trường hợp Admin thực hiện lọc hoặc tìm kiếm đánh giá, hệ thống cập nhật và hiển thị danh sách đánh giá theo tiêu chí đã chọn.
    
    \item Khi Admin chọn một đánh giá cụ thể, hệ thống hiển thị nội dung chi tiết của đánh giá đó.
\end{itemize}

Sau khi xem nội dung đánh giá, Admin có thể lựa chọn hành động xử lý:

\begin{itemize}
    \item Nếu Admin chọn ẩn đánh giá, hệ thống hiển thị thông báo xác nhận.
    \begin{itemize}
        \item Nếu Admin đồng ý, hệ thống cập nhật trạng thái đánh giá sang ``Ẩn'' và làm mới danh sách.
        \item Nếu Admin hủy, hệ thống không thực hiện thay đổi và quay lại màn hình trước đó.
    \end{itemize}
    
    \item Nếu Admin chọn xóa đánh giá, hệ thống hiển thị thông báo xác nhận.
    \begin{itemize}
        \item Nếu Admin đồng ý, hệ thống xóa đánh giá khỏi cơ sở dữ liệu và cập nhật lại danh sách.
        \item Nếu Admin hủy, hệ thống giữ nguyên dữ liệu và quay lại danh sách đánh giá.
    \end{itemize}
\end{itemize}

Sau khi hoàn tất thao tác, hệ thống hiển thị lại danh sách đánh giá đã được cập nhật và quy trình kết thúc.
%Sơ đồ hoạt động khách hàng đánh giá sản phẩm
\subsection{Sơ đồ hoạt động khách hàng đánh giá sản phẩm}

\begin{figure}[H]
    \centering
    \includegraphics[width=0.6\textwidth]{graphics/main/chapter2/section2/acctivitykhachhangdanhgia.png}
    \caption{Sơ đồ hoạt động đánh giá sản phẩm}
    \label{fig:activity_dg}
\end{figure}
 %Mô tả hoạt động khách hàng đánh giá sản phẩm

\textbf{Mục đích:}

Chức năng đánh giá sản phẩm cho phép khách hàng gửi nhận xét và phản hồi về sản phẩm đã mua. Thông qua chức năng này, khách hàng có thể chia sẻ trải nghiệm sử dụng sản phẩm, góp phần cung cấp thông tin tham khảo cho những người mua sau và giúp hệ thống nâng cao chất lượng dịch vụ.

\textbf{Luồng hoạt động:}

Khách hàng bắt đầu bằng việc lựa chọn sản phẩm đã mua để thực hiện đánh giá.

\begin{itemize}
    \item Sau khi chọn sản phẩm hợp lệ (đã mua), hệ thống hiển thị biểu mẫu đánh giá.
    
    \item Khách hàng nhập nội dung đánh giá vào biểu mẫu.
    
    \item Hệ thống tiến hành kiểm tra tính hợp lệ của nội dung đánh giá.
\end{itemize}

\begin{itemize}
    \item Nếu nội dung không hợp lệ (thiếu thông tin, vi phạm quy định, vượt quá giới hạn ký tự,...), hệ thống hiển thị thông báo lỗi và yêu cầu khách hàng nhập lại nội dung.
    
    \item Nếu nội dung hợp lệ, hệ thống lưu đánh giá vào cơ sở dữ liệu và hiển thị thông báo hoàn thành.
\end{itemize}

Sau khi nhận được thông báo hoàn thành, khách hàng có thể:

\begin{itemize}
    \item Chọn ``Đồng ý'' để kết thúc quy trình đánh giá.
    
    \item Chọn ``Hủy'' để thoát khỏi chức năng mà không thực hiện thêm thao tác nào.
\end{itemize}

Quy trình kết thúc khi khách hàng xác nhận hoàn tất hoặc thoát khỏi chức năng.

\section{Sơ đồ tuần tự (Sequence Diagram)}

\subsection{Sơ đồ tuần tự xử lý thanh toán}
\begin{figure}[H]
    \centering
    \includegraphics[width=0.8\textwidth]{graphics/main/chapter2/section2/sequence_vnpay.jpg}
    \caption{Sơ đồ tuần tự xử lý thanh toán}
    \label{fig:sequence_vnpay}
\end{figure}
\subsubsection{Mục đích}
Sequence diagram này mô tả quy trình thanh toán trực tuyến của hệ thống với hai phương thức: thanh toán khi nhận hàng (COD) và thanh toán qua cổng VNPay. Sơ đồ thể hiện tương tác giữa khách hàng, hệ thống backend và cổng thanh toán VNPay trong suốt quá trình đặt hàng và thanh toán.

\subsubsection{Mô tả luồng hoạt động}
Luồng thanh toán COD
\begin{enumerate}[noitemsep]
    \item Khách hàng chọn phương thức thanh toán COD
    \item Khách hàng xác nhận thanh toán COD
    \item Backend tạo đơn hàng mới
    \item Backend hiển thị kết quả thanh toán COD cho khách hàng
\end{enumerate}

Luồng thanh toán VNPay
\begin{enumerate}[noitemsep]
    \item Khách hàng chọn phương thức thanh toán VNPay
    \item Backend chuyển hướng khách hàng đến cổng thanh toán VNPay
    \item VNPay hiển thị giao diện thanh toán cho khách hàng
    \item Khách hàng thực hiện thanh toán trên VNPay
    \item VNPay trả kết quả giao dịch (thành công/thất bại) về Backend
    \item Backend xử lý kết quả:
    \begin{itemize}[noitemsep]
        \item Nếu thanh toán thành công: Backend hiển thị kết quả thành công
        \item Nếu thanh toán thất bại: Backend hiển thị kết quả thất bại
    \end{itemize}
\end{enumerate}

Xem lịch sử đơn hàng
\begin{enumerate}[noitemsep]
    \item Sau khi thanh toán, khách hàng có thể xem lịch sử đơn hàng
    \item Backend hiển thị danh sách đơn hàng của khách hàng
\end{enumerate}
\subsection{Sơ đồ tuần tự phân tích da mặt}
\begin{figure}[H]
    \centering
    \includegraphics[width=0.8\textwidth]{graphics/main/chapter2/section2/sequence_acne.jpg}
    \caption{Sơ đồ tuần tự phân tích da mặt}
    \label{fig:sequence_acne}
\end{figure}
\subsubsection{Mục đích}
Sequence diagram này mô tả quy trình phân tích da mặt sử dụng trí tuệ nhân tạo. Sơ đồ thể hiện luồng xử lý từ khi người dùng upload ảnh khuôn mặt, qua các bước nhận diện khuôn mặt bằng MediaPipe, phân tích mụn bằng mô hình CNN, đến khi trả về kết quả phân tích cùng lời khuyên và gợi ý sản phẩm phù hợp.

\subsubsection{Mô tả luồng hoạt động}

\begin{enumerate}[noitemsep]
    \item Người dùng upload ảnh khuôn mặt lên hệ thống Backend
    \item Backend gửi ảnh đến MediaPipe để nhận diện khuôn mặt và các điểm landmark
    \item MediaPipe trả về tọa độ khuôn mặt và các điểm landmark
    \item Backend cắt từng vùng da mặt (Trán, Má, Mũi, Cằm) dựa trên landmark
    \item Backend gửi từng vùng da đã cắt đến mô hình CNN để phân tích mụn
    \item CNN Model phát hiện mụn và trả về kết quả (số lượng, vị trí, mức độ)
    \item Backend tổng hợp kết quả từ tất cả các vùng da
    \item Backend trả về kết quả phân tích chi tiết cho người dùng
    \item Backend đưa ra lời khuyên chăm sóc da dựa trên kết quả phân tích
    \item Backend gợi ý các sản phẩm phù hợp với tình trạng da của người dùng
\end{enumerate}

%Seque admin quản lýđánh giá sản phẩm

\subsection{Sơ đồ tuần tự Admin quản lý đánh giá sản phẩm}
\begin{figure}[H]
    \centering
    \includegraphics[width=0.8\textwidth]{graphics/main/chapter2/section2/sequead1.jpg}
    \caption{Admin quản lý đánh giá}
    \label{fig:sequence_adreview}
\end{figure}
 %Mô tả admin đánh giá sản phẩm

\textbf{Mục đích:}

Sơ đồ tuần tự mô tả quá trình Admin tương tác với hệ thống để quản lý đánh giá sản phẩm, bao gồm việc xem danh sách đánh giá, lọc theo tiêu chí và thực hiện xóa đánh giá khi cần thiết. Quá trình này đảm bảo Admin có thể kiểm soát nội dung đánh giá hiển thị trên website.

\textbf{Luồng hoạt động:}

Admin bắt đầu bằng việc mở trang quản lý đánh giá.

\begin{itemize}
    \item Trang quản lý đánh giá gửi yêu cầu lấy danh sách đánh giá đến Hệ thống xử lý.
    \item Hệ thống xử lý thực hiện truy vấn dữ liệu từ cơ sở dữ liệu đánh giá.
    \item Dữ liệu đánh giá được trả về cho Hệ thống xử lý.
    \item Hệ thống gửi kết quả về Trang quản lý để hiển thị danh sách đánh giá cho Admin.
\end{itemize}

\textbf{Trường hợp lọc đánh giá:}

\begin{itemize}
    \item Admin nhập tiêu chí lọc.
    \item Trang quản lý gửi yêu cầu lọc đến Hệ thống xử lý.
    \item Hệ thống truy vấn dữ liệu theo tiêu chí đã nhập.
    \item Kết quả lọc được trả về và danh sách đánh giá được cập nhật trên giao diện.
\end{itemize}

\textbf{Trường hợp xóa đánh giá:}

\begin{itemize}
    \item Admin chọn một đánh giá và nhấn nút ``Xóa''.
    \item Hệ thống hiển thị thông báo xác nhận.
    \item Nếu Admin chọn ``Đồng ý'', Trang quản lý gửi yêu cầu xóa (ID đánh giá) đến Hệ thống xử lý.
    \item Hệ thống xử lý yêu cầu xóa dữ liệu trong cơ sở dữ liệu.
    \item Sau khi xóa thành công, hệ thống trả về kết quả và cập nhật lại danh sách đánh giá.
    \item Giao diện hiển thị thông báo thành công và làm mới danh sách.
\end{itemize}

Quy trình kết thúc khi hệ thống hoàn tất cập nhật và hiển thị dữ liệu mới nhất cho Admin.

%Mô tả hoạt động khách hàng đánh giá sản phẩm

\subsection{Sơ đồ tuần tự đánh giá sản phẩm}
\begin{figure}[H]
    \centering
    \includegraphics[width=0.8\textwidth]{graphics/main/chapter2/section2/sequekh1.jpg}
    \caption{Khách hàng đánh giá sản phẩm}
    \label{fig:sequence_adreview}
\end{figure}
%Mô tả hoạt động khách hàng đánh giá sản phẩm
\subsubsection{Mô tả hoạt động Khách hàng -- Sơ đồ tuần tự đánh giá sản phẩm}

\textbf{Mục đích:}

Sơ đồ tuần tự mô tả quá trình khách hàng thực hiện đánh giá sản phẩm sau khi mua hàng. Quy trình bao gồm kiểm tra điều kiện hợp lệ (đã nhận hàng), nhập nội dung đánh giá, xử lý dữ liệu và lưu trữ vào hệ thống.

\textbf{Luồng hoạt động:}

Khách hàng bắt đầu bằng việc chọn chức năng đánh giá tại trang chi tiết sản phẩm.

\begin{itemize}
    \item Trang chi tiết sản phẩm gửi yêu cầu kiểm tra (UserID, ProductID) đến Hệ thống xử lý.
    \item Hệ thống xử lý kiểm tra trạng thái đơn hàng trong cơ sở dữ liệu.
    \item Nếu đơn hàng ở trạng thái đã nhận hàng, hệ thống trả kết quả hợp lệ và hiển thị form nhập đánh giá.
\end{itemize}

Khách hàng nhập số sao và nội dung bình luận, sau đó nhấn nút ``Gửi đánh giá''.

\begin{itemize}
    \item Trang chi tiết sản phẩm truyền dữ liệu đánh giá đến Hệ thống xử lý.
    \item Hệ thống kiểm tra tính hợp lệ của nội dung (spam, từ cấm, thiếu thông tin,...).
\end{itemize}

\textbf{Trường hợp nội dung hợp lệ:}

\begin{itemize}
    \item Hệ thống lưu thông tin đánh giá mới vào cơ sở dữ liệu.
    \item Cập nhật điểm trung bình của sản phẩm.
    \item Trả về kết quả lưu thành công.
    \item Giao diện hiển thị thông báo ``Cảm ơn bạn đã đánh giá''.
\end{itemize}

\textbf{Trường hợp nội dung không hợp lệ:}

\begin{itemize}
    \item Hệ thống không lưu dữ liệu.
    \item Giao diện hiển thị thông báo lỗi và yêu cầu khách hàng nhập lại nội dung.
\end{itemize}


\chapter{TRIỂN KHAI HỆ THỐNG}
\label{chap:chap4-system-implementation}

\section{Cài đặt Hệ thống}

\subsection{Yêu cầu hệ thống}
\begin{itemize}[noitemsep]
    \item Node.js phiên bản 18.x trở lên
    \item Microsoft SQL Server (LocalDB hoặc Express Edition)
    \item NPM hoặc Yarn package manager
\end{itemize}

\subsection{Backend - Express.js}
Backend được xây dựng trên nền tảng Express.js chạy trên Node.js runtime.

\textbf{Các bước cài đặt:}
\begin{enumerate}[noitemsep]
    \item Clone repository và di chuyển vào thư mục backend
    \item Cài đặt dependencies: npm install
    \item Cấu hình kết nối database trong file .env
    \item Chạy migration để tạo database schema
    \item Khởi động server: npm start hoặc npm run dev
\end{enumerate}

\subsection{Database - Microsoft SQL Server}
Hệ thống sử dụng Microsoft SQL Server LocalDB để lưu trữ dữ liệu.

\textbf{Cấu hình:}
\begin{itemize}[noitemsep]
    \item Server: localhost MSSQLLocalDB
    \item Database name: GlowUpDB
    \item Authentication: Windows Authentication
\end{itemize}

\subsection{Frontend - Nuxt 3.4}
Frontend được phát triển với Nuxt.js phiên bản 3.4, framework Vue.js cho server-side rendering.

\textbf{Các bước cài đặt:}
\begin{enumerate}[noitemsep]
    \item Di chuyển vào thư mục frontend
    \item Cài đặt dependencies: npm install
    \item Cấu hình API endpoint trong file .env
    \item Khởi động development server: npm run dev
    \item Build production: npm run build
\end{enumerate}

\subsection{Khởi động Hệ thống}
\begin{enumerate}[noitemsep]
    \item Khởi động SQL Server
    \item Khởi động Backend server (mặc định port 8081)
    \item Khởi động Frontend server (mặc định port 3000)
    \item Truy cập ứng dụng tại: http://localhost:3000
\end{enumerate}
\section{Kết quả xây dựng hệ thống}

\subsection{Giao diện đăng ký tài khoản}
\begin{figure}[H]
    \centering
    \includegraphics[width=1\textwidth]{graphics/main/chapter3/dk.jpg}
    \caption{Giao diện đăng ký tài khoản}
    \label{fig:register_ui}
\end{figure}




Giao diện đăng ký cho phép người dùng tạo tài khoản mới để sử dụng các chức năng của hệ thống. Giao diện được thiết kế nhằm đảm bảo việc thu thập thông tin người dùng một cách đầy đủ, chính xác và tuân thủ các yêu cầu bảo mật cơ bản.

\textbf{Các thành phần chính:}
\begin{itemize}[noitemsep]
    \item Trường nhập thông tin cá nhân cơ bản của người dùng
    \item Trường nhập Email dùng làm thông tin đăng nhập
    \item Trường nhập mật khẩu
    \item Trường xác nhận lại mật khẩu
    \item Tùy chọn chấp nhận điều khoản sử dụng và chính sách bảo mật
    \item Nút thực hiện đăng ký tài khoản
\end{itemize}

\subsection{Giao diện đăng nhập tài khoản}

\begin{figure}[H]
    \centering
    \includegraphics[width=1\textwidth]{graphics/main/chapter3/dn.jpg}
    \caption{Giao diện đăng nhập tài khoản}
    \label{fig:login_ui}
\end{figure}
\textbf{Các thành phần chính:}
\begin{itemize}[noitemsep]
    \item Trường nhập thông tin định danh người dùng (Email)
    \item Trường nhập mật khẩu với chức năng ẩn/hiện
    \item Tùy chọn ghi nhớ phiên đăng nhập
    \item Chức năng hỗ trợ khôi phục mật khẩu khi người dùng quên thông tin đăng nhập
    \item Nút thực hiện đăng nhập
    \item Tùy chọn đăng nhập thông qua dịch vụ xác thực bên thứ ba
    \item Liên kết điều hướng đến giao diện đăng ký tài khoản mới
\end{itemize}


\subsection{Giao diện trang chủ của khách hàng}
\begin{figure}[H]
    \centering
    \includegraphics[width=1\textwidth]{graphics/main/chapter3/home.jpg}
    \caption{Giao diện trang chủ của khách hàng}
    \label{fig:home_ui}
\end{figure}
Trang chủ là giao diện đầu tiên mà người dùng tiếp cận khi truy cập vào website GlowUp, có vai trò giới thiệu tổng quan hệ thống và định hướng người dùng đến các chức năng chính. Giao diện được thiết kế theo hướng thân thiện với người dùng, hỗ trợ người dùng nhanh chóng tiếp cận thông tin và thực hiện các thao tác mua sắm.

Phần đầu trang cung cấp thông tin thương hiệu và hệ thống điều hướng, cho phép người dùng truy cập nhanh đến các khu vực chức năng quan trọng của hệ thống. Khu vực nội dung chính tập trung giới thiệu các sản phẩm và danh mục nổi bật, hỗ trợ người dùng khám phá và lựa chọn sản phẩm phù hợp với nhu cầu.

Ngoài ra, hệ thống tích hợp công cụ hỗ trợ trực tuyến nhằm hỗ trợ người dùng trong quá trình tìm kiếm thông tin và mua sắm. Thiết kế tổng thể của trang chủ hướng đến việc nâng cao trải nghiệm người dùng và tăng khả năng chuyển đổi từ truy cập sang hành động mua hàng.

\subsection{Giao diện giới thiệu website}
\begin{figure}[H]
    \centering
    \includegraphics[width=1\textwidth]{graphics/main/chapter3/gioithieu.jpg}
    \caption{Giao diện giới thiệu website}
    \label{fig:intro_ui}
\end{figure}
Trang giới thiệu được xây dựng nhằm cung cấp thông tin tổng quan về thương hiệu và định hướng hoạt động của hệ thống, góp phần xây dựng niềm tin đối với người dùng. Nội dung trang tập trung trình bày về mục tiêu phát triển, giá trị cốt lõi và cam kết chất lượng dịch vụ của GlowUp.

Trang giới thiệu cung cấp thông tin về quá trình hình thành và phát triển của hệ thống, năng lực chuyên môn và định hướng phục vụ khách hàng. Thông tin được trình bày một cách trực quan, giúp người dùng dễ dàng nắm bắt các giá trị nổi bật của thương hiệu.

Bên cạnh đó, trang giới thiệu làm rõ các điểm khác biệt và lợi ích khi người dùng lựa chọn mua sắm trên hệ thống, qua đó góp phần nâng cao mức độ tin cậy và thúc đẩy quyết định sử dụng dịch vụ.


\subsection{Giao diện liên hệ}
\begin{figure}[H]
    \centering
    \includegraphics[width=1\textwidth]{graphics/main/chapter3/lienhe.jpg}
    \caption{Giao diện liên hệ}
    \label{fig:contact_ui}
\end{figure}

Trang liên hệ được thiết kế nhằm cung cấp cho người dùng các kênh thông tin để kết nối và trao đổi với đội ngũ hỗ trợ của hệ thống. Giao diện trang bao gồm các thông tin liên lạc cơ bản như địa chỉ email, số điện thoại và địa chỉ văn phòng, giúp người dùng dễ dàng tìm kiếm và sử dụng khi cần hỗ trợ.
\subsection{Giao diện danh mục sản phẩm}
\begin{figure}[H]
    \centering
    \includegraphics[width=1\textwidth]{graphics/main/chapter3/danhmuc.jpg}
    \caption{Giao diện danh mục sản phẩm}
    \label{fig:category_ui}
\end{figure}
Trang danh mục sản phẩm được thiết kế nhằm hỗ trợ người dùng dễ dàng duyệt và lựa chọn sản phẩm theo từng nhóm chức năng. Giao diện hiển thị các danh mục sản phẩm chính của hệ thống, giúp người dùng nhanh chóng tiếp cận đúng nhóm sản phẩm phù hợp với nhu cầu.

Các danh mục được trình bày dưới dạng các khối hiển thị trực quan, cho phép người dùng truy cập vào danh sách sản phẩm chi tiết của từng danh mục. Cách bố trí giao diện đảm bảo tính rõ ràng, dễ sử dụng và tương thích trên nhiều thiết bị, góp phần nâng cao trải nghiệm người dùng trong quá trình tìm kiếm sản phẩm.
\subsection{Giao diện sản phẩm}
\begin{figure}[H]
    \centering
    \includegraphics[width=1\textwidth]{graphics/main/chapter3/sp.jpg}
    \caption{Giao diện sản phẩm }
    \label{fig:product_ui}
\end{figure}
Trang sản phẩm cung cấp danh sách tổng hợp các sản phẩm hiện có trong hệ thống. Người dùng có thể duyệt sản phẩm theo danh mục và sử dụng các chức năng lọc, sắp xếp để thu hẹp phạm vi tìm kiếm theo nhu cầu.

Giao diện hiển thị thông tin cơ bản của từng sản phẩm như hình ảnh, tên và giá bán, giúp người dùng nhanh chóng so sánh và lựa chọn sản phẩm phù hợp.


\subsection{Giao diện sản phẩm chi tiết}
\begin{figure}[H]
    \centering
    \includegraphics[width=1\textwidth]{graphics/main/chapter3/sp_detail.jpg}
    \caption{Giao diện sản phẩm chi tiết}
    \label{fig:product_detail_ui}
\end{figure}
Trang chi tiết sản phẩm cung cấp đầy đủ thông tin về một sản phẩm cụ thể, bao gồm hình ảnh minh họa, thông tin thương hiệu, giá bán, đánh giá người dùng và mô tả chi tiết về thành phần, công dụng, hướng dẫn sử dụng.

Giao diện được thiết kế theo bố cục rõ ràng, hỗ trợ người dùng nắm bắt thông tin nhanh chóng, từ đó đưa ra quyết định mua hàng chính xác.


\subsection{Giao diện đánh giá sản phẩm}
\begin{figure}[H]
    \centering
    \includegraphics[width=1\textwidth]{graphics/main/chapter3/cmt.jpg}
    \caption{Giao diện đánh giá sản phẩm}
    \label{fig:review_ui}
\end{figure}
Chức năng đánh giá sản phẩm cho phép người dùng chia sẻ nhận xét và mức độ hài lòng sau khi sử dụng sản phẩm. Hệ thống hỗ trợ xem tổng quan điểm đánh giá, lọc đánh giá theo mức sao và gửi đánh giá mới.

Phần đánh giá góp phần tăng tính minh bạch của hệ thống, đồng thời hỗ trợ người mua tham khảo ý kiến từ cộng đồng trước khi đưa ra quyết định mua hàng.
\subsection{Giao diện gợi ý sản phẩm}
\begin{figure}[H]
    \centering
    \includegraphics[width=1\textwidth]{graphics/main/chapter3/goi_y_sp.jpg}
    \caption{Giao diện gợi ý sản phẩm}
    \label{fig:suggested_product_ui}
\end{figure}
Trang gợi ý sản phẩm được hiển thị trên trang chi tiết sản phẩm nhằm gợi ý các sản phẩm liên quan hoặc thay thế. Chức năng này giúp mở rộng lựa chọn cho người dùng và hỗ trợ tăng khả năng mua thêm sản phẩm.

Việc hiển thị các sản phẩm liên quan góp phần nâng cao trải nghiệm người dùng và hiệu quả kinh doanh của hệ thống thông qua cơ chế gợi ý.

\subsection{Giao diện giỏ hàng}
\begin{figure}[H]
    \centering
    \includegraphics[width=1\textwidth]{graphics/main/chapter3/gio_hang.jpg}
    \caption{Giao diện giỏ hàng}
    \label{fig:cart_ui}
\end{figure}
Trang giỏ hàng cho phép người dùng xem lại các sản phẩm đã thêm vào trước khi tiến hành thanh toán. Giao diện được thiết kế trực quan với hai khu vực chính gồm danh sách sản phẩm và phần tóm tắt đơn hàng.

Danh sách sản phẩm hiển thị thông tin cơ bản của từng sản phẩm như hình ảnh, tên sản phẩm, giá và số lượng. Người dùng có thể điều chỉnh số lượng sản phẩm, chọn hoặc bỏ chọn sản phẩm cần mua, cũng như xóa sản phẩm khỏi giỏ hàng. Chức năng chọn tất cả giúp người dùng thao tác nhanh khi giỏ hàng có nhiều sản phẩm.

Phần tóm tắt đơn hàng hiển thị các thông tin tổng hợp như số lượng sản phẩm đã chọn, tạm tính, phí vận chuyển và tổng tiền thanh toán. Nút chuyển sang bước thanh toán được đặt nổi bật nhằm hỗ trợ người dùng tiếp tục quy trình mua sắm một cách thuận tiện.

\subsection{Giao diện thanh toán}
\begin{figure}[H]
    \centering
    \includegraphics[width=1\textwidth]{graphics/main/chapter3/donhang.jpg}
    \caption{Giao diện thanh toán}
    \label{fig:order_ui}
\end{figure}
Trang thanh toán cho phép người dùng xác nhận thông tin giao hàng và lựa chọn phương thức thanh toán trước khi hoàn tất đơn hàng. Giao diện được bố trí khoa học, giúp người dùng dễ dàng kiểm tra lại thông tin cá nhân, địa chỉ nhận hàng và phương thức thanh toán đã chọn.

Hệ thống hỗ trợ nhiều hình thức thanh toán khác nhau như thanh toán khi nhận hàng (COD) và thanh toán trực tuyến qua cổng trung gian. Ngoài ra, người dùng có thể áp dụng mã giảm giá (nếu có) để giảm giá trị đơn hàng. Thông tin tóm tắt đơn hàng được hiển thị rõ ràng để người dùng kiểm tra lại tổng số tiền cần thanh toán trước khi xác nhận đặt hàng.

\subsection{Giao diện quản lý đơn hàng}
\begin{figure}[H]
    \centering
    \includegraphics[width=1\textwidth]{graphics/main/chapter3/lichsu.jpg}
    \caption{Giao diện quản lý đơn hàng}
    \label{fig:order_history_ui}
\end{figure}
Trang quản lý đơn hàng cho phép người dùng theo dõi danh sách các đơn hàng đã đặt và trạng thái xử lý của từng đơn. Giao diện hỗ trợ lọc đơn hàng theo trạng thái nhằm giúp người dùng dễ dàng tra cứu và quản lý lịch sử mua sắm.

Mỗi đơn hàng được hiển thị kèm theo các thông tin cơ bản như mã đơn hàng, thời gian đặt hàng, trạng thái xử lý, phương thức thanh toán và tổng giá trị đơn hàng. Tại mỗi đơn hàng, người dùng có thể thực hiện một số thao tác phù hợp với trạng thái hiện tại của đơn, chẳng hạn như hủy đơn khi đơn chưa được xử lý hoặc tiếp tục thanh toán đối với các đơn chưa hoàn tất giao dịch.
\subsection{Giao diện chuyển giao diện sang vnpay để thanh toán}
\begin{figure}[H]
    \centering
    \includegraphics[width=1\textwidth]{graphics/main/chapter3/vnpay.jpg}
    \caption{Giao diện chuyển giao diện sang vnpay để thanh toán}
    \label{fig:vnpay_ui}
\end{figure}
\begin{figure}[H]
    \centering
    \includegraphics[width=1\textwidth]{graphics/main/chapter3/tinh_vnpay.jpg}
    \caption{Giao diện nhập tài khoản thanh toán}
    \label{fig:payment_account_ui}
\end{figure}
Khi người dùng lựa chọn thanh toán trực tuyến qua VNPay, hệ thống sẽ chuyển hướng đến cổng thanh toán của VNPay. Giao diện thanh toán được thiết kế theo tiêu chuẩn bảo mật, cung cấp nhiều phương thức thanh toán khác nhau như thẻ nội địa, thẻ quốc tế và các hình thức thanh toán điện tử.

Trang thanh toán hiển thị đầy đủ thông tin đơn hàng để người dùng kiểm tra trước khi xác nhận giao dịch. Quá trình thanh toán được thực hiện thông qua các bước xác thực nhằm đảm bảo tính an toàn và bảo mật thông tin cho người dùng.





\subsection{Giao diện AI chatbot chăm sóc khách hàng}
\begin{figure}[H]
    \centering
    \includegraphics[width=1\textwidth]{graphics/main/chapter3/AI_chatbot.jpg}
    \caption{Giao diện AI chatbot chăm sóc khách hàng}
    \label{fig:chatbot_ui}
\end{figure}
Hệ thống tích hợp chatbot AI nhằm hỗ trợ tư vấn sản phẩm và giải đáp thắc mắc cho người dùng trong quá trình mua sắm. Chatbot được hiển thị dưới dạng cửa sổ trò chuyện trên giao diện website, cho phép người dùng đặt câu hỏi và nhận phản hồi theo thời gian thực.

Dựa trên nội dung tương tác và nhu cầu của người dùng, chatbot có thể gợi ý các sản phẩm phù hợp, cung cấp thông tin cơ bản về sản phẩm và hỗ trợ người dùng truy cập nhanh đến trang chi tiết sản phẩm. Chức năng chatbot góp phần nâng cao trải nghiệm người dùng và hỗ trợ quá trình ra quyết định mua hàng một cách hiệu quả.

\subsection{Giao diện AI phân tích da mặt}
\begin{figure}[H]
    \centering
    \includegraphics[width=1\textwidth]{graphics/main/chapter3/AI_acne.jpg}
    \caption{Giao diện AI phân tích da mặt}
    \label{fig:skin_analysis_ui}
\end{figure}
Giao diện phân tích da mặt sử dụng trí tuệ nhân tạo nhằm giúp người dùng nhận biết tình trạng da và mức độ mụn thông qua hình ảnh khuôn mặt được tải lên hệ thống. Giao diện được thiết kế đơn giản, thân thiện với người dùng, bao gồm các thành phần chính như sau:
\subsection{Giao diện kết quả phân tích da mặt}
\begin{figure}[H]
    \centering
    \includegraphics[width=1\textwidth]{graphics/main/chapter3/ketqua_acne.jpg}
    \caption{Giao diện kết quả phân tích da mặt}
    \label{fig:analysis_result_ui}
\end{figure}
Giao diện kết quả phân tích da mặt hiển thị chi tiết tình trạng da và mức độ mụn sau khi người dùng tải ảnh khuôn mặt lên hệ thống. Kết quả được trình bày rõ ràng, bao gồm các thông tin như số lượng mụn trên từng vùng da (trán, má, mũi, cằm), mức độ mụn tổng thể và lời khuyên chăm sóc da phù hợp.






\subsection{Giao diện dasboard admin}
\begin{figure}[H]
    \centering
    \includegraphics[width=1\textwidth]{graphics/main/chapter3/admin_dasboard.jpg}
    \caption{Giao diện dashboard admin}
    \label{fig:admin_dashboard_ui}
\end{figure}
Giao diện Dashboard dành cho quản trị viên cung cấp cái nhìn tổng thể về tình hình hoạt động của hệ thống thông qua các chỉ số thống kê trực quan. Tại đây, quản trị viên có thể theo dõi nhanh các thông số quan trọng như tổng doanh thu, số lượng đơn hàng mới, số lượng khách hàng đăng ký và các biểu đồ tăng trưởng theo thời gian.

Các biểu đồ được thiết kế sinh động, hỗ trợ việc phân tích dữ liệu và giúp người quản lý đưa ra các quyết định kinh doanh kịp thời. Ngoài ra, Dashboard còn hiển thị danh sách các hoạt động gần đây và trạng thái của các tiến trình quan trọng trong hệ thống.



\subsection{Giao diện quản lý sản phẩm của admin}
\begin{figure}[H]
    \centering
    \includegraphics[width=1\textwidth]{graphics/main/chapter3/admin_sp.jpg}
    \caption{Giao diện quản lý sản phẩm của admin}
    \label{fig:admin_product_ui}
\end{figure}
Giao diện quản lý sản phẩm cho phép quản trị viên thực hiện các thao tác quản lý danh sách sản phẩm hiện có trên hệ thống. Chức năng chính bao gồm thêm mới sản phẩm, chỉnh sửa thông tin sản phẩm (tên, giá, mô tả, hình ảnh, tồn kho) và xóa hoặc ẩn các sản phẩm không còn kinh doanh.

Hệ thống hỗ trợ tìm kiếm sản phẩm theo tên và lọc theo danh mục, giúp việc quản lý kho hàng trở nên thuận tiện và chính xác. Quản trị viên có thể cập nhật nhanh trạng thái sản phẩm để đảm bảo thông tin hiển thị cho khách hàng luôn là mới nhất.



\subsection{Giao diện quản lý đơn hàng của admin}
\begin{figure}[H]
    \centering
    \includegraphics[width=1\textwidth]{graphics/main/chapter3/admin_order.jpg}
    \caption{Giao diện quản lý đơn hàng của admin}
    \label{fig:admin_order_ui}
\end{figure}
Giao diện quản lý đơn hàng là nơi quản trị viên theo dõi và xử lý toàn bộ các giao dịch phát sinh trên hệ thống. Danh sách đơn hàng được hiển thị kèm theo thông tin chi tiết về khách hàng, tổng tiền, phương thức thanh toán và trạng thái xử lý hiện tại (chờ xác nhận, đang giao, đã hoàn thành, đã hủy).

Quản trị viên có thể xác nhận đơn hàng, cập nhật trạng thái vận chuyển và xem chi tiết từng đơn hàng để đảm bảo quy trình phục vụ khách hàng được diễn ra trôi chảy. Công cụ lọc theo trạng thái và thời gian hỗ trợ việc tra cứu và thống kê đơn hàng một cách hiệu quả.






\subsection{Giao diện quản lý danh mục sản phẩm}
\begin{figure}[H]
    \centering
    \includegraphics[width=1\textwidth]{graphics/main/chapter3/admin_danhmuc.jpg}
    \caption{Giao diện quản lý danh mục sản phẩm}
    \label{fig:admin_category_ui}
\end{figure}
Giao diện này cho phép quản trị viên quản lý cấu trúc cây danh mục của website. Các chức năng bao gồm tạo mới danh mục, cập nhật tên, mô tả và hình ảnh minh họa cho danh mục. Việc tổ chức danh mục khoa học giúp khách hàng dễ dàng tìm kiếm và tiếp cận sản phẩm, đồng thời tối ưu hóa giao diện hiển thị cho hệ thống.




\subsection{Giao diện quản lý đánh giá sản phẩm của admin}
\begin{figure}[H]
    \centering
    \includegraphics[width=1\textwidth]{graphics/main/chapter3/admin_cmt.jpg}
    \caption{Giao diện quản lý đánh giá sản phẩm của admin}
    \label{fig:admin_review_ui}
\end{figure}
Trang quản lý đánh giá giúp quản trị viên theo dõi các phản hồi từ khách hàng về sản phẩm. Quản trị viên có thể xem nội dung nhận xét, số sao đánh giá và phản hồi lại các ý kiến của khách hàng. Chức năng này cũng hỗ trợ việc kiểm duyệt các nội dung không phù hợp nhằm duy trì môi trường mua sắm văn minh và tin cậy.




\subsection{Giao diện quản lý vận chuyển của admin}
\begin{figure}[H]
    \centering
    \includegraphics[width=1\textwidth]{graphics/main/chapter3/admin_vanchuyen.jpg}
    \caption{Giao diện quản lý vận chuyển của admin}
    \label{fig:admin_shipping_ui}
\end{figure}
Giao diện quản lý vận chuyển cho phép quản trị viên thiết lập các đơn vị vận chuyển và quản lý trạng thái luân chuyển của các gói hàng. Hệ thống hiển thị danh sách các phiếu vận chuyển kèm theo mã vận đơn và trạng thái tương ứng, giúp người quản lý kiểm soát chặt chẽ quá trình giao hàng đến tay người tiêu dùng.


\subsection{Giao diện chi tiết vận chuyển của admin}
\begin{figure}[H]
    \centering
    \includegraphics[width=1\textwidth]{graphics/main/chapter3/admin_detail_vanchuyen.jpg}
    \caption{Giao diện chi tiết vận chuyển của admin}
    \label{fig:admin_shipping_detail_ui}
\end{figure}
Trang chi tiết vận chuyển cung cấp thông tin cụ thể về lộ trình và các mốc thời gian xử lý của một đơn hàng nhất định. Quản trị viên có thể theo dõi vị trí hiện tại của đơn hàng, các ghi chú từ đơn vị vận chuyển và thời gian dự kiến giao hàng thành công, đảm bảo thông tin luôn được cập nhật kịp thời cho khách hàng.


\subsection{Giao diện quản lý khuyến mãi của admin}
\begin{figure}[H]
    \centering
    \includegraphics[width=1\textwidth]{graphics/main/chapter3/admin_voucher.jpg}
    \caption{Giao diện quản lý khuyến mãi của admin}
    \label{fig:admin_promotion_ui}
\end{figure}
Giao diện quản lý khuyến mãi cho phép quản trị viên tạo và quản lý các mã giảm giá (vouchers), các chương trình ưu đãi đặc biệt. Quản trị viên có thể thiết lập giá trị giảm giá, thời hạn áp dụng, số lượng mã phát hành và các điều kiện đi kèm. Đây là công cụ quan trọng trong việc thúc đẩy doanh số và thu hút khách hàng quay lại mua sắm.


\subsection{Giao diện quản lý cửa hàng }
\begin{figure}[H]
    \centering
    \includegraphics[width=1\textwidth]{graphics/main/chapter3/admin_cuahang.jpg}
    \caption{Giao diện quản lý cửa hàng}
    \label{fig:admin_store_ui}
\end{figure}
Trang quản lý cửa hàng hỗ trợ quản trị viên quản lý thông tin về các địa điểm hoặc chi nhánh cửa hàng vật lý. Các thông tin như tên cửa hàng, địa chỉ, số điện thoại liên hệ và tọa độ vị trí được thiết lập tại đây, giúp hệ thống đồng bộ dữ liệu và hỗ trợ khách hàng tìm kiếm cửa hàng gần nhất một cách dễ dàng.


\chapter{KẾT LUẬN VÀ HƯỚNG PHÁT TRIỂN}
\label{chap:chap4-conclusion-and-future-development}

\section{Kết luận}
Trải qua quá trình nghiên cứu, phân tích và triển khai thực tế, đề tài “Xây dựng website bán mỹ phẩm kết hợp AI hỗ trợ tư vấn da mặt trị mụn và chatbot gợi ý sản phẩm” đã hoàn thành đúng tiến độ và đạt được các mục tiêu trọng tâm đề ra tại Chương 1. Đề tài đã hoàn thành việc xây dựng một website thương mại điện tử tích hợp trí tuệ nhân tạo. Kết quả cho thấy khả năng ứng dụng AI vào lĩnh vực tư vấn sản phẩm và chăm sóc khách hàng.

Cụ thể, nhóm đã đạt được các thành tựu sau:
\begin{itemize}[noitemsep]
    \item \textbf{Về mặt hạ tầng và công nghệ:} Xây dựng thành công hệ thống Full-stack hoàn chỉnh. Sử dụng Nuxt cho Frontend giúp tối ưu hóa hiệu năng hiển thị và trải nghiệm người dùng; Express.js đóng vai trò là một Backend mạnh mẽ, xử lý logic nghiệp vụ mạch lạc; và Microsoft SQL Server đảm bảo tính toàn vẹn, bảo mật cho dữ liệu kinh doanh của cửa hàng.

    \item \textbf{Về nghiệp vụ thương mại điện tử:} Hệ thống đã mô phỏng hoàn thiện quy trình mua sắm trực tuyến hiện đại, từ khâu duyệt sản phẩm theo danh mục, quản lý giỏ hàng linh hoạt đến tích hợp thành công cổng thanh toán VNPAY -- một giải pháp thanh toán điện tử phổ biến tại Việt Nam, giúp tối ưu hóa tỷ lệ chuyển đổi và tăng tính chuyên nghiệp cho cửa hàng tư nhân.

    \item \textbf{Về đột phá công nghệ AI:} Đây là điểm sáng nhất của đồ án. Nhóm đã thực hiện hóa thành công chức năng phân tích da mặt trực tuyến. Bằng cách kết hợp MediaPipe để nhận diện các điểm đặc trưng (landmarks) và mô hình CNN (Convolutional Neural Network) để phân tích tình trạng mụn, hệ thống đã cung cấp một công cụ hỗ trợ tư vấn khách quan, khoa học. Chatbot AI đi kèm không chỉ giải đáp thắc mắc mà còn đóng vai trò như một nhân viên tư vấn ảo, gợi ý sản phẩm dựa trên chính kết quả phân tích da của người dùng.

    \item \textbf{Về giá trị thực tiễn:} Website GlowUp đã góp phần giải quyết bài toán chuyển đổi số cho các cửa hàng mỹ phẩm tư nhân, giúp chủ cửa hàng quản lý dữ liệu tập trung, theo dõi doanh thu trực quan và tiếp cận khách hàng hiệu quả hơn thông qua các công nghệ thông minh.
\end{itemize}

\section{Hạn chế}
Mặc dù đã nỗ lực hoàn thiện, tuy nhiên do hạn chế về thời gian cũng như kiến thức chuyên sâu trong lĩnh vực thị giác máy tính, hệ thống vẫn khó tránh khỏi một số thiếu sót cần nhìn nhận khách quan:
\begin{itemize}[noitemsep]
    \item \textbf{Độ chính xác của mô hình AI:} Kết quả phân tích da mặt vẫn phụ thuộc khá nhiều vào các yếu tố ngoại cảnh như chất lượng camera của thiết bị, cường độ ánh sáng môi trường và góc chụp của người dùng. Trong một số điều kiện ánh sáng phức tạp, mô hình có thể nhầm lẫn giữa các loại mụn hoặc vết thâm.

    \item \textbf{Sự đa dạng của tập dữ liệu (Dataset):} Tập dữ liệu dùng để huấn luyện mô hình CNN hiện tại chủ yếu tập trung vào các loại mụn phổ biến. Hệ thống chưa có khả năng nhận diện sâu các bệnh lý da liễu phức tạp hơn như viêm da cơ địa, nám nội tiết hay các dấu hiệu lão hóa sớm.

    \item \textbf{Tối ưu hóa quy trình vận chuyển:} Hệ thống hiện mới chỉ dừng lại ở việc cập nhật trạng thái đơn hàng thủ công từ phía Admin, chưa tích hợp API thực tế với các đơn vị vận chuyển (như Giao Hàng Nhanh, Viettel Post) để tính phí tự động và theo dõi lộ trình thời gian thực.

    \item \textbf{Khả năng chịu tải (Scalability):} Mặc dù SQL Server xử lý tốt dữ liệu hiện tại, nhưng khi số lượng người dùng đồng thời tăng lên hàng chục ngàn người cùng lúc thực hiện phân tích ảnh AI, hệ thống có thể đối mặt với thách thức về tài nguyên phần cứng và độ trễ phản hồi.
\end{itemize}

\section{Hướng phát triển}
Từ những kết quả thu được và các hạn chế nêu trên, nhóm đã định hướng các bước phát triển tiếp theo để đưa GlowUp trở thành một nền tảng thương mại điện tử AI toàn diện:
\begin{itemize}[noitemsep]
    \item \textbf{Nâng cấp lõi công nghệ AI:} Nghiên cứu và áp dụng các kiến trúc mạng thần kinh tiên tiến hơn nhằm tăng độ chính xác của hệ thống. Đồng thời, mở rộng khả năng phân tích sang nhiều tình trạng da khác như lỗ chân lông, độ ẩm và nếp nhăn để từng bước xây dựng một “bác sĩ da liễu ảo” hỗ trợ người dùng.

    \item \textbf{Cá nhân hóa trải nghiệm khách hàng:} Xây dựng hệ thống gợi ý (Recommendation System) dựa trên lịch sử mua hàng và lịch sử kết quả soi da. Hệ thống có thể tự động gửi thông báo nhắc nhở người dùng chăm sóc da hoặc gợi ý các sản phẩm bổ sung khi sản phẩm cũ sắp hết.

    \item \textbf{Mở rộng nền tảng (Multi-platform):} Phát triển ứng dụng di động trên nền tảng Flutter (tận dụng kiến thức sẵn có của nhóm) để người dùng có thể chụp ảnh soi da trực tiếp từ điện thoại một cách thuận tiện, đồng thời khai thác hiệu quả các cảm biến hình ảnh chất lượng cao trên smartphone.

    \item \textbf{Tích hợp hệ sinh thái Logistics và Tài chính:} Kết nối trực tiếp với các đơn vị giao nhận và mở rộng thêm các hình thức thanh toán ví điện tử khác như Momo, ShopeePay, cũng như các giải pháp “Mua trước trả sau” (BNPL) nhằm tối ưu hóa trải nghiệm mua sắm cho người dùng.

    \item \textbf{Xây dựng cộng đồng:} Tích hợp thêm chuyên mục Blog chia sẻ kiến thức làm đẹp và diễn đàn người dùng, nơi khách hàng có thể chia sẻ lộ trình trị mụn thành công nhờ sự hỗ trợ của AI GlowUp, từ đó gia tăng độ uy tín và mức độ gắn kết với thương hiệu.
\end{itemize}

\end{spacing}

% ===== TÀI LIỆU THAM KHẢO =====
\clearpage
\begin{thebibliography}{99}
\addcontentsline{toc}{chapter}{TÀI LIỆU THAM KHẢO}
\fontfamily{ptm}\selectfont % Times New Roman

\bibitem{nuxt_assets}
Nuxt, ``Assets,'' Nuxt Documentation, 2024. [Online]. Available: \url{https://nuxt.com/docs/4.x/getting-started/assets}. [Accessed: Dec. 20, 2025].

\bibitem{nodejs_express_dailydev}
Daily.dev, ``Setup Node.js Express project: A beginner's guide,'' Daily.dev Blog, 2024. [Online]. Available: \url{https://daily.dev/blog/setup-nodejs-express-project-a-beginners-guide}. [Accessed: Dec. 20, 2025].

\bibitem{rag_docker_ray_qdrant}
I. A. Sarthak, ``Implementing a full-stack production RAG with Docker, Ray, Qdrant, and LM Studio,'' Medium, 2024. [Online]. Available: \url{https://iasarthak.medium.com/implementing-a-full-stack-production-rag-with-docker-ray-qdrant-and-lm-studio-b0441ef52420}. [Accessed: Dec. 20, 2025].

\bibitem{acne_kaggle_dataset}
Kaggle, ``Acne Dataset,'' Kaggle Datasets, 2024. [Online]. Available: \url{https://www.kaggle.com/datasets/nayanchaure/acne-dataset}. [Accessed: Dec. 20, 2025].

\bibitem{sql_database_projects_ms}
Microsoft, ``Get started with SQL database projects,'' Microsoft Learn, 2024. [Online]. Available: \url{https://learn.microsoft.com/en-us/sql/tools/sql-database-projects/get-started?view=sql-server-ver17&pivots=sq1-visual-studio}. [Accessed: Dec. 20, 2025].

\end{thebibliography}
\appendix
\chapter*{Phụ lục}
\addcontentsline{toc}{chapter}{Phụ lục}

\section*{Mã nguồn dự án}
\addcontentsline{toc}{section}{Mã nguồn dự án}

Toàn bộ mã nguồn của hệ thống được quản lý trên GitHub, bao gồm các repository sau:

\textbf{Backend Repository:}
\begin{itemize}[noitemsep]
    \item URL: https://github.com/trongvu2003/GlowUp\_Web\_BE.git
    \item Mô tả: Mã nguồn backend được phát triển bằng Express.js, xử lý API, tích hợp database, VNPay, Ollama chatbot và AI services
    \item Công nghệ: Node.js, Express.js, Microsoft SQL Server, Qdrant
\end{itemize}

\textbf{Frontend Repository:}
\begin{itemize}[noitemsep]
    \item URL: https://github.com/Toru2004/GlowUp\_Web.git
    \item Mô tả: Mã nguồn frontend được phát triển bằng Nuxt.js 3.4, cung cấp giao diện người dùng cho hệ thống
    \item Công nghệ: Nuxt.js 3.4, Vue.js, TailwindCSS
\end{itemize}

\section*{Bộ dữ liệu}
\addcontentsline{toc}{section}{Bộ dữ liệu}

Dự án sử dụng bộ dữ liệu công khai để huấn luyện mô hình CNN phân loại mụn:

\textbf{Acne vs Non-Acne Dataset:}
\begin{itemize}[noitemsep]
    \item Nguồn: Kaggle
    \item URL: https://www.kaggle.com/datasets/captainsengupta2000/acne-vs-non-acne-5
    \item Tác giả: Captain Sengupta
    \item Mô tả: Bộ dữ liệu chứa hình ảnh da có mụn và không có mụn, được sử dụng để huấn luyện mô hình CNN phân loại
    \item Số lượng ảnh: 10 000 ảnh
    \item Định dạng: JPEG/PNG images
\end{itemize}

\section*{Công cụ và thư viện}
\addcontentsline{toc}{section}{Công cụ và thư viện}

Các công cụ và thư viện chính được sử dụng trong dự án:

\textbf{Backend:}
\begin{itemize}[noitemsep]
    \item Express.js - Web framework
    \item Sequelize - ORM cho SQL Server
    \item MSSQL - Database driver
    \item Qdrant Client - Vector database client
    \item Ollama - LLM integration
\end{itemize}

\textbf{Frontend:}
\begin{itemize}[noitemsep]
    \item Nuxt.js 3.4 - Vue.js framework
    \item TailwindCSS - UI styling
    \item Axios - HTTP client
    \item Pinia - State management
\end{itemize}

\textbf{AI/ML:}
\begin{itemize}[noitemsep]
    \item PyTorch - Deep learning framework
    \item Flask/FastAPI - Python web framework
    \item OpenCV - Image processing
    \item NumPy - Numerical computing
    \item Pillow - Image handling
\end{itemize}

\section*{Tài liệu tham khảo kỹ thuật}
\addcontentsline{toc}{section}{Tài liệu tham khảo kỹ thuật}

Các tài liệu và API documentation được sử dụng:

\begin{itemize}[noitemsep]
    \item Express.js Documentation: https://expressjs.com/
    \item Nuxt.js Documentation: https://nuxt.com/
    \item Qdrant Documentation: https://qdrant.tech/documentation/
    \item Ollama Documentation: https://ollama.com/docs
    \item VNPay API Documentation: Tài liệu sandbox từ VNPay
    \item PyTorch Documentation: https://pytorch.org/docs/
\end{itemize}

\section*{Môi trường phát triển}
\addcontentsline{toc}{section}{Môi trường phát triển}

Thông tin về môi trường phát triển được sử dụng:

\begin{itemize}[noitemsep]
    \item IDE: Visual Studio Code
    \item Version Control: Git
    \item Package Manager: NPM (Node.js), pip (Python)
    \item Container Platform: Docker Desktop
    \item Testing Tool: Postman (API testing)
    \item Database Management: Azure Data Studio, SQL Server Management Studio
\end{itemize}

\section*{Cấu trúc thư mục dự án}
\addcontentsline{toc}{section}{Cấu trúc thư mục dự án}

\textbf{Backend Structure:}
\begin{itemize}[noitemsep]
    \item /src - Source code chính
    \item /config - Cấu hình hệ thống
    \item /models - Database models
    \item /controllers - Request handlers
    \item /services - Business logic
    \item /routes - API routes
    \item /middleware - Middleware functions
    \item /utils - Utility functions
\end{itemize}

\textbf{Frontend Structure:}
\begin{itemize}[noitemsep]
    \item /pages - Nuxt pages
    \item /components - Vue components
    \item /layouts - Layout templates
    \item /composables - Composition functions
    \item /stores - Pinia stores
    \item /assets - Static assets
    \item /public - Public files
\end{itemize}

\textbf{AI Model Structure:}
\begin{itemize}[noitemsep]
    \item /api - API endpoints
    \item /config - Cấu hình hệ thống
    \item /model - Trained models
    \item /schemas - Data validation schemas
    \item /services - Business logic và service layer
\end{itemize}

\end{document}