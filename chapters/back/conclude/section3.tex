\section{Hướng phát triển}
Từ những kết quả thu được và các hạn chế nêu trên, nhóm đã định hướng các bước phát triển tiếp theo để đưa GlowUp trở thành một nền tảng thương mại điện tử AI toàn diện:
\begin{itemize}[noitemsep]
    \item \textbf{Nâng cấp lõi công nghệ AI:} Nghiên cứu và áp dụng các kiến trúc mạng thần kinh tiên tiến hơn nhằm tăng độ chính xác của hệ thống. Đồng thời, mở rộng khả năng phân tích sang nhiều tình trạng da khác như lỗ chân lông, độ ẩm và nếp nhăn để từng bước xây dựng một “bác sĩ da liễu ảo” hỗ trợ người dùng.

    \item \textbf{Cá nhân hóa trải nghiệm khách hàng:} Xây dựng hệ thống gợi ý (Recommendation System) dựa trên lịch sử mua hàng và lịch sử kết quả soi da. Hệ thống có thể tự động gửi thông báo nhắc nhở người dùng chăm sóc da hoặc gợi ý các sản phẩm bổ sung khi sản phẩm cũ sắp hết.

    \item \textbf{Mở rộng nền tảng (Multi-platform):} Phát triển ứng dụng di động trên nền tảng Flutter (tận dụng kiến thức sẵn có của nhóm) để người dùng có thể chụp ảnh soi da trực tiếp từ điện thoại một cách thuận tiện, đồng thời khai thác hiệu quả các cảm biến hình ảnh chất lượng cao trên smartphone.

    \item \textbf{Tích hợp hệ sinh thái Logistics và Tài chính:} Kết nối trực tiếp với các đơn vị giao nhận và mở rộng thêm các hình thức thanh toán ví điện tử khác như Momo, ShopeePay, cũng như các giải pháp “Mua trước trả sau” (BNPL) nhằm tối ưu hóa trải nghiệm mua sắm cho người dùng.

    \item \textbf{Xây dựng cộng đồng:} Tích hợp thêm chuyên mục Blog chia sẻ kiến thức làm đẹp và diễn đàn người dùng, nơi khách hàng có thể chia sẻ lộ trình trị mụn thành công nhờ sự hỗ trợ của AI GlowUp, từ đó gia tăng độ uy tín và mức độ gắn kết với thương hiệu.
\end{itemize}
