\section{Kết quả xây dựng hệ thống}

\subsection{Giao diện đăng ký tài khoản}
\begin{figure}[H]
    \centering
    \includegraphics[width=1\textwidth]{graphics/main/chapter3/dk.jpg}
    \caption{Giao diện đăng ký tài khoản}
    \label{fig:register_ui}
\end{figure}




Giao diện đăng ký cho phép người dùng tạo tài khoản mới để sử dụng các chức năng của hệ thống. Giao diện được thiết kế nhằm đảm bảo việc thu thập thông tin người dùng một cách đầy đủ, chính xác và tuân thủ các yêu cầu bảo mật cơ bản.

\textbf{Các thành phần chính:}
\begin{itemize}[noitemsep]
    \item Trường nhập thông tin cá nhân cơ bản của người dùng
    \item Trường nhập Email dùng làm thông tin đăng nhập
    \item Trường nhập mật khẩu
    \item Trường xác nhận lại mật khẩu
    \item Tùy chọn chấp nhận điều khoản sử dụng và chính sách bảo mật
    \item Nút thực hiện đăng ký tài khoản
\end{itemize}

\subsection{Giao diện đăng nhập tài khoản}

\begin{figure}[H]
    \centering
    \includegraphics[width=1\textwidth]{graphics/main/chapter3/dn.jpg}
    \caption{Giao diện đăng nhập tài khoản}
    \label{fig:login_ui}
\end{figure}
\textbf{Các thành phần chính:}
\begin{itemize}[noitemsep]
    \item Trường nhập thông tin định danh người dùng (Email)
    \item Trường nhập mật khẩu với chức năng ẩn/hiện
    \item Tùy chọn ghi nhớ phiên đăng nhập
    \item Chức năng hỗ trợ khôi phục mật khẩu khi người dùng quên thông tin đăng nhập
    \item Nút thực hiện đăng nhập
    \item Tùy chọn đăng nhập thông qua dịch vụ xác thực bên thứ ba
    \item Liên kết điều hướng đến giao diện đăng ký tài khoản mới
\end{itemize}


\subsection{Giao diện trang chủ của khách hàng}
\begin{figure}[H]
    \centering
    \includegraphics[width=1\textwidth]{graphics/main/chapter3/home.jpg}
    \caption{Giao diện trang chủ của khách hàng}
    \label{fig:home_ui}
\end{figure}
Trang chủ là giao diện đầu tiên mà người dùng tiếp cận khi truy cập vào website GlowUp, có vai trò giới thiệu tổng quan hệ thống và định hướng người dùng đến các chức năng chính. Giao diện được thiết kế theo hướng thân thiện với người dùng, hỗ trợ người dùng nhanh chóng tiếp cận thông tin và thực hiện các thao tác mua sắm.

Phần đầu trang cung cấp thông tin thương hiệu và hệ thống điều hướng, cho phép người dùng truy cập nhanh đến các khu vực chức năng quan trọng của hệ thống. Khu vực nội dung chính tập trung giới thiệu các sản phẩm và danh mục nổi bật, hỗ trợ người dùng khám phá và lựa chọn sản phẩm phù hợp với nhu cầu.

Ngoài ra, hệ thống tích hợp công cụ hỗ trợ trực tuyến nhằm hỗ trợ người dùng trong quá trình tìm kiếm thông tin và mua sắm. Thiết kế tổng thể của trang chủ hướng đến việc nâng cao trải nghiệm người dùng và tăng khả năng chuyển đổi từ truy cập sang hành động mua hàng.

\subsection{Giao diện giới thiệu website}
\begin{figure}[H]
    \centering
    \includegraphics[width=1\textwidth]{graphics/main/chapter3/gioithieu.jpg}
    \caption{Giao diện giới thiệu website}
    \label{fig:intro_ui}
\end{figure}
Trang giới thiệu được xây dựng nhằm cung cấp thông tin tổng quan về thương hiệu và định hướng hoạt động của hệ thống, góp phần xây dựng niềm tin đối với người dùng. Nội dung trang tập trung trình bày về mục tiêu phát triển, giá trị cốt lõi và cam kết chất lượng dịch vụ của GlowUp.

Trang giới thiệu cung cấp thông tin về quá trình hình thành và phát triển của hệ thống, năng lực chuyên môn và định hướng phục vụ khách hàng. Thông tin được trình bày một cách trực quan, giúp người dùng dễ dàng nắm bắt các giá trị nổi bật của thương hiệu.

Bên cạnh đó, trang giới thiệu làm rõ các điểm khác biệt và lợi ích khi người dùng lựa chọn mua sắm trên hệ thống, qua đó góp phần nâng cao mức độ tin cậy và thúc đẩy quyết định sử dụng dịch vụ.


\subsection{Giao diện liên hệ}
\begin{figure}[H]
    \centering
    \includegraphics[width=1\textwidth]{graphics/main/chapter3/lienhe.jpg}
    \caption{Giao diện liên hệ}
    \label{fig:contact_ui}
\end{figure}

Trang liên hệ được thiết kế nhằm cung cấp cho người dùng các kênh thông tin để kết nối và trao đổi với đội ngũ hỗ trợ của hệ thống. Giao diện trang bao gồm các thông tin liên lạc cơ bản như địa chỉ email, số điện thoại và địa chỉ văn phòng, giúp người dùng dễ dàng tìm kiếm và sử dụng khi cần hỗ trợ.
\subsection{Giao diện danh mục sản phẩm}
\begin{figure}[H]
    \centering
    \includegraphics[width=1\textwidth]{graphics/main/chapter3/danhmuc.jpg}
    \caption{Giao diện danh mục sản phẩm}
    \label{fig:category_ui}
\end{figure}
Trang danh mục sản phẩm được thiết kế nhằm hỗ trợ người dùng dễ dàng duyệt và lựa chọn sản phẩm theo từng nhóm chức năng. Giao diện hiển thị các danh mục sản phẩm chính của hệ thống, giúp người dùng nhanh chóng tiếp cận đúng nhóm sản phẩm phù hợp với nhu cầu.

Các danh mục được trình bày dưới dạng các khối hiển thị trực quan, cho phép người dùng truy cập vào danh sách sản phẩm chi tiết của từng danh mục. Cách bố trí giao diện đảm bảo tính rõ ràng, dễ sử dụng và tương thích trên nhiều thiết bị, góp phần nâng cao trải nghiệm người dùng trong quá trình tìm kiếm sản phẩm.
\subsection{Giao diện sản phẩm}
\begin{figure}[H]
    \centering
    \includegraphics[width=1\textwidth]{graphics/main/chapter3/sp.jpg}
    \caption{Giao diện sản phẩm }
    \label{fig:product_ui}
\end{figure}
Trang sản phẩm cung cấp danh sách tổng hợp các sản phẩm hiện có trong hệ thống. Người dùng có thể duyệt sản phẩm theo danh mục và sử dụng các chức năng lọc, sắp xếp để thu hẹp phạm vi tìm kiếm theo nhu cầu.

Giao diện hiển thị thông tin cơ bản của từng sản phẩm như hình ảnh, tên và giá bán, giúp người dùng nhanh chóng so sánh và lựa chọn sản phẩm phù hợp.


\subsection{Giao diện sản phẩm chi tiết}
\begin{figure}[H]
    \centering
    \includegraphics[width=1\textwidth]{graphics/main/chapter3/sp_detail.jpg}
    \caption{Giao diện sản phẩm chi tiết}
    \label{fig:product_detail_ui}
\end{figure}
Trang chi tiết sản phẩm cung cấp đầy đủ thông tin về một sản phẩm cụ thể, bao gồm hình ảnh minh họa, thông tin thương hiệu, giá bán, đánh giá người dùng và mô tả chi tiết về thành phần, công dụng, hướng dẫn sử dụng.

Giao diện được thiết kế theo bố cục rõ ràng, hỗ trợ người dùng nắm bắt thông tin nhanh chóng, từ đó đưa ra quyết định mua hàng chính xác.


\subsection{Giao diện đánh giá sản phẩm}
\begin{figure}[H]
    \centering
    \includegraphics[width=1\textwidth]{graphics/main/chapter3/cmt.jpg}
    \caption{Giao diện đánh giá sản phẩm}
    \label{fig:review_ui}
\end{figure}
Chức năng đánh giá sản phẩm cho phép người dùng chia sẻ nhận xét và mức độ hài lòng sau khi sử dụng sản phẩm. Hệ thống hỗ trợ xem tổng quan điểm đánh giá, lọc đánh giá theo mức sao và gửi đánh giá mới.

Phần đánh giá góp phần tăng tính minh bạch của hệ thống, đồng thời hỗ trợ người mua tham khảo ý kiến từ cộng đồng trước khi đưa ra quyết định mua hàng.
\subsection{Giao diện gợi ý sản phẩm}
\begin{figure}[H]
    \centering
    \includegraphics[width=1\textwidth]{graphics/main/chapter3/goi_y_sp.jpg}
    \caption{Giao diện gợi ý sản phẩm}
    \label{fig:suggested_product_ui}
\end{figure}
Trang gợi ý sản phẩm được hiển thị trên trang chi tiết sản phẩm nhằm gợi ý các sản phẩm liên quan hoặc thay thế. Chức năng này giúp mở rộng lựa chọn cho người dùng và hỗ trợ tăng khả năng mua thêm sản phẩm.

Việc hiển thị các sản phẩm liên quan góp phần nâng cao trải nghiệm người dùng và hiệu quả kinh doanh của hệ thống thông qua cơ chế gợi ý.

\subsection{Giao diện giỏ hàng}
\begin{figure}[H]
    \centering
    \includegraphics[width=1\textwidth]{graphics/main/chapter3/gio_hang.jpg}
    \caption{Giao diện giỏ hàng}
    \label{fig:cart_ui}
\end{figure}
Trang giỏ hàng cho phép người dùng xem lại các sản phẩm đã thêm vào trước khi tiến hành thanh toán. Giao diện được thiết kế trực quan với hai khu vực chính gồm danh sách sản phẩm và phần tóm tắt đơn hàng.

Danh sách sản phẩm hiển thị thông tin cơ bản của từng sản phẩm như hình ảnh, tên sản phẩm, giá và số lượng. Người dùng có thể điều chỉnh số lượng sản phẩm, chọn hoặc bỏ chọn sản phẩm cần mua, cũng như xóa sản phẩm khỏi giỏ hàng. Chức năng chọn tất cả giúp người dùng thao tác nhanh khi giỏ hàng có nhiều sản phẩm.

Phần tóm tắt đơn hàng hiển thị các thông tin tổng hợp như số lượng sản phẩm đã chọn, tạm tính, phí vận chuyển và tổng tiền thanh toán. Nút chuyển sang bước thanh toán được đặt nổi bật nhằm hỗ trợ người dùng tiếp tục quy trình mua sắm một cách thuận tiện.

\subsection{Giao diện thanh toán}
\begin{figure}[H]
    \centering
    \includegraphics[width=1\textwidth]{graphics/main/chapter3/donhang.jpg}
    \caption{Giao diện thanh toán}
    \label{fig:order_ui}
\end{figure}
Trang thanh toán cho phép người dùng xác nhận thông tin giao hàng và lựa chọn phương thức thanh toán trước khi hoàn tất đơn hàng. Giao diện được bố trí khoa học, giúp người dùng dễ dàng kiểm tra lại thông tin cá nhân, địa chỉ nhận hàng và phương thức thanh toán đã chọn.

Hệ thống hỗ trợ nhiều hình thức thanh toán khác nhau như thanh toán khi nhận hàng (COD) và thanh toán trực tuyến qua cổng trung gian. Ngoài ra, người dùng có thể áp dụng mã giảm giá (nếu có) để giảm giá trị đơn hàng. Thông tin tóm tắt đơn hàng được hiển thị rõ ràng để người dùng kiểm tra lại tổng số tiền cần thanh toán trước khi xác nhận đặt hàng.

\subsection{Giao diện quản lý đơn hàng}
\begin{figure}[H]
    \centering
    \includegraphics[width=1\textwidth]{graphics/main/chapter3/lichsu.jpg}
    \caption{Giao diện quản lý đơn hàng}
    \label{fig:order_history_ui}
\end{figure}
Trang quản lý đơn hàng cho phép người dùng theo dõi danh sách các đơn hàng đã đặt và trạng thái xử lý của từng đơn. Giao diện hỗ trợ lọc đơn hàng theo trạng thái nhằm giúp người dùng dễ dàng tra cứu và quản lý lịch sử mua sắm.

Mỗi đơn hàng được hiển thị kèm theo các thông tin cơ bản như mã đơn hàng, thời gian đặt hàng, trạng thái xử lý, phương thức thanh toán và tổng giá trị đơn hàng. Tại mỗi đơn hàng, người dùng có thể thực hiện một số thao tác phù hợp với trạng thái hiện tại của đơn, chẳng hạn như hủy đơn khi đơn chưa được xử lý hoặc tiếp tục thanh toán đối với các đơn chưa hoàn tất giao dịch.
\subsection{Giao diện chuyển giao diện sang vnpay để thanh toán}
\begin{figure}[H]
    \centering
    \includegraphics[width=1\textwidth]{graphics/main/chapter3/vnpay.jpg}
    \caption{Giao diện chuyển giao diện sang vnpay để thanh toán}
    \label{fig:vnpay_ui}
\end{figure}
\begin{figure}[H]
    \centering
    \includegraphics[width=1\textwidth]{graphics/main/chapter3/tinh_vnpay.jpg}
    \caption{Giao diện nhập tài khoản thanh toán}
    \label{fig:payment_account_ui}
\end{figure}
Khi người dùng lựa chọn thanh toán trực tuyến qua VNPay, hệ thống sẽ chuyển hướng đến cổng thanh toán của VNPay. Giao diện thanh toán được thiết kế theo tiêu chuẩn bảo mật, cung cấp nhiều phương thức thanh toán khác nhau như thẻ nội địa, thẻ quốc tế và các hình thức thanh toán điện tử.

Trang thanh toán hiển thị đầy đủ thông tin đơn hàng để người dùng kiểm tra trước khi xác nhận giao dịch. Quá trình thanh toán được thực hiện thông qua các bước xác thực nhằm đảm bảo tính an toàn và bảo mật thông tin cho người dùng.





\subsection{Giao diện AI chatbot chăm sóc khách hàng}
\begin{figure}[H]
    \centering
    \includegraphics[width=1\textwidth]{graphics/main/chapter3/AI_chatbot.jpg}
    \caption{Giao diện AI chatbot chăm sóc khách hàng}
    \label{fig:chatbot_ui}
\end{figure}
Hệ thống tích hợp chatbot AI nhằm hỗ trợ tư vấn sản phẩm và giải đáp thắc mắc cho người dùng trong quá trình mua sắm. Chatbot được hiển thị dưới dạng cửa sổ trò chuyện trên giao diện website, cho phép người dùng đặt câu hỏi và nhận phản hồi theo thời gian thực.

Dựa trên nội dung tương tác và nhu cầu của người dùng, chatbot có thể gợi ý các sản phẩm phù hợp, cung cấp thông tin cơ bản về sản phẩm và hỗ trợ người dùng truy cập nhanh đến trang chi tiết sản phẩm. Chức năng chatbot góp phần nâng cao trải nghiệm người dùng và hỗ trợ quá trình ra quyết định mua hàng một cách hiệu quả.

\subsection{Giao diện AI phân tích da mặt}
\begin{figure}[H]
    \centering
    \includegraphics[width=1\textwidth]{graphics/main/chapter3/AI_acne.jpg}
    \caption{Giao diện AI phân tích da mặt}
    \label{fig:skin_analysis_ui}
\end{figure}
Giao diện phân tích da mặt sử dụng trí tuệ nhân tạo nhằm giúp người dùng nhận biết tình trạng da và mức độ mụn thông qua hình ảnh khuôn mặt được tải lên hệ thống. Giao diện được thiết kế đơn giản, thân thiện với người dùng, bao gồm các thành phần chính như sau:
\subsection{Giao diện kết quả phân tích da mặt}
\begin{figure}[H]
    \centering
    \includegraphics[width=1\textwidth]{graphics/main/chapter3/ketqua_acne.jpg}
    \caption{Giao diện kết quả phân tích da mặt}
    \label{fig:analysis_result_ui}
\end{figure}
Giao diện kết quả phân tích da mặt hiển thị chi tiết tình trạng da và mức độ mụn sau khi người dùng tải ảnh khuôn mặt lên hệ thống. Kết quả được trình bày rõ ràng, bao gồm các thông tin như số lượng mụn trên từng vùng da (trán, má, mũi, cằm), mức độ mụn tổng thể và lời khuyên chăm sóc da phù hợp.






\subsection{Giao diện dasboard admin}
\begin{figure}[H]
    \centering
    \includegraphics[width=1\textwidth]{graphics/main/chapter3/admin_dasboard.jpg}
    \caption{Giao diện dashboard admin}
    \label{fig:admin_dashboard_ui}
\end{figure}
Giao diện Dashboard dành cho quản trị viên cung cấp cái nhìn tổng thể về tình hình hoạt động của hệ thống thông qua các chỉ số thống kê trực quan. Tại đây, quản trị viên có thể theo dõi nhanh các thông số quan trọng như tổng doanh thu, số lượng đơn hàng mới, số lượng khách hàng đăng ký và các biểu đồ tăng trưởng theo thời gian.

Các biểu đồ được thiết kế sinh động, hỗ trợ việc phân tích dữ liệu và giúp người quản lý đưa ra các quyết định kinh doanh kịp thời. Ngoài ra, Dashboard còn hiển thị danh sách các hoạt động gần đây và trạng thái của các tiến trình quan trọng trong hệ thống.



\subsection{Giao diện quản lý sản phẩm của admin}
\begin{figure}[H]
    \centering
    \includegraphics[width=1\textwidth]{graphics/main/chapter3/admin_sp.jpg}
    \caption{Giao diện quản lý sản phẩm của admin}
    \label{fig:admin_product_ui}
\end{figure}
Giao diện quản lý sản phẩm cho phép quản trị viên thực hiện các thao tác quản lý danh sách sản phẩm hiện có trên hệ thống. Chức năng chính bao gồm thêm mới sản phẩm, chỉnh sửa thông tin sản phẩm (tên, giá, mô tả, hình ảnh, tồn kho) và xóa hoặc ẩn các sản phẩm không còn kinh doanh.

Hệ thống hỗ trợ tìm kiếm sản phẩm theo tên và lọc theo danh mục, giúp việc quản lý kho hàng trở nên thuận tiện và chính xác. Quản trị viên có thể cập nhật nhanh trạng thái sản phẩm để đảm bảo thông tin hiển thị cho khách hàng luôn là mới nhất.



\subsection{Giao diện quản lý đơn hàng của admin}
\begin{figure}[H]
    \centering
    \includegraphics[width=1\textwidth]{graphics/main/chapter3/admin_order.jpg}
    \caption{Giao diện quản lý đơn hàng của admin}
    \label{fig:admin_order_ui}
\end{figure}
Giao diện quản lý đơn hàng là nơi quản trị viên theo dõi và xử lý toàn bộ các giao dịch phát sinh trên hệ thống. Danh sách đơn hàng được hiển thị kèm theo thông tin chi tiết về khách hàng, tổng tiền, phương thức thanh toán và trạng thái xử lý hiện tại (chờ xác nhận, đang giao, đã hoàn thành, đã hủy).

Quản trị viên có thể xác nhận đơn hàng, cập nhật trạng thái vận chuyển và xem chi tiết từng đơn hàng để đảm bảo quy trình phục vụ khách hàng được diễn ra trôi chảy. Công cụ lọc theo trạng thái và thời gian hỗ trợ việc tra cứu và thống kê đơn hàng một cách hiệu quả.






\subsection{Giao diện quản lý danh mục sản phẩm}
\begin{figure}[H]
    \centering
    \includegraphics[width=1\textwidth]{graphics/main/chapter3/admin_danhmuc.jpg}
    \caption{Giao diện quản lý danh mục sản phẩm}
    \label{fig:admin_category_ui}
\end{figure}
Giao diện này cho phép quản trị viên quản lý cấu trúc cây danh mục của website. Các chức năng bao gồm tạo mới danh mục, cập nhật tên, mô tả và hình ảnh minh họa cho danh mục. Việc tổ chức danh mục khoa học giúp khách hàng dễ dàng tìm kiếm và tiếp cận sản phẩm, đồng thời tối ưu hóa giao diện hiển thị cho hệ thống.




\subsection{Giao diện quản lý đánh giá sản phẩm của admin}
\begin{figure}[H]
    \centering
    \includegraphics[width=1\textwidth]{graphics/main/chapter3/admin_cmt.jpg}
    \caption{Giao diện quản lý đánh giá sản phẩm của admin}
    \label{fig:admin_review_ui}
\end{figure}
Trang quản lý đánh giá giúp quản trị viên theo dõi các phản hồi từ khách hàng về sản phẩm. Quản trị viên có thể xem nội dung nhận xét, số sao đánh giá và phản hồi lại các ý kiến của khách hàng. Chức năng này cũng hỗ trợ việc kiểm duyệt các nội dung không phù hợp nhằm duy trì môi trường mua sắm văn minh và tin cậy.




\subsection{Giao diện quản lý vận chuyển của admin}
\begin{figure}[H]
    \centering
    \includegraphics[width=1\textwidth]{graphics/main/chapter3/admin_vanchuyen.jpg}
    \caption{Giao diện quản lý vận chuyển của admin}
    \label{fig:admin_shipping_ui}
\end{figure}
Giao diện quản lý vận chuyển cho phép quản trị viên thiết lập các đơn vị vận chuyển và quản lý trạng thái luân chuyển của các gói hàng. Hệ thống hiển thị danh sách các phiếu vận chuyển kèm theo mã vận đơn và trạng thái tương ứng, giúp người quản lý kiểm soát chặt chẽ quá trình giao hàng đến tay người tiêu dùng.


\subsection{Giao diện chi tiết vận chuyển của admin}
\begin{figure}[H]
    \centering
    \includegraphics[width=1\textwidth]{graphics/main/chapter3/admin_detail_vanchuyen.jpg}
    \caption{Giao diện chi tiết vận chuyển của admin}
    \label{fig:admin_shipping_detail_ui}
\end{figure}
Trang chi tiết vận chuyển cung cấp thông tin cụ thể về lộ trình và các mốc thời gian xử lý của một đơn hàng nhất định. Quản trị viên có thể theo dõi vị trí hiện tại của đơn hàng, các ghi chú từ đơn vị vận chuyển và thời gian dự kiến giao hàng thành công, đảm bảo thông tin luôn được cập nhật kịp thời cho khách hàng.


\subsection{Giao diện quản lý khuyến mãi của admin}
\begin{figure}[H]
    \centering
    \includegraphics[width=1\textwidth]{graphics/main/chapter3/admin_voucher.jpg}
    \caption{Giao diện quản lý khuyến mãi của admin}
    \label{fig:admin_promotion_ui}
\end{figure}
Giao diện quản lý khuyến mãi cho phép quản trị viên tạo và quản lý các mã giảm giá (vouchers), các chương trình ưu đãi đặc biệt. Quản trị viên có thể thiết lập giá trị giảm giá, thời hạn áp dụng, số lượng mã phát hành và các điều kiện đi kèm. Đây là công cụ quan trọng trong việc thúc đẩy doanh số và thu hút khách hàng quay lại mua sắm.


\subsection{Giao diện quản lý cửa hàng }
\begin{figure}[H]
    \centering
    \includegraphics[width=1\textwidth]{graphics/main/chapter3/admin_cuahang.jpg}
    \caption{Giao diện quản lý cửa hàng}
    \label{fig:admin_store_ui}
\end{figure}
Trang quản lý cửa hàng hỗ trợ quản trị viên quản lý thông tin về các địa điểm hoặc chi nhánh cửa hàng vật lý. Các thông tin như tên cửa hàng, địa chỉ, số điện thoại liên hệ và tọa độ vị trí được thiết lập tại đây, giúp hệ thống đồng bộ dữ liệu và hỗ trợ khách hàng tìm kiếm cửa hàng gần nhất một cách dễ dàng.

