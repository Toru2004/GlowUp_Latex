\section{Dữ liệu thực tế }
Năm 2023, doanh thu toàn cầu của ngành mỹ phẩm và chăm sóc cá nhân ước đạt khoảng 625,7 tỷ USD. Thị trường này đang tăng trưởng đều đặn với tốc độ $\sim$3,3\%/năm giai đoạn 2023--2028. Một số báo cáo khác cho rằng quy mô thị trường năm 2023 dao động từ 617,2 đến 626 tỷ USD và dự kiến tăng 9\% lên khoảng 670,8 tỷ USD vào năm 2024.
	Tại Việt Nam, theo Statista và các khảo sát quốc tế, thị trường mỹ phẩm và chăm sóc cá nhân năm 2023 đạt khoảng 2,4 tỷ USD và dự kiến tăng lên khoảng 2,74 tỷ USD vào năm 2025. Tốc độ tăng trưởng trung bình khoảng 3,2--3,3\%/năm trong giai đoạn 2023--2027. Trong đó, phân khúc chăm sóc cá nhân chiếm thị phần lớn nhất, ước khoảng 1,20 tỷ USD vào năm 2025, trong khi các sản phẩm chăm sóc da và trang điểm tăng trưởng nhanh nhờ sự quan tâm ngày càng cao của người tiêu dùng trẻ.
	
\subsection*{Hành vi người tiêu dùng và xu hướng mua sắm trực tuyến}
	
	Trên toàn cầu, đến năm 2024, khoảng 26\% doanh thu ngành mỹ phẩm đến từ kênh trực tuyến và tỷ lệ này được dự báo sẽ tiếp tục tăng. Riêng tại Mỹ, thương mại điện tử chiếm khoảng 41\% doanh thu mỹ phẩm. Người tiêu dùng, đặc biệt là thế hệ Gen Z, có xu hướng tìm hiểu kỹ sản phẩm thông qua nền tảng số và chịu ảnh hưởng mạnh từ mạng xã hội.
	Tại Việt Nam, hơn 60\% phụ nữ sử dụng sản phẩm chăm sóc da hằng ngày. Các sản phẩm phổ biến bao gồm sữa rửa mặt (49\%), nước hoa (41\%), kem chống nắng (31\%) và kem dưỡng (25\%). Thị trường chủ yếu do các thương hiệu nước ngoài chiếm lĩnh với hơn 90\% thị phần, trong đó thương hiệu Hàn Quốc dẫn đầu.

\subsection*{Báo cáo doanh thu thị trường Việt Nam trong tương lai}
	Kênh mua sắm trực tuyến tại Việt Nam tăng trưởng mạnh với mức tăng 47,6\% năm 2022 và 52,2\% năm 2023, đạt khoảng 37,7 nghìn tỷ đồng (xấp xỉ 1,5 tỷ USD). Đến năm 2023, doanh thu online chiếm khoảng 19\% tổng thị trường mỹ phẩm, tăng mạnh so với mức 8\% năm 2018.
	Sự phát triển của các sàn thương mại điện tử như Shopee, Lazada và TikTok Shop cùng xu hướng livestream bán hàng đã thay đổi đáng kể thói quen mua sắm của người tiêu dùng, đặc biệt là giới trẻ.
