\subsection*{ Phương pháp nghiên cứu}

Để thực hiện đề tài ``Xây dựng website bán mỹ phẩm kết hợp AI hỗ trợ tư vấn da mặt trị mụnvà chatbot gợi ý sản phẩm'', quá trình nghiên cứu được triển khai dựa trên sự kết hợp giữa các phương pháp nghiên cứu lý thuyết và thực tiễn, nhằm đảm bảo tính khoa học, khả thi và phù hợp với yêu cầu thực tế.

Phương pháp nghiên cứu tài liệu được sử dụng để thu thập và tổng hợp các kiến thức liên quan đến thương mại điện tử, thiết kế website bán hàng, trải nghiệm người dùng (UI/UX) cũng như các công nghệ phát triển web hiện đại. Các tài liệu tham khảo bao gồm sách chuyên ngành, giáo trình, bài báo khoa học, website chuyên môn và các nguồn thông tin uy tín trên Internet. Việc nghiên cứu tài liệu giúp xây dựng nền tảng lý thuyết vững chắc cho đề tài.

Bên cạnh đó, đề tài áp dụng phương pháp phân tích và khảo sát thực tế thông qua việc tìm hiểu, đánh giá một số website bán mỹ phẩm phổ biến trong và ngoài nước. Từ đó, tiến hành phân tích ưu điểm, hạn chế về giao diện, chức năng, quy trình mua hàng và quản lý hệ thống, làm cơ sở đề xuất giải pháp thiết kế phù hợp cho website của đề tài.

Ngoài ra, phương pháp phân tích yêu cầu hệ thống được sử dụng để xác định rõ các chức năng cần thiết đối với cả người dùng và người quản trị. Trên cơ sở đó, tiến hành thiết kế cấu trúc website, xây dựng giao diện và phát triển các chức năng chính theo mô hình hệ thống bán hàng trực tuyến.

Cuối cùng, phương pháp thực nghiệm và kiểm thử được áp dụng trong quá trình triển khai website. Hệ thống sau khi xây dựng sẽ được kiểm tra về tính ổn định, khả năng hoạt động, tính thân thiện với người dùng và khả năng đáp ứng các yêu cầu đề ra. Kết quả kiểm thử là cơ sở để đánh giá mức độ hoàn thiện của website và rút ra các nhận xét, hướng phát triển trong tương lai.
