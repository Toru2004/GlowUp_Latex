\section{Sơ đồ hoạt động (Activity Diagram)}




\subsection{Sơ đồ hoạt động xử lý thanh toán}
\begin{figure}[H]
    \centering
    \includegraphics[width=0.8\textwidth]{graphics/main/chapter2/section2/activity_vnpay.jpg}
    \caption{Sơ đồ hoạt động thanh toán qua VNPAY}
    \label{fig:activity_vnpay}
\end{figure}

\textbf{Mục đích:}  
Chức năng thanh toán và quản lý đơn hàng nhằm hỗ trợ người dùng hoàn tất quá trình mua hàng trên hệ thống một cách thuận tiện và an toàn. Hệ thống cung cấp hai hình thức thanh toán là thanh toán khi nhận hàng (COD) và thanh toán trực tuyến qua cổng VNPay, giúp người dùng linh hoạt lựa chọn phương thức phù hợp. Đồng thời, chức năng này cho phép người dùng theo dõi lịch sử đơn hàng và gửi yêu cầu hủy đơn khi cần thiết, hỗ trợ Admin quản lý và xử lý trạng thái đơn hàng một cách hiệu quả.

\textbf{Luồng hoạt động:}  
Người dùng bắt đầu bằng việc lựa chọn phương thức thanh toán cho đơn hàng.  

\begin{itemize}
    \item Trường hợp người dùng chọn thanh toán khi nhận hàng (COD), hệ thống tiến hành tạo đơn hàng và hiển thị kết quả thanh toán cho người dùng.
    \item Trường hợp người dùng chọn thanh toán qua VNPay, hệ thống chuyển hướng người dùng đến cổng thanh toán VNPay để thực hiện giao dịch. Sau khi quá trình thanh toán được xử lý:
    \begin{itemize}
        \item Nếu giao dịch thành công, hệ thống cập nhật trạng thái đơn hàng và hiển thị kết quả thanh toán thành công.
        \item Nếu giao dịch thất bại, hệ thống thông báo kết quả thất bại để người dùng có thể thực hiện lại hoặc lựa chọn phương thức thanh toán khác.
    \end{itemize}
\end{itemize}

Sau khi hoàn tất thanh toán, người dùng có thể xem lịch sử đơn hàng. Trong trường hợp người dùng có nhu cầu hủy đơn, hệ thống tiếp nhận yêu cầu hủy và chuyển cho Admin xử lý. Sau khi Admin xác nhận, hệ thống cập nhật lại trạng thái đơn hàng tương ứng và thông báo kết quả cho người dùng.

\subsection{Sơ đồ hoạt động phân tích da mặt}
\begin{figure}[H]
    \centering
    \includegraphics[width=0.6\textwidth]{graphics/main/chapter2/section2/activity_acne.jpg}
    \caption{Sơ đồ hoạt động phân tích da mặt}
    \label{fig:activity_acne}
\end{figure}


\textbf{Mục đích:}  
Chức năng phân tích da mặt bằng AI nhằm hỗ trợ người dùng nhận biết tình trạng da và mức độ mụn thông qua hình ảnh khuôn mặt được tải lên hệ thống. Hệ thống ứng dụng công nghệ MediaPipe để nhận diện khuôn mặt và các điểm đặc trưng, kết hợp mô hình CNN để phân tích tình trạng mụn trên từng vùng da. Dựa trên kết quả phân tích, hệ thống đưa ra lời khuyên chăm sóc da và gợi ý sản phẩm phù hợp, giúp người dùng lựa chọn sản phẩm hiệu quả và cá nhân hóa trải nghiệm mua sắm.

\textbf{Luồng hoạt động:}  
Người dùng thực hiện tải ảnh khuôn mặt lên hệ thống. Hệ thống sử dụng MediaPipe để nhận diện khuôn mặt và xác định các vùng da chính gồm trán, má trái/phải, mũi và cằm. Sau đó, từng vùng da được cắt ra và đưa vào mô hình CNN để phân tích tình trạng mụn.  

Kết quả phân tích (bao gồm số lượng, vị trí và mức độ mụn) được hệ thống tổng hợp và hiển thị cho người dùng. Dựa trên kết quả này, hệ thống đồng thời thực hiện hai chức năng: đưa ra lời khuyên chăm sóc da phù hợp và gợi ý các sản phẩm tương ứng với tình trạng da của người dùng.
%Sơ đồ hoạt động admin đánh giá sản phẩm
\subsection{Sơ đồ hoạt động admin quản lý đánh giá}

\begin{figure}[H]
    \centering
    \includegraphics[width=0.6\textwidth]{graphics/main/chapter2/section2/acctivityadmindanhgia.png}
    \caption{Sơ đồ hoạt động ADMIN quản lý đánh giá}
    \label{fig:activity_dg}
\end{figure}

%Mô tả sơ đồ hoạt động đánh giá sản phẩm

\textbf{Mục đích:}

Chức năng quản lý đánh giá cho phép Admin theo dõi, kiểm soát và xử lý các đánh giá sản phẩm do khách hàng gửi lên hệ thống. Thông qua chức năng này, Admin có thể xem danh sách đánh giá, tìm kiếm hoặc lọc đánh giá theo tiêu chí nhất định, xem chi tiết nội dung đánh giá và thực hiện các thao tác như ẩn hoặc xóa những đánh giá vi phạm quy định.

Mục đích chính của chức năng là đảm bảo nội dung hiển thị trên website phù hợp, duy trì tính minh bạch, chuyên nghiệp và nâng cao chất lượng trải nghiệm người dùng.

\textbf{Luồng hoạt động:}

Admin bắt đầu bằng việc truy cập vào chức năng quản lý đánh giá trong hệ thống. Hệ thống hiển thị danh sách các đánh giá sản phẩm hiện có.

\begin{itemize}
    \item Trường hợp Admin thực hiện lọc hoặc tìm kiếm đánh giá, hệ thống cập nhật và hiển thị danh sách đánh giá theo tiêu chí đã chọn.
    
    \item Khi Admin chọn một đánh giá cụ thể, hệ thống hiển thị nội dung chi tiết của đánh giá đó.
\end{itemize}

Sau khi xem nội dung đánh giá, Admin có thể lựa chọn hành động xử lý:

\begin{itemize}
    \item Nếu Admin chọn ẩn đánh giá, hệ thống hiển thị thông báo xác nhận.
    \begin{itemize}
        \item Nếu Admin đồng ý, hệ thống cập nhật trạng thái đánh giá sang ``Ẩn'' và làm mới danh sách.
        \item Nếu Admin hủy, hệ thống không thực hiện thay đổi và quay lại màn hình trước đó.
    \end{itemize}
    
    \item Nếu Admin chọn xóa đánh giá, hệ thống hiển thị thông báo xác nhận.
    \begin{itemize}
        \item Nếu Admin đồng ý, hệ thống xóa đánh giá khỏi cơ sở dữ liệu và cập nhật lại danh sách.
        \item Nếu Admin hủy, hệ thống giữ nguyên dữ liệu và quay lại danh sách đánh giá.
    \end{itemize}
\end{itemize}

Sau khi hoàn tất thao tác, hệ thống hiển thị lại danh sách đánh giá đã được cập nhật và quy trình kết thúc.
%Sơ đồ hoạt động khách hàng đánh giá sản phẩm
\subsection{Sơ đồ hoạt động khách hàng đánh giá sản phẩm}

\begin{figure}[H]
    \centering
    \includegraphics[width=0.6\textwidth]{graphics/main/chapter2/section2/acctivitykhachhangdanhgia.png}
    \caption{Sơ đồ hoạt động đánh giá sản phẩm}
    \label{fig:activity_dg}
\end{figure}
 %Mô tả hoạt động khách hàng đánh giá sản phẩm

\textbf{Mục đích:}

Chức năng đánh giá sản phẩm cho phép khách hàng gửi nhận xét và phản hồi về sản phẩm đã mua. Thông qua chức năng này, khách hàng có thể chia sẻ trải nghiệm sử dụng sản phẩm, góp phần cung cấp thông tin tham khảo cho những người mua sau và giúp hệ thống nâng cao chất lượng dịch vụ.

\textbf{Luồng hoạt động:}

Khách hàng bắt đầu bằng việc lựa chọn sản phẩm đã mua để thực hiện đánh giá.

\begin{itemize}
    \item Sau khi chọn sản phẩm hợp lệ (đã mua), hệ thống hiển thị biểu mẫu đánh giá.
    
    \item Khách hàng nhập nội dung đánh giá vào biểu mẫu.
    
    \item Hệ thống tiến hành kiểm tra tính hợp lệ của nội dung đánh giá.
\end{itemize}

\begin{itemize}
    \item Nếu nội dung không hợp lệ (thiếu thông tin, vi phạm quy định, vượt quá giới hạn ký tự,...), hệ thống hiển thị thông báo lỗi và yêu cầu khách hàng nhập lại nội dung.
    
    \item Nếu nội dung hợp lệ, hệ thống lưu đánh giá vào cơ sở dữ liệu và hiển thị thông báo hoàn thành.
\end{itemize}

Sau khi nhận được thông báo hoàn thành, khách hàng có thể:

\begin{itemize}
    \item Chọn ``Đồng ý'' để kết thúc quy trình đánh giá.
    
    \item Chọn ``Hủy'' để thoát khỏi chức năng mà không thực hiện thêm thao tác nào.
\end{itemize}

Quy trình kết thúc khi khách hàng xác nhận hoàn tất hoặc thoát khỏi chức năng.
