\section{Xây dựng Use Case Diagram}

Sơ đồ Use Case được sử dụng nhằm mô tả các chức năng của hệ thống website thương mại điện tử
thông qua mối quan hệ tương tác giữa hệ thống và các tác nhân bên ngoài.
Sơ đồ giúp làm rõ phạm vi hệ thống, các chức năng nghiệp vụ chính cũng như vai trò của từng đối tượng sử dụng.

\subsection{Các tác nhân trong hệ thống}

Hệ thống website thương mại điện tử có hai tác nhân chính tham gia vào quá trình sử dụng hệ thống, bao gồm:
\begin{itemize}
    \item \textbf{Người mua (Customer)}: Là người sử dụng hệ thống để tìm kiếm, lựa chọn sản phẩm, đặt hàng,
    thanh toán, theo dõi đơn hàng và đánh giá sản phẩm.
    \item \textbf{Người bán (Admin/Người quản trị)}: Là người quản lý toàn bộ hoạt động của hệ thống,
    bao gồm quản lý sản phẩm, đơn hàng, người dùng, xử lý giao hàng và theo dõi doanh thu.
\end{itemize}

\subsection{Các nhóm Use Case chính}

Dựa trên yêu cầu chức năng của hệ thống, các Use Case được phân thành các nhóm chính như sau:

\subsubsection{Nhóm Use Case dành cho người mua}

Nhóm Use Case dành cho người mua bao gồm các chức năng hỗ trợ hoạt động mua sắm trực tuyến,
nhằm mang lại trải nghiệm thuận tiện và dễ sử dụng cho khách hàng.

Các Use Case chính bao gồm:
\begin{itemize}
    \item Đăng ký, đăng nhập hệ thống
    \item Duyệt và tìm kiếm sản phẩm
    \item Xem chi tiết sản phẩm
    \item Quản lý giỏ hàng
    \item Đặt hàng và thanh toán
    \item Theo dõi trạng thái đơn hàng
    \item Đánh giá và nhận xét sản phẩm
    \item Nhận tư vấn từ chatbot AI và AI gợi ý sản phẩm phù hợp
\end{itemize}

\subsubsection{Nhóm Use Case dành cho người bán}

Nhóm Use Case dành cho người bán tập trung vào các chức năng quản trị và vận hành hệ thống,
đảm bảo quá trình bán hàng và chăm sóc khách hàng diễn ra hiệu quả.

Các Use Case chính bao gồm:
\begin{itemize}
    \item Quản lý tài khoản người dùng
    \item Quản lý danh mục và sản phẩm
    \item Quản lý đơn hàng
    \item Cập nhật trạng thái giao hàng
    \item Quản lý phương thức thanh toán
    \item Theo dõi thống kê doanh thu và báo cáo tổng hợp
    \item Quản lý đánh giá và phản hồi từ khách hàng
\end{itemize}

\subsection{Sơ đồ Use Case tổng quát của hệ thống}

Sơ đồ Use Case tổng quát thể hiện mối quan hệ giữa các tác nhân và các chức năng chính của hệ thống website thương mại điện tử.
Thông qua sơ đồ, có thể thấy rõ phạm vi chức năng của hệ thống cũng như trách nhiệm của từng tác nhân.

\begin{figure}[H]
    \centering
    \includegraphics[width=0.95\textwidth]{graphics/front/tmdt.drawio.png}
    \caption{Sơ đồ Use Case tổng quát của website thương mại điện tử}
    \label{fig:usecase}
\end{figure}

\subsection{Nhận xét}

Từ sơ đồ Use Case, có thể nhận thấy hệ thống đáp ứng đầy đủ các yêu cầu chức năng đã được đề cập trong phần
đặc tả yêu cầu hệ thống. Việc phân chia rõ ràng các Use Case theo từng tác nhân giúp hệ thống
dễ dàng mở rộng, bảo trì và phát triển trong tương lai.
