\section{Xây dựng Use Case Diagram}
\subsection{Use case Diagram tổng quát}
\begin{figure}[H]
    \centering
    \includegraphics[scale=0.3]{graphics/main/chapter2/section2/usecasetq.jpg}
    \caption{Use case Diagram tổng quát}
    \label{fig:usecase_overall}
\end{figure}
Các Actor trong Hệ thống bao gồm:
\begin{itemize}[noitemsep]
    \item Customer: Khách hàng sử dụng website để xem và mua sản phẩm mỹ phẩm, quản lý giỏ hàng, đặt hàng, thanh toán, đánh giá sản phẩm và nhận tư vấn từ AI chatbot cùng phân tích khuôn mặt để gợi ý sản phẩm phù hợp.

    \item Admin: Quản trị viên hệ thống sử dụng web admin panel để quản lý toàn bộ hệ thống bao gồm danh mục, sản phẩm, đơn hàng, người dùng, vận chuyển, đánh giá, khuyến mãi và thông tin cửa hàng.

    \item UserBase: Người dùng cơ bản thực hiện các chức năng xác thực như đăng ký tài khoản mới, đăng nhập vào hệ thống, đăng xuất và quản lý thông tin tài khoản cá nhân.
\end{itemize}
%----------------------------------------------------------------Đăng nhập đăng ký---------------
\subsection{Use case Diagram chi tiết về xử lý đăng nhập đăng ký}
\begin{figure}[H]
    \centering
    \includegraphics[width=0.8\textwidth]{graphics/main/chapter2/section2/usecase_authen.jpg}
    \caption{Use case Diagram người dùng đăng nhập đăng ký}
    \label{fig:usecase_user}
\end{figure}
Dưới đây là các đặc tả use case của người dùng đăng nhập đăng ký.
\begin{table}[H]
    \centering
    \caption{Đặc tả Use Case UC-01: Đăng ký tài khoản}
    \label{tab:uc01}
\begin{tabular}{|p{4cm}|p{11cm}|}
    \hline
    \textbf{Actor} & Customer, Barber Owner \\
    \hline
\textbf{Mô tả} & Người dùng tạo tài khoản mới \\
\hline
\textbf{Tiền điều kiện} & Chưa có tài khoản \\
\hline
\textbf{Luồng chính} & 
1. Nhập thông tin (email, mật khẩu, họ tên) \newline
2. Hệ thống xác thực thông tin \newline
3. Gửi xác thực qua email \newline
4. Click vào link email nhận được \newline
5. Tạo tài khoản thành công \\
\hline
\textbf{Luồng phụ} & 
2a. Thông tin không hợp lệ $\rightarrow$ Hiển thị lỗi \newline
4a. Link sai/hết hạn $\rightarrow$ Gửi lại email \\
\hline
\textbf{Hậu điều kiện} & Tài khoản được tạo và kích hoạt \\
\hline
\end{tabular}
\end{table}



\begin{table}[H]
\centering
\caption{Đặc tả Use Case UC-02: Đăng nhập}
\label{tab:uc02}
\begin{tabular}{|p{4cm}|p{11cm}|}
\hline
\textbf{Actor} & Customer, Barber Owner, Admin \\
\hline
\textbf{Mô tả} & Đăng nhập vào hệ thống \\
\hline
\textbf{Tiền điều kiện} & Có tài khoản hợp lệ \\
\hline
\textbf{Luồng chính} & 
1. Nhập email và mật khẩu \newline
2. Hệ thống xác thực \newline
3. Cấp token truy cập \newline
4. Chuyển đến màn hình chính \\
\hline
\textbf{Luồng phụ} & 
2a. Sai thông tin đăng nhập $\rightarrow$ Thông báo lỗi \newline
2b. Tài khoản bị khóa $\rightarrow$ Thông báo liên hệ admin \\
\hline
\textbf{Hậu điều kiện} & Người dùng truy cập được hệ thống \\
\hline
\end{tabular}
\end{table}



\begin{table}[H]
\centering
\caption{Đặc tả Use Case UC-03: Đăng xuất}
\label{tab:uc03}
\begin{tabular}{|p{4cm}|p{11cm}|}
\hline
\textbf{Actor} & Customer, Barber Owner, Admin \\
\hline
\textbf{Mô tả} & Thoát khỏi hệ thống \\
\hline
\textbf{Luồng chính} & 
1. Chọn đăng xuất \newline
2. Xóa token và session \newline
3. Chuyển về màn hình đăng nhập \\
\hline
\textbf{Hậu điều kiện} & Phiên làm việc kết thúc \\
\hline
\end{tabular}
\end{table}



\begin{table}[H]
\centering
\caption{Đặc tả Use Case UC-04: Quản lý hồ sơ cá nhân}
\label{tab:uc04}
\begin{tabular}{|p{4cm}|p{11cm}|}
\hline
\textbf{Actor} & Customer, Barber Owner \\
\hline
\textbf{Mô tả} & Xem và cập nhật thông tin cá nhân \\
\hline
\textbf{Luồng chính} & 
1. Truy cập hồ sơ cá nhân \newline
2. Xem thông tin hiện tại \newline
3. Chỉnh sửa thông tin (tên, SĐT, ảnh đại diện) \newline
4. Lưu thay đổi \newline
5. Hệ thống cập nhật thành công \\
\hline
\textbf{Luồng phụ} & 
Extend: Cập nhật số điện thoại \newline
Extend: Đổi mật khẩu (yêu cầu mật khẩu cũ) \newline
Extend: Cập nhật ảnh đại diện \\
\hline
\textbf{Hậu điều kiện} & Thông tin cá nhân được cập nhật \\
\hline
\end{tabular}
\end{table}
Hệ thống website thương mại điện tử có hai tác nhân chính tham gia vào quá trình sử dụng hệ thống, bao gồm:
\begin{itemize}
    \item \textbf{Người mua (Customer)}: Là người sử dụng hệ thống để tìm kiếm, lựa chọn sản phẩm, đặt hàng,
    thanh toán, theo dõi đơn hàng và đánh giá sản phẩm.
    \item \textbf{Người bán (Admin/Người quản trị)}: Là người quản lý toàn bộ hoạt động của hệ thống,
    bao gồm quản lý sản phẩm, đơn hàng, người dùng, xử lý giao hàng và theo dõi doanh thu.
\end{itemize}






\subsection{Use case Diagram chi tiết về xử lý thanh toán}
\begin{figure}[H]
    \centering
    \includegraphics[width=1\textwidth]{graphics/main/chapter2/section2/usecase_vnpay.jpg}
    \caption{Use case Diagram xử lý thanh toán}
    \label{fig:usecase_payment}
\end{figure}
Dưới đây là các đặc tả use case của quy trình thanh toán qua cổng VNPAY.
\begin{table}[H]
\centering
\caption{Đặc tả Use Case: Thanh toán khi nhận hàng (COD)}
\begin{tabular}{|p{4cm}|p{10cm}|}
\hline
\textbf{Thông tin} & \textbf{Mô tả} \\
\hline
Tên Use Case & Thanh toán khi nhận hàng (COD) \\
\hline
Actor & Customer \\
\hline
Mô tả & Khách hàng chọn thanh toán tiền mặt khi nhận hàng \\
\hline
Điều kiện trước & Khách hàng đã thêm sản phẩm vào giỏ hàng và chọn phương thức thanh toán \\
\hline
Luồng chính & 
\begin{enumerate}[noitemsep]
    \item Khách hàng chọn phương thức "Thanh toán khi nhận hàng"
    \item Khách hàng xác nhận đặt hàng
    \item Hệ thống tạo đơn hàng mới với trạng thái "Chờ giao hàng"
    \item Hệ thống hiển thị thông báo đặt hàng thành công
    \item Hệ thống gửi email xác nhận đơn hàng
\end{enumerate} \\
\hline
Điều kiện sau & Đơn hàng được tạo thành công với trạng thái chờ giao hàng \\
\hline
\end{tabular}
\end{table}

\begin{table}[H]
\centering
\caption{Đặc tả Use Case: Thanh toán qua VNPay}
\begin{tabular}{|p{4cm}|p{10cm}|}
\hline
\textbf{Thông tin} & \textbf{Mô tả} \\
\hline
Tên Use Case & Thanh toán qua VNPay \\
\hline
Actor & Customer \\
\hline
Mô tả & Khách hàng thanh toán trực tuyến qua cổng thanh toán VNPay \\
\hline
Điều kiện trước & Khách hàng đã thêm sản phẩm vào giỏ hàng và chọn phương thức thanh toán \\
\hline
Luồng chính & 
\begin{enumerate}[noitemsep]
    \item Khách hàng chọn phương thức "Thanh toán VNPay"
    \item Khách hàng xác nhận thanh toán
    \item Hệ thống tạo đơn hàng tạm thời và mã giao dịch
    \item Hệ thống chuyển hướng khách hàng đến cổng VNPay
    \item Khách hàng nhập thông tin thanh toán và xác nhận
    \item VNPay xử lý giao dịch và gửi kết quả về hệ thống
    \item Hệ thống xác thực chữ ký và cập nhật trạng thái đơn hàng
    \item Hệ thống hiển thị kết quả thanh toán
    \item Hệ thống gửi email xác nhận thanh toán
\end{enumerate} \\
\hline
Luồng thay thế & 
\textbf{6a. Thanh toán thất bại:}
\begin{enumerate}[noitemsep]
    \item VNPay trả về kết quả thất bại
    \item Hệ thống cập nhật trạng thái "Thanh toán thất bại"
    \item Hệ thống hiển thị thông báo lỗi
    \item Đề xuất khách hàng thử lại hoặc chọn phương thức khác
\end{enumerate} \\
\hline
Điều kiện sau & Đơn hàng được tạo với trạng thái "Đã thanh toán" hoặc "Thanh toán thất bại" \\
\hline
\end{tabular}
\end{table}

\begin{table}[H]
\centering
\caption{Đặc tả Use Case: Xem lịch sử đơn hàng}
\begin{tabular}{|p{4cm}|p{10cm}|}
\hline
\textbf{Thông tin} & \textbf{Mô tả} \\
\hline
Tên Use Case & Xem lịch sử đơn hàng \\
\hline
Actor & Customer \\
\hline
Mô tả & Khách hàng xem danh sách các đơn hàng đã đặt \\
\hline
Điều kiện trước & Khách hàng đã đăng nhập vào hệ thống \\
\hline
Luồng chính & 
\begin{enumerate}[noitemsep]
    \item Khách hàng truy cập trang lịch sử đơn hàng
    \item Hệ thống hiển thị danh sách đơn hàng của khách hàng
    \item Khách hàng xem chi tiết từng đơn hàng
\end{enumerate} \\
\hline
Điều kiện sau & Danh sách đơn hàng được hiển thị thành công \\
\hline
\end{tabular}
\end{table}

\begin{table}[H]
\centering
\caption{Đặc tả Use Case: Quản lý đơn hàng}
\begin{tabular}{|p{4cm}|p{10cm}|}
\hline
\textbf{Thông tin} & \textbf{Mô tả} \\
\hline
Tên Use Case & Quản lý đơn hàng \\
\hline
Actor & Admin \\
\hline
Mô tả & Admin quản lý và cập nhật trạng thái các đơn hàng trong hệ thống \\
\hline
Điều kiện trước & Admin đã đăng nhập vào hệ thống \\
\hline
Luồng chính & 
\begin{enumerate}[noitemsep]
    \item Admin truy cập trang quản lý đơn hàng
    \item Hệ thống hiển thị danh sách tất cả đơn hàng
    \item Admin xem chi tiết đơn hàng
    \item Admin cập nhật trạng thái đơn hàng (đang xử lý, đang giao, hoàn thành, hủy)
    \item Hệ thống lưu thay đổi và thông báo cho khách hàng
\end{enumerate} \\
\hline
Điều kiện sau & Trạng thái đơn hàng được cập nhật thành công \\
\hline
\end{tabular}
\end{table}


\subsection{Use case Diagram chi tiết về phân tích da mặt}
\begin{figure}[H]
    \centering
    \includegraphics[width=1\textwidth]{graphics/main/chapter2/section2/usecase_acne.jpg}
    \caption{Use case Diagram phân tích da mặt}
    \label{fig:usecase_acne}
\end{figure}
\begin{table}[H]
\centering
\caption{Đặc tả Use Case: Phân tích Da mặt}
\begin{tabular}{|p{4cm}|p{10cm}|}
\hline
\textbf{Thông tin} & \textbf{Mô tả} \\
\hline
Tên Use Case & Phân tích Da mặt \\
\hline
Actor & Customer \\
\hline
Mô tả & Khách hàng upload ảnh khuôn mặt để phân tích tình trạng da và nhận gợi ý sản phẩm phù hợp \\
\hline
Điều kiện trước & Khách hàng đã đăng nhập vào hệ thống và có ảnh khuôn mặt rõ nét \\
\hline
Luồng chính & 
\begin{enumerate}[noitemsep]
    \item Khách hàng truy cập chức năng phân tích da mặt
    \item Khách hàng upload ảnh khuôn mặt
    \item Hệ thống sử dụng MediaPipe để nhận diện khuôn mặt
    \item Hệ thống cắt từng vùng da mặt (trán, má, mũi, cằm)
    \item Hệ thống sử dụng mô hình CNN để phân tích mụn trên từng vùng
    \item Hệ thống trả về kết quả phân tích chi tiết
    \item Hệ thống đưa ra lời khuyên chăm sóc da
    \item Hệ thống gợi ý các sản phẩm phù hợp với tình trạng da
\end{enumerate} \\
\hline
Luồng thay thế & 
\textbf{3a. Không phát hiện khuôn mặt:}
\begin{enumerate}[noitemsep]
    \item Hệ thống thông báo lỗi "Không tìm thấy khuôn mặt"
    \item Yêu cầu khách hàng upload lại ảnh rõ nét hơn
    \item Quay lại bước 2
\end{enumerate}
\textbf{3b. Ảnh không đủ chất lượng:}
\begin{enumerate}[noitemsep]
    \item Hệ thống thông báo "Ảnh không đủ rõ nét"
    \item Đề xuất khách hàng chụp ảnh trong điều kiện ánh sáng tốt
    \item Quay lại bước 2
\end{enumerate} \\
\hline
Điều kiện sau & Khách hàng nhận được kết quả phân tích da, lời khuyên chăm sóc và danh sách sản phẩm gợi ý \\
\hline
\end{tabular}
\end{table}


% usecase đánh giá sản phẩm admin
\subsection{Use case Diagram đánh giá sản phẩm admin}
\begin{figure}[H]
    \centering
    \includegraphics[width=1\textwidth]{graphics/main/chapter2/section2/usecasedgamin.png}
    \caption{Use case Diagram đánh giá sản phẩm }
    \label{fig:usecase_addg}
\end{figure}
 
%đặc tả admin
\subsubsection{Đặc tả Use Case Quản lý đánh giá sản phẩm}

\begin{table}[H]
\centering
\renewcommand{\arraystretch}{1.3}
\begin{tabular}{|p{4cm}|p{10cm}|}
\hline
\textbf{Thuộc tính} & \textbf{Nội dung} \\ \hline

Tên Use Case & Quản lý đánh giá sản phẩm \\ \hline

Actor & Admin \\ \hline

Mô tả & 
Cho phép Admin xem, kiểm duyệt hoặc xóa các đánh giá sản phẩm do khách hàng gửi lên hệ thống. \\ \hline

Điều kiện tiên quyết & 
Admin đã đăng nhập thành công vào hệ thống. \\ \hline

Hậu điều kiện & 
Đánh giá được cập nhật trạng thái (hiển thị/ẩn) hoặc bị xóa khỏi hệ thống. \\ \hline

Luồng chính & 
1. Admin truy cập trang quản lý đánh giá. \newline
2. Hệ thống hiển thị danh sách đánh giá của khách hàng. \newline
3. Admin chọn một đánh giá cụ thể. \newline
4. Admin có thể lọc đánh giá. \newline
4. Thực hiện thao tác duyệt, ẩn hoặc xóa đánh giá. \newline
5. Hệ thống cập nhật dữ liệu và hiển thị danh sách mới. \\ \hline

Luồng thay thế & 
Nếu xảy ra lỗi hệ thống hoặc không tìm thấy đánh giá, hệ thống hiển thị thông báo lỗi và không thực hiện thay đổi. \\ \hline

\end{tabular}
\caption{Đặc tả Use Case Quản lý đánh giá sản phẩm (Admin)}
\end{table}

% usecase khách hàng đánh giá sản phẩm 
\subsection{Use case Diagram khách hàng đánh giá sản phẩm}
\begin{figure}[H]
    \centering
    \includegraphics[width=1\textwidth]{graphics/main/chapter2/section2/usecasecusdg.png}
    \caption{Use case Diagram đánh giá sản phẩm }
    \label{fig:usecase_gdg}
\end{figure}
 
%đặc tả khách hàng đánh giá sản phẩm 
\subsubsection{Đặc tả Use Case Khách hàng đánh giá sản phẩm}

\begin{table}[H]
\centering
\renewcommand{\arraystretch}{1.3}
\begin{tabular}{|p{4cm}|p{10cm}|}
\hline
\textbf{Thuộc tính} & \textbf{Nội dung} \\ \hline

Tên Use Case & Đánh giá sản phẩm \\ \hline

Tác nhân & Customer \\ \hline

Mô tả & 
Cho phép khách hàng đánh giá sản phẩm đã mua khi đơn hàng ở trạng thái hoàn thành. \\ \hline

Điều kiện tiên quyết & 
- Khách hàng đã đăng nhập. \newline
- Đơn hàng đã hoàn thành. \\ \hline

Hậu điều kiện & 
Đánh giá được lưu vào hệ thống và hiển thị tại trang sản phẩm. \\ \hline

Luồng chính & 
1. Khách hàng đăng nhập. \newline
2. Truy cập mục “Đơn hàng của tôi”. \newline
3. Chọn đơn hàng đã hoàn thành. \newline
4. Chọn “Đánh giá sản phẩm”. \newline
5. Nhập nội dung và chọn số sao. \newline
6. Gửi đánh giá. \newline
7. Hệ thống lưu và thông báo thành công. \\ \hline

Luồng mở rộng (Extend) & 
- Đánh giá sao (1–5 sao). \newline
- Viết bình luận chi tiết. \newline
- Gửi ảnh/video minh họa. \newline
- Sửa đánh giá sau khi đã gửi. \\ \hline

Luồng thay thế & 
- Đơn hàng chưa hoàn thành → không được đánh giá. \newline
- Nội dung vi phạm → hệ thống từ chối. \newline
- Ảnh/video sai định dạng → báo lỗi. \\ \hline

\end{tabular}
\caption{Đặc tả Use Case Khách hàng đánh giá sản phẩm}
\end{table}
