\section{Xây dựng Use Case Diagram}
\subsection{Use case Diagram tổng quát}
% \begin{figure}[H]
%     \centering
%     \includegraphics[scale=0.21]{figures/usecase_tq.jpg}
%     \caption{Use case Diagram tổng quát}
%     \label{fig:usecase_overall}
% \end{figure}
Các Actor trong Hệ thống bao gồm:
\begin{itemize}[noitemsep]
    \item Customer: Người dùng cuối sử dụng ứng dụng mobile để đặt lịch cắt tóc, nhận gợi ý kiểu tóc và tư vấn chăm sóc tóc thông qua hệ thống AI.

    \item Barber Owner: Chủ cửa tiệm hoặc người quản lý sử dụng ứng dụng mobile để quản lý cửa hàng, dịch vụ, thợ cắt tóc và xử lý các lịch hẹn của khách hàng.

    \item Admin: Người quản trị hệ thống sử dụng web admin panel để quản lý người dùng, cửa tiệm, cấu hình hệ thống và giám sát hoạt động của AI chatbot.

    \item AI System: Hệ thống trí tuệ nhân tạo tự động hỗ trợ khách hàng thông qua phân tích khuôn mặt, gợi ý kiểu tóc và chatbot tư vấn.
\end{itemize}
%----------------------------------------------------------------Đăng nhập đăng ký---------------
\subsection{Use case Diagram chi tiết về xử lý đăng nhập đăng ký}
\begin{figure}[H]
    \centering
    \includegraphics[width=0.8\textwidth]{graphics/main/chapter2/section2/usecase_authen.jpg}
    \caption{Use case Diagram người dùng đăng nhập đăng ký}
    \label{fig:usecase_user}
\end{figure}
Dưới đây là các đặc tả use case của người dùng đăng nhập đăng ký.
\begin{table}[H]
    \centering
    \caption{Đặc tả Use Case UC-01: Đăng ký tài khoản}
    \label{tab:uc01}
\begin{tabular}{|p{4cm}|p{11cm}|}
    \hline
    \textbf{Actor} & Customer, Barber Owner \\
    \hline
\textbf{Mô tả} & Người dùng tạo tài khoản mới \\
\hline
\textbf{Tiền điều kiện} & Chưa có tài khoản \\
\hline
\textbf{Luồng chính} & 
1. Nhập thông tin (email, mật khẩu, họ tên) \newline
2. Hệ thống xác thực thông tin \newline
3. Gửi xác thực qua email \newline
4. Click vào link email nhận được \newline
5. Tạo tài khoản thành công \\
\hline
\textbf{Luồng phụ} & 
2a. Thông tin không hợp lệ $\rightarrow$ Hiển thị lỗi \newline
4a. Link sai/hết hạn $\rightarrow$ Gửi lại email \\
\hline
\textbf{Hậu điều kiện} & Tài khoản được tạo và kích hoạt \\
\hline
\end{tabular}
\end{table}



\begin{table}[H]
\centering
\caption{Đặc tả Use Case UC-02: Đăng nhập}
\label{tab:uc02}
\begin{tabular}{|p{4cm}|p{11cm}|}
\hline
\textbf{Actor} & Customer, Barber Owner, Admin \\
\hline
\textbf{Mô tả} & Đăng nhập vào hệ thống \\
\hline
\textbf{Tiền điều kiện} & Có tài khoản hợp lệ \\
\hline
\textbf{Luồng chính} & 
1. Nhập email và mật khẩu \newline
2. Hệ thống xác thực \newline
3. Cấp token truy cập \newline
4. Chuyển đến màn hình chính \\
\hline
\textbf{Luồng phụ} & 
2a. Sai thông tin đăng nhập $\rightarrow$ Thông báo lỗi \newline
2b. Tài khoản bị khóa $\rightarrow$ Thông báo liên hệ admin \\
\hline
\textbf{Hậu điều kiện} & Người dùng truy cập được hệ thống \\
\hline
\end{tabular}
\end{table}



\begin{table}[H]
\centering
\caption{Đặc tả Use Case UC-03: Đăng xuất}
\label{tab:uc03}
\begin{tabular}{|p{4cm}|p{11cm}|}
\hline
\textbf{Actor} & Customer, Barber Owner, Admin \\
\hline
\textbf{Mô tả} & Thoát khỏi hệ thống \\
\hline
\textbf{Luồng chính} & 
1. Chọn đăng xuất \newline
2. Xóa token và session \newline
3. Chuyển về màn hình đăng nhập \\
\hline
\textbf{Hậu điều kiện} & Phiên làm việc kết thúc \\
\hline
\end{tabular}
\end{table}



\begin{table}[H]
\centering
\caption{Đặc tả Use Case UC-04: Quản lý hồ sơ cá nhân}
\label{tab:uc04}
\begin{tabular}{|p{4cm}|p{11cm}|}
\hline
\textbf{Actor} & Customer, Barber Owner \\
\hline
\textbf{Mô tả} & Xem và cập nhật thông tin cá nhân \\
\hline
\textbf{Luồng chính} & 
1. Truy cập hồ sơ cá nhân \newline
2. Xem thông tin hiện tại \newline
3. Chỉnh sửa thông tin (tên, SĐT, ảnh đại diện) \newline
4. Lưu thay đổi \newline
5. Hệ thống cập nhật thành công \\
\hline
\textbf{Luồng phụ} & 
Extend: Cập nhật số điện thoại \newline
Extend: Đổi mật khẩu (yêu cầu mật khẩu cũ) \newline
Extend: Cập nhật ảnh đại diện \\
\hline
\textbf{Hậu điều kiện} & Thông tin cá nhân được cập nhật \\
\hline
\end{tabular}
\end{table}
Hệ thống website thương mại điện tử có hai tác nhân chính tham gia vào quá trình sử dụng hệ thống, bao gồm:
\begin{itemize}
    \item \textbf{Người mua (Customer)}: Là người sử dụng hệ thống để tìm kiếm, lựa chọn sản phẩm, đặt hàng,
    thanh toán, theo dõi đơn hàng và đánh giá sản phẩm.
    \item \textbf{Người bán (Admin/Người quản trị)}: Là người quản lý toàn bộ hoạt động của hệ thống,
    bao gồm quản lý sản phẩm, đơn hàng, người dùng, xử lý giao hàng và theo dõi doanh thu.
\end{itemize}

\subsection{Các nhóm Use Case chính}

Dựa trên yêu cầu chức năng của hệ thống, các Use Case được phân thành các nhóm chính như sau:

\subsubsection{Nhóm Use Case dành cho người mua}

Nhóm Use Case dành cho người mua bao gồm các chức năng hỗ trợ hoạt động mua sắm trực tuyến,
nhằm mang lại trải nghiệm thuận tiện và dễ sử dụng cho khách hàng.

Các Use Case chính bao gồm:
\begin{itemize}
    \item Đăng ký, đăng nhập hệ thống
    \item Duyệt và tìm kiếm sản phẩm
    \item Xem chi tiết sản phẩm
    \item Quản lý giỏ hàng
    \item Đặt hàng và thanh toán
    \item Theo dõi trạng thái đơn hàng
    \item Đánh giá và nhận xét sản phẩm
    \item Nhận tư vấn từ chatbot AI và AI gợi ý sản phẩm phù hợp
\end{itemize}

\subsubsection{Nhóm Use Case dành cho người bán}

Nhóm Use Case dành cho người bán tập trung vào các chức năng quản trị và vận hành hệ thống,
đảm bảo quá trình bán hàng và chăm sóc khách hàng diễn ra hiệu quả.

Các Use Case chính bao gồm:
\begin{itemize}
    \item Quản lý tài khoản người dùng
    \item Quản lý danh mục và sản phẩm
    \item Quản lý đơn hàng
    \item Cập nhật trạng thái giao hàng
    \item Quản lý phương thức thanh toán
    \item Theo dõi thống kê doanh thu và báo cáo tổng hợp
    \item Quản lý đánh giá và phản hồi từ khách hàng
\end{itemize}

\subsection{Sơ đồ Use Case tổng quát của hệ thống}

Sơ đồ Use Case tổng quát thể hiện mối quan hệ giữa các tác nhân và các chức năng chính của hệ thống website thương mại điện tử.
Thông qua sơ đồ, có thể thấy rõ phạm vi chức năng của hệ thống cũng như trách nhiệm của từng tác nhân.

% \begin{figure}[H]
%     \centering
%     \includegraphics[width=0.95\textwidth]{graphics/front/tmdt.drawio.png}
%     \caption{Sơ đồ Use Case tổng quát của website thương mại điện tử}
%     \label{fig:usecase}
% \end{figure}

\subsection{Nhận xét}

Từ sơ đồ Use Case, có thể nhận thấy hệ thống đáp ứng đầy đủ các yêu cầu chức năng đã được đề cập trong phần
đặc tả yêu cầu hệ thống. Việc phân chia rõ ràng các Use Case theo từng tác nhân giúp hệ thống
dễ dàng mở rộng, bảo trì và phát triển trong tương lai.
